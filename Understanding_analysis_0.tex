\chapter{序}

撰写\textit{Understanding Analysis}的首要目标,是打造一本基础性的单学期教材,让学生领略以数学严谨方式研究实变函数所带来的丰厚回报。实分析课程应当致力于挑战并提升数学直觉,而非验证直觉。然而入门课程往往过度围绕标准微积分序列中的熟悉定理展开。虽然用严谨论证证明多项式连续性能很好体现连续性定义的恰当选择,但这并非该学科诞生的初衷,更非其应列为必修的理由。通过将焦点转向未经训练的直觉极易失效的领域(如无穷级数重排、无处可导的连续函数、Fourier级数),我试图让初学者接触学科真正重要的成就,从而重焕这门课程的学术活力。

\section*{核心目标}

近年来,标准本科数学课程体系持续承受着来自多方面的压力。随着计算机技术日益普及,数学思维能发挥价值的领域也在扩展。如今大多数数学专业学生并非为纯数学研究生阶段做准备,而是期待进入银行、医学、法律等需要分析技能的领域。另一重要影响是持续十余年的微积分教学改革运动,其合理核心在于以更直观的方式呈现微积分,强调几何论证而非符号推演。尽管存在这些趋势——或许正因如此——几乎所有本科数学项目仍要求至少一学期的实分析课程。这导致教师面临更艰巨的任务:向更不熟悉公理化论证本质的多元化学生群体教授这门艰深的抽象课程。

关键在于,任何向更广泛群体推广数学的思潮,最终都必须直面理论分析极具挑战性甚至令人望而生畏的事实。当前存在一种令人遗憾的折衷方案:通过削减趣味性来降低课程难度。被删减的恰恰是体现分析学精髓的内容。更好的解决之道,是找到让高阶主题既易于理解又值得钻研的传授方式。

我认为实分析课程应达成三项基本目标:
\begin{enumerate}
\item 尤其对经历微积分改革教学的学生,需说服他们认识函数严谨研究的必要性。必须精心阐释精确定义和公理化方法的不可或缺性。
\item 在主要接触图形化、数值化或直觉化论证后,学生需要学习严谨数学证明的构成要素与写作规范。
\item 必须让夯实极限逻辑结构的艰苦工作获得丰厚回报。具体而言,实分析不应只是标准入门微积分的精细重述。学生应当接触实数线的精妙复杂性、各类收敛的微妙差异,以及无限悖论中蕴藏的智识乐趣。
\end{enumerate}


\textit{Understanding Analysis}的核心理念是聚焦那些赋予分析学内在魅力的问题。Cantor 集是否包含无理数?函数不连续点的集合能否任意构造?导数是否连续?导数是否可积?无限可微函数是否必为Taylor级数极限?将这些主题置于中心位置,其意义在于:没有严谨分析的艰苦工作,就无从触及这些问题的本质。

\section*{本书结构}

作为入门教材,本书虽引入若干精深主题作为后续内容的预告与动机,但各章主体仍精炼聚焦于构成分析课程核心的基础内容。完备性、紧性、序列与函数极限、连续性、一致收敛、微分与积分等核心结论均有涵盖。特色在于重点布局:例如积分章节以解读连续性与 Riemann 积分关系为主线,虽涵盖证明微积分基本定理所需的积分性质,但核心脉络是探索基于连续性的可积函数特征刻画。无论是否涉及Lebesgue 零测准则,这种问题导向的框架设计极具价值——正是这些问题本身才至关重要。数学绝非静态学科,学生应当了解所学数学的历史成因,进而认识到该学科永无止境。关于积分,本书末章补充内容涉及广义 Riemann 积分的最新进展,正是对此理念的明确诠释。

各章结构具有以下显著特征。

\paragraph{讨论环节:}每章开篇通过启发性案例和开放性问题展开探讨。这部分行文刻意保持非正式风格,充分运用微积分中的常见函数与结论,旨在自由探索知识疆域,为后续定义和定理奠定语境。一个反复出现的主题是解决有限情境操作被简单推广到无限情境时产生的悖论(例如逐项微分无穷级数、调换双重求和的顺序)。这些探索性引言之后,文风转为严谨凝练但仍避免过度形式化。当问题框架确立后,后续内容的展开便有了充分动机,学习成效也清晰可期。

\paragraph{项目环节:}每章倒数第二节(末节为简短结语)采用将练习融入讲解的写作方式。证明过程仅勾勒轮廓而不完整呈现,另附补充练习以阐明讨论内容。这种设计旨在提供灵活性——既可作自学指南,也能作为授课主题。我曾用其替代期末考试,尤其适合以小组汇报为终点的协作作业。各章正文已包含必要工具,让学生运用新学技能自主探寻推动论述的核心问题答案,能带来显著的教学满足感。

\section*{课程构建}

成功的教学总是与时间赛跑。尽管本书按12-14周学期设计,仍需对教学内容做出取舍。

\begin{itemize}
\item 引言部分可讨论、布置阅读、省略或用优选内容替代。此处未证明任何后续出现的定理,但发展了若干重要案例(Cantor集、Dirichlet无处连续函数),这些终需在教学中涉及。
\item 第\ref{chap:3}章 \(\mathbb{R}\) 的基础拓扑学内容远超必需。后续章节仅需开闭集基本结论及对列紧性的透彻理解。基于开覆盖的紧性刻画及完备/连通集章节因其理论价值而保留,但对后续证明非关键。例外是将中值定理(IVT)呈现为连续函数保持连通性特例的讲法。为使连通性真正可选,本书包含IVT的两个直接证明:一个用上确界,另一个用区间套。完备集同理——虽然Baire纲定理的证明可借完备集不可数来引介,但二者可独立处理。

\item 所有项目环节(\ref{sec:1.5},\ref{sec:2.8},\ref{sec:3.5},\ref{sec:4.6},\ref{sec:5.4},\ref{sec:6.6},\ref{sec:7.6},\ref{sec:8.1}-\ref{sec:8.4})均为选修,因后续章节不依赖这些内容。第\ref{chap:8}章的四个专题也采用这种练习驱动的项目式写法,其中仅Fourier级数单元因篇幅较长可能需要专门讲解。
\end{itemize}

\section*{读者定位}

本课程唯一前提是扎实掌握单变量微积分结论。虽不需线性代数定理,但具备该课程培养的抽象论证与证明写作能力将大有裨益。本书完全不涉及复数。

\textit{Understanding Analysis}的证明写作始终心系初学者。我们牺牲简洁性与文体考量,以详尽阐述为优先。多数证明都附带关于论证语境的讨论:证明应包含什么?哪些定义相关?整体策略为何?是否类似已有证明?当面临选择时,我们舍弃效率以强化既有技法。特别熟悉或可预测的论证通常以练习形式略述,让学生直接参与核心内容的建构。

寻找重复出现的理念既存在于证明写作层面,也存在于更宏观的阐述层面。我试图通过把握近似性这一统一主题以及从有限到无限的过渡,为课程赋予叙事性语调。援引书末的一段话:实数由有理数逼近;连续函数值由邻近值逼近;曲线由直线逼近;面积由矩形和逼近;连续函数由多项式逼近。每种情形下,逼近对象都是具体且易于理解的,关键在于这些特性何时以及如何在极限过程中得以保留。通过聚焦这种反复出现的模式,每个后续主题都建立在前一个的直觉基础上。问题显得更自然,而从原本可能看似冗长的定理和证明列表中,浮现出混乱中的方法论。

本书始终强调核心思想而非普适性,并不追求成为完整演绎的结果目录。它旨在激发智识想象力。感兴趣的读者将为从更一般空间上的复值函数开始的进阶课程做好充分准备,而满足于单学期课程者也能深刻领会实分析的精髓与目的。再次引用第八章结尾:"通过用有限对象构建的路径审视数学中不同的无限性,Weierstrass和其他分析学奠基人创建了范式,将数学探索的疆域拓展到先前无法企及的领域。"

这段探索构成了我智识生涯的主要兴奋点。怀着无比欣喜,我将这份指南献给这段旅程中那些最令人惊叹的亮点。祝您旅途愉快!

\section*{致谢}

本书的雏形源于与瓦萨学院的 Benjamin Lotto 的长期对话。早期章节结构与全书主旨很大程度上源于我们多年间共享的课堂笔记、思想与经验。书稿最终呈现的效果令我欣慰,毫无疑问,正是本杰明早期的贡献使其成为一部远更出色的著作。

主要写作完成于我在弗吉尼亚大学访学期间。特别感谢Nat Martin和Larry Thomas的时间与智慧馈赠。尤其感谢 Loren Pitt,其建议范围远超本书范畴。同时感谢 Julie Riddleberger 对图表制作的协助。贝拉明学院的 Marian Robbins、卡尔顿学院的Steve Kennedy、圣奥拉夫学院的Paul Humke以及弗吉尼亚大学的Tom Kriete均采用本书初稿授课。我珍视这个团队提出的诸多改进建议,并要特别感谢Paul Humke对积分章节的贡献。

明德学院院系与行政部门对本项目给予了大力支持。David Guertin 多次提供技术援助,Priscilla Bremser 审阅了早期章节草稿,Rick Chartrand 的深刻见解极大改善了后续部分内容。历经本书漫长演变过程的学生名单已无法尽述,但我要特别提及 Brooke Sargent——其细致的课堂笔记构成了初稿基础,以及 Jesse Johnson——为完善书中大量习题的呈现方式付出了不懈努力。Springer制作团队堪称一流,我向所有人致以诚挚谢意,尤其要感谢Sheldon Axler 远超其职责范围的鼓励与建议。

最近重读已完成的书稿时,我惊觉自己频繁借助历史背景来引出一个观点。这并非我刻意设定的目标,而是反映了数学教育中一个令人振奋的趋势——通过学科历史赋予数学人文温度。根据个人经验,分析学领域的这一变革很大程度上归功于两部著作:David Bressoud 的 \textit{A Radical Approach to Real Analysis},以及 E. Hairer 与 G. Wanner合著的\textit{Analysis by Its History}。 Bressoud 的著作对最后一章 Fourier 级数的阐述影响尤为深远,这两本书都可作为本课程的绝佳补充资料。虽然我会在具有启发性或特别重要的地方尽力标注历史渊源,但本书呈现的许多定理如今已成为学科共识,故未逐一考证归属。唯一的例外是关于广义 Riemann 积分的内容——该理论由Jaroslav Kurzweil 和 Ralph Henstock各自独立提出,第\ref{sec:8.1}节严格遵循了Robert Bartle 在 \textit{Return to the Riemann Integral} 一文中的论述。这位作者在文中以斜体强调呼吁数学教师用广义黎曼积分取代无处不在的 Lebesgue 积分。但愿 Bartle 教授能将本书对此的收录视为对其呼吁的热切响应。

就个人而言,我欢迎任何性质的反馈,并将通过网页链接分享所有富有启发性的意见——以及修正建议。从萌生想法到本书出版已近四年光阴,这段漫长旅程离不开众多人士的坚定支持,尤其是我非凡的妻子 Katy。在这个庞大项目涉及的纷繁决策与艰辛工作中,能将本书献予她,是纯粹而轻盈的喜悦。

\begin{flushright}
Stephen Abbott
\end{flushright}
明德学院,佛蒙特州

2000年8月