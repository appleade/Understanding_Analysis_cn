\chapter{实数}
\label{chap:1}
\section{讨论: \(\sqrt{2}\) 的无理性}
\label{sec:1.1}
在他杰出的职业生涯接近尾声时,著名的英国数学家 G.H. Hardy 在1940年首次发表的《一个数学家的辩白(A Mathematician’s Apology)》一文中,雄辩地阐述了研究数学的理由。Hardy辩护的核心论点是,数学是一门美学学科。对于Hardy来说,工程师和经济学家所应用的那些数学几乎没有吸引力。他所说的“真正的数学”,“如果要证明其合理性,就必须像艺术一样被证明。”

为了帮助阐明他的观点,Hardy引用了两个来自古希腊数学的定理,他认为这些定理具有一种难以捉摸的美,尽管难以定义,但很容易识别。第一个结果是 Euclid 证明的素数有无限多个。第二个结果是由公元前500年左右的 Pythagoras 学派发现的,即 \(\sqrt{2}\) 是无理数。正是这第二个定理值得我们关注。(数论课程会重点讨论第一个定理。)这个论证仅使用了算术,但其深度和重要性不容小觑。正如Hardy所说:“[它]是一个‘简单’的定理,无论是在思想还是执行上都很简单,但毫无疑问,它属于最高级别的定理。[它]就像刚被发现时一样新鲜和重要——两千年的时光没有在它身上留下任何痕迹。”

\begin{Thm}
  \label{thm:1.1.1}
  不存在平方为2的有理数。
\end{Thm}

\begin{proof}
有理数是指可以表示为 \(p/q\) 形式的数,其中 \(p\) 和 \(q\) 是整数。因此,该定理断言的是,无论 \(p\) 和 \(q\) 如何选择, \({\left( p/q\right) }^{2} = 2\) 的情况永远不会发生。我们的论证方法是间接的,使用了一种称为反证法的论证类型。其思路是假设存在一个有理数,其平方为2,然后沿着逻辑线进行推理,直到得出一个不可接受的结论。此时,我们将被迫回溯我们的步骤,并拒绝某些有理数的平方等于2的错误假设。简而言之,我们将通过证明该定理不可能为假来证明其为真。

因此,为了进行反证,假设存在整数 \(p\) 和 \(q\) 满足

\begin{equation}
\label{eq:1.1}
{\left( \frac{p}{q}\right) }^{2} = 2
\end{equation}

我们还可以假设 \(p\) 和 \(q\) 没有公因数,因为如果它们有公因数,我们可以简单地将其约去,并将分数改写为最简形式。现在,方程~\eqref{eq:1.1}意味着

\begin{equation}
\label{eq:1.2}
{p}^{2} = 2{q}^{2}.
\end{equation}

由此可见,整数 \({p}^{2}\) 是一个偶数(它可以被$2$整除),因此 \(p\) 也必须是偶数,因为奇数的平方是奇数。这使得我们可以将 \(p = {2r}\) 表示为 \(r\) 的平方,其中 \(r\) 也是一个整数。如果我们将 \({2r}\) 代入方程~\eqref{eq:1.2}中的 \(p\) ,经过一些代数运算,可以得到以下关系

\[
2{r}^{2} = {q}^{2}.
\]

但现在荒谬之处显而易见。最后一个方程表明 \({q}^{2}\) 是偶数,因此 \(q\) 也必须是偶数。因此,我们已经证明,当 \(p\) 和 \(q\) 最初被假设为没有公因数时,它们都是偶数(即可以被2整除)。从这个逻辑困境中,我们只能得出结论:方程~\eqref{eq:1.1}对于任何整数 \(p\) 和 \(q\) 都不成立,因此定理得证。  
\end{proof}

Hardy 对数学定理中美的定义的一个组成部分是:数学美的结果对其他数学思想网络具有持久而深远的影响。在这种情况下,受到冲击的是希腊人对几何长度与算术数之间关系的理解。在此发现之前,人们普遍假设并常用的事实是:给定两条线段 \(\overline{AB}\) 和 \(\overline{CD}\) ,总能找到第三条线段,其长度可以均匀地分割前两条线段。用现代术语来说,这相当于断言 \(\overline{CD}\) 的长度是 \(\overline{AB}\) 长度的有理数倍。然而,观察单位正方形的对角线(图\ref{fig:1.1}),现在可以(使用勾股定理)得出情况并非总是如此。由于毕达哥拉斯学派隐含地将数解释为有理数,他们被迫接受数的概念严格弱于长度的概念。

\begin{figure}[h]
  \centering
  \includegraphics[width=0.2\textwidth]{images/01955a90-3fc5-700a-bdb0-df1c9e2a30b7_2_651_400_296_277_0.jpg}
  \caption{作为几何长度的 \(\sqrt{2}\)}
  \label{fig:1.1}
\end{figure}

与其放弃算术而偏爱几何(正如希腊人似乎所做的那样),我们解决这一限制的方法是通过从有理数转向更大的数系来加强数的概念。从现代的角度来看,这应该是一个熟悉且有些自然的现象。我们从自然数开始

\[
\mathbb{N} = \{ 1,2,3,4,5,\ldots \} \text{ . }
\]

德国数学家 Leopold Kronecker (1823-1891) 曾断言:“自然数是上帝的杰作,其余的一切都是人类的创造。” 讨论这一主张的有效性是另一个有趣的话题。目前,它至少为我们提供了一个起点。如果我们将注意力限制在自然数 \(\mathbb{N}\) 上,那么我们可以很好地执行加法,但如果我们想要有一个加法单位元(零)以及定义减法所需的加法逆元,我们必须将我们的系统扩展到整数:

\[
\mathbb{Z} = \{ \ldots , - 3, - 2, - 1,0,1,2,3,\ldots \}
\]



接下来的问题是乘法和除法。数字$1$是乘法单位元,但为了定义除法,我们需要有乘法逆元。因此,我们再次将我们的系统扩展到有理数

\[
\mathbb{Q} = \left\{ \frac{p}{q}\mid p,q\in \mathbb{Z}, q\ne 0 \right\}
\]

综上所述,上一段讨论的 \(\mathbb{Q}\) 的性质基本上构成了所谓“域”的定义。更正式地说,域是满足下列性质的集合:加法和乘法是定义良好的运算;满足交换律、结合律以及熟悉的分配律 \(a\left( {b + c}\right)  = {ab} + {ac}\) ;必须存在一个加法单位元,并且每个元素都必须有一个加法逆元;必须存在一个乘法单位元,并且域中所有非零元素都必须有乘法逆元。 \(\mathbb{Z}\) 和 \(\mathbb{N}\) 都不是域。有限集合 \(\{ 0,1,2,3,4\}\) 在加法和乘法按模$5$计算时是一个域。这一点并不立即显而易见,但可以作为一个有趣的练习(练习1.3.1)。

集合 \(\mathbb{Q}\) 上也有一个自然定义的序。给定任意两个有理数 \(r\) 和 \(s\) ,以下之一必然成立:

\[
r < s,\;r = s,\;\text{ 或}\;r > s.
\]

这种序关系具有传递性,即如果 \(r < s\) 且 \(s < t\) ,那么 \(r < t\) ,因此我们方便地将有理数想象为从左到右排列在数轴上。与 \(\mathbb{Z}\) 不同,这里没有间隔。给定任意两个有理数 \(r < s\) ,有理数 \(\left( {r + s}\right) /2\) 位于两者之间,这意味着有理数是紧密排列在一起的。

由于 \(\mathbb{Q}\) 的域属性允许我们安全地执行加、减、乘、除等代数运算,让我们回顾一下 \(\mathbb{Q}\) 所缺少的是什么。根据定理~\ref{thm:1.1.1},显然我们不能总是取平方根。然而,问题实际上比这更根本。仅使用有理数,我们可以很好地近似 \(\sqrt{2}\) (图~\ref{fig:1.2})。例如, \({1.414}^{2} = {1.999396}\) 。通过增加近似值的小数位数,我们可以更接近 \(\sqrt{2}\) 的值,但即便如此,我们现在清楚地意识到,在有理数线上 \(\sqrt{2}\) 应该存在的位置有一个“洞”。当然,还有其他许多洞——例如在 \(\sqrt{3}\) 和 \(\sqrt{5}\) 处。回到古希腊数学家的困境,如果我们希望数线上的每个长度都对应一个实际的数字,那么我们需要对数字系统进行另一次扩展。因此,我们在链 \(\mathbb{N} \subseteq  \mathbb{Z} \subseteq  \mathbb{Q}\) 中附加了实数 \(\mathbb{R}\) 。

\begin{figure}[h]
  \centering
  \includegraphics[width=0.7\textwidth]{images/01955a90-3fc5-700a-bdb0-df1c9e2a30b7_3_494_424_962_160_0.jpg}
  \caption{用有理数近似 \(\sqrt{2}\) }
  \label{fig:1.2}
\end{figure}


如何从 \(\mathbb{Q}\) 实际构造 \(\mathbb{R}\) 的问题相当复杂。这在第 \ref{sec:1.3} 节中进行了讨论,然后在第 \ref{sec:8.4} 节中更详细地再次讨论。目前,可以不太准确地说, \(\mathbb{R}\) 是通过填补 \(\mathbb{Q}\) 中的空白而获得的。每当有一个空洞时,就会定义一个新的无理数,并将其放入已经存在于 \(\mathbb{Q}\) 上的排序中。实数就是这些无理数与更熟悉的有理数的并集。无理数集具有哪些性质?有理数和无理数集如何结合在一起?在有理数和无理数之间是否存在某种对称性,或者我们能否在某种意义上论证一种类型的实数比另一种更常见?到目前为止,我们看到的生成无理数示例的唯一方法是通过平方根。不足为奇的是,其他根如 \(\sqrt[3]{2}\) 或 \(\sqrt[5]{3}\) 通常也是无理数。所有无理数是否都可以表示为 \(n\) 次根和有理数的代数组合,或者是否存在这种形式之外的其他无理数?

\section{一些预备知识}
\label{sec:1.2}

后续发展所需的词汇来自集合论和函数理论。这应该是熟悉的领域,但简要回顾一下术语可能是个好主意,即使这只是为了建立一些符号上的共识。

\subsection{集合}

直观地说,集合是任何对象的集合。这些对象被称为集合的元素。就我们的目的而言,所讨论的集合通常是实数集,尽管我们也会遇到函数集,并且在极少数情况下,还会遇到元素是其他集合的集合。

给定一个集合 \(A\) ,如果 \(x\) (无论它是什么)是 \(A\) 的元素,我们写作 \(x \in  A\) 。如果 \(x\) 不是 \(A\) 的元素,那么我们写作 \(x \notin  A\) 。给定两个集合 \(A\) 和 \(B\) ,并集写作 \(A \cup  B\) ,并通过断言以下内容来定义

\[
x \in  A \cup  B \Leftrightarrow x \in  A\text{ 或 }x \in  B\text{ (可能同时成立). }
\]

交集 \(A \cap  B\) 是由以下规则定义的集合

\[
x \in  A \cap  B \Leftrightarrow x \in  A\text{ 且 }x \in  B\text{ . }
\]

\begin{Eg}
  \label{eg:1.2.1}
  \begin{enumerate}[label = (\roman*)]
  \item 有许多可接受的方式来断言集合的内容。在前一节中,自然数集合通过列出元素来定义: \(\mathbb{N} = \{ 1,2,3,\ldots \}\) 。
  \item 集合也可以用文字描述。例如,我们可以将集合 \(E\) 定义为偶自然数的集合。
  \item 有时提供一种规则或算法来确定集合的元素更为高效。例如,设

\[
S = \left\{  {r \in  \mathbb{Q} : {r}^{2} < 2}\right\}  .
\]

写成自然语言, \(S\) 的定义是:“设 \(S\) 为所有平方小于2的有理数的集合。”由此可知 \(1 \in  S,4/3 \in  S\) ,但 \(3/2 \notin  S\) ——因为 \(9/4 \geq  2\) 。
  \end{enumerate}
\end{Eg}
 
使用之前定义的集合来说明交集和并集的操作,我们观察到

\[
\mathbb{N} \cup  E = \mathbb{N},\;\mathbb{N} \cap  E = E,\;\mathbb{N} \cap  S = \{ 1\} ,E \cap  S = \varnothing .
\]

集合 \(\varnothing\) 被称为空集,理解为不包含任何元素的集合。一个等价的陈述是说 \(E\) 和 \(S\) 是不相交的。

关于两个集合的相等性,有必要说明一下(因为我们刚刚使用了这个概念)。包含关系 \(A \subseteq  B\) 或 \(B \supseteq  A\) 用于表示 \(A\) 的每个元素也是 \(B\) 的元素。在这种情况下,我们说 \(A\) 是 \(B\) 的子集,或者说 \(B\) 包含 \(A\) 。断言 \(A = B\) 意味着 \(A \subseteq  B\) 和 \(B \subseteq  A\) 。换句话说, \(A\) 和 \(B\) 具有完全相同的元素。

在接下来的章节中,我们经常会想要将并集和交集操作应用于无限集合的集合。

\begin{Eg}
  \label{eg:1.2.2}
设

\[
{A}_{1} = \mathbb{N} = \{ 1,2,3,\ldots \} ,
\]

\[
{A}_{2} = \{ 2,3,4,\ldots \}
\]

\[
{A}_{3} = \{ 3,4,5,\ldots \}
\]

并且,一般来说,对于每个 \(n \in  \mathbb{N}\) ,定义集合

\[
{A}_{n} = \{ n,n + 1,n + 2,\ldots \} .
\]

结果是一个嵌套的集合链

\[
{A}_{1} \supseteq  {A}_{2} \supseteq  {A}_{3} \supseteq  {A}_{4} \supseteq  \cdots ,
\]

其中每个后续集合都是所有先前集合的子集。用符号表示,

\[
\mathop{\bigcup }\limits_{{n = 1}}^{\infty }{A}_{n},\;\mathop{\bigcup }\limits_{{n \in  \mathbb{N}}}{A}_{n},{A}_{1} \cup  {A}_{2} \cup  {A}_{3} \cup  \cdots
\]

都是表示集合的等价方式,该集合的元素由至少出现在一个特定 \({A}_{n}\) 中的任何元素组成。由于这个特定集合集的嵌套性质,不难看出

\[
\mathop{\bigcup }\limits_{{n = 1}}^{\infty }{A}_{n} = {A}_{1}
\]

交集的概念同样可以自然地扩展到无限集合集。对于这个例子,我们有

\[
\mathop{\bigcap }\limits_{{n = 1}}^{\infty }{A}_{n} = \varnothing .
\]

让我们确保理解为什么会出现这种情况。假设我们有某个自然数 \(m\) ,我们认为它可能实际上满足 \(m \in  \mathop{\bigcap }\limits_{{n = 1}}^{\infty }{A}_{n}\) 。这意味着 \(m \in  {A}_{n}\) 对于集合集中的每一个 \({A}_{n}\) 都成立。由于 \(m\) 不是 \({A}_{m + 1}\) 的元素,因此不存在这样的 \(m\) ,交集为空。  
\end{Eg}

如前所述,我们遇到的大多数集合都是实数集。给定 \(A \subseteq  \mathbb{R}\) , \(A\) 的补集,记作 \({A}^{c}\) ,指的是 \(\mathbb{R}\) 中不属于 \(A\) 的所有元素的集合。因此,对于 \(A \subseteq  \mathbb{R}\) ,

\[
{A}^{c} = \{ x \in  \mathbb{R} : x \notin  A\} .
\]

在接下来的工作中,我们会多次提到 De Morgan 定律,该定律指出

\[
{\left( A \cap  B\right) }^{c} = {A}^{c} \cup  {B}^{c}, \quad{\left( A \cup  B\right) }^{c} = {A}^{c} \cap  {B}^{c}.
\]

这些命题的证明在练习1.2.3中讨论。

诚然,本次讨论开始时提出的集合定义有些不精确。定义句以“直观地说”开头,作为一门旨在为实变函数理论提供严格基础的课程的开始,这似乎是一种奇怪的方式。然而,在某种意义上,这是不可避免的。对基础每一层的修复都会揭示出下面需要关注的内容。集合理论在过去一个世纪中受到了严格的审视,正是因为现代数学的很大一部分建立在这一基础之上。但只有在理解了我们关于集合行为的朴素印象为何不足之后,这样的研究才是真正可取的。对于我们前进的方向,这种情况不会发生,尽管在第1.6节中给出了一些潜在陷阱的提示。

\subsection{函数}
\begin{Def}
  \label{def:1.2.3}
  给定两个集合 \(A\) 和 \(B\) ,从 \(A\) 到 \(B\) 的函数是一种规则或映射,它将每个元素 \(x \in  A\) 与 \(B\) 中的单个元素关联起来。在这种情况下,我们写作 \(f : A \rightarrow  B\) 。给定一个元素 \(x \in  A\) ,表达式 \(f\left( x\right)\) 用于表示通过 \(f\) 与 \(x\) 关联的 \(B\) 中的元素。集合 \(A\) 被称为 \(f\) 的定义域。 \(f\) 的值域不一定等于 \(B\) ,而是指由 \(\{ y \in  B : y = f\left( x\right)\) ,其中 \(x \in  A\}\) 给出的 \(B\) 的子集。
\end{Def}


这个函数的定义或多或少是由 Peter Lejeune Dirichlet (1805-1859) 在19世纪30年代提出的。Dirichlet 是一位德国数学家,他是我们即将进行的函数严格方法发展的领导者之一。他的主要动机是解决围绕 Fourier 级数收敛性的问题。Dirichlet的贡献在第\ref{sec:8.3}节中占据重要地位,该节介绍了 Fourier 级数,但我们也会在之前的几章中多次提到他的名字。目前重要的是,我们看到 Dirichlet 的函数定义如何将这一术语从“公式”类型的解释中解放出来。在Dirichlet时代之前,“函数”一词通常被理解为指代诸如 \(f\left( x\right)  = {x}^{2} + 1\) 或 \(g\left( x\right)  = \sqrt{{x}^{4} + 4}\) 等代数实体。定义~\ref{def:1.2.3}则允许了更广泛的可能性。

\begin{Eg}
  \label{eg:1.2.4}
1829年,Dirichlet 提出了不规则函数

\[
g\left( x\right)  = \left\{  \begin{array}{ll} 1 & \text{ if }x \in  \mathbb{Q} \\  0 & \text{ if }x \notin  \mathbb{Q}. \end{array}\right.
\]

\(g\) 的定义域是 \(\mathbb{R}\) ,值域是集合 \(\{ 0,1\}\) 。在通常意义上, \(g\) 没有单一的公式,而且绘制这个函数的图像相当困难(参见第\ref{sec:4.1}节的粗略尝试),但根据定义~\ref{def:1.2.3}的标准,它无疑符合函数的资格。当我们研究连续、可微或可积函数的理论性质时,像这样的例子将为我们遇到的许多猜想提供一个宝贵的测试平台。
\end{Eg}


\begin{Eg}[三角不等式]
  \label{eg:1.2.5}
绝对值函数非常重要,因此值得用特殊的符号 \(\left| x\right|\) 来代替通常的 \(f\left( x\right)\) 或 \(g\left( x\right)\) 。它通过分段定义为每个实数定义

\[
\left| x\right|  = \left\{  \begin{array}{ll} x & x \geq  0 \\   - x & x < 0 \end{array}\right.
\]

关于乘法和除法,绝对值函数满足 $\forall a,b\in \mathbb{R}$
\begin{enumerate}
\item \(\left| {ab}\right|  = \left| a\right| \left| b\right|\) 
\item \label{item:1.2.1}    \(\left| {a + b}\right|  \leq  \left| a\right|  + \left| b\right|\)
\end{enumerate}

验证这些性质(练习1.2.4)只需检查当 \(a,b\) 和 \(a + b\) 为正和负时出现的不同情况。性质\ref{item:1.2.1}称为三角不等式。这个看似无害的不等式实际上非常重要,并且将经常以下列方式使用。给定三个实数 \(a,b\) 和 \(c\) ,我们当然有

\[
\left| {a - b}\right|  = \left| {\left( {a - c}\right)  + \left( {c - b}\right) }\right| .
\]

根据三角不等式,

\[
\left| {\left( {a - c}\right)  + \left( {c - b}\right) }\right|  \leq  \left| {a - c}\right|  + \left| {c - b}\right| ,
\]

所以我们得到
\begin{equation}
\label{eq:1.4}
\left| {a - b}\right|  \leq  \left| {a - c}\right|  + \left| {c - b}\right| .
\end{equation}
现在,表达式 \(\left| {a - b}\right|\) 等于 \(\left| {b - a}\right|\) ,最好理解为数轴上点 \(a\) 和点 \(b\) 之间的距离。通过这种解释,方程\eqref{eq:1.4}提出了一个合理的陈述:“从 \(a\) 到 \(b\) 的距离小于或等于从 \(a\) 到 \(c\) 的距离加上从 \(c\) 到 \(b\) 的距离。”设想这些点是平面上的点(而不是实数轴上的点),那么为什么这被称为“三角不等式”就显而易见了。
\end{Eg}

\subsection{逻辑与证明}
撰写严谨的数学证明是一项最好通过实践来学习的技能,接下来有大量的实践训练机会。正如 Hardy 所指出的,这类数学具有一种艺术特质,这种特质可能容易掌握,也可能不容易掌握,但这并不意味着有什么特别神秘的事情在发生。证明在某种程度上是一篇短文。它是一系列精心构思的步骤说明,读者按照这些说明进行推导,应该会完全相信所讨论命题的真实性。为了达到这一目的,证明中的每一步都必须从前一步逻辑地推导出来,或者由其他公认的事实来证明其合理性。除了要有效之外,这些步骤还必须连贯地组合在一起,形成一个有说服力的论证。诚然,数学有专门的术语,但这并不意味着一个好的证明可以不使用语法正确的自然语言来撰写。

到目前为止,我们看到的唯一证明(针对定理~\ref{thm:1.1.1})使用了一种称为反证法的间接策略。这种强大的技术将在我们接下来的工作中多次使用。然而,大多数证明都是直接的。(值得一提的是,当直接证明可用时使用间接证明通常被认为是不礼貌的。)直接证明从某个有效的陈述开始,通常取自定理的假设,然后通过严格的逻辑推理逐步展示定理的结论。正如我们在定理~\ref{thm:1.1.1}中看到的,间接证明总是从否定我们想要证明的内容开始。这并不总是像听起来那么容易。然后,论证继续进行,直到(希望)发现与某些其他公认事实的逻辑矛盾。很多时候,这个公认的事实是定理假设的一部分。当矛盾与定理的假设相关时,从技术上讲,我们得到的是所谓的逆否证明。

下一个命题说明了刚才讨论的许多问题,并引入了更多内容。
\begin{Thm}
  \label{thm:1.2.6}
两个实数 \(a\) 和 \(b\) 相等,当且仅当对于每一个实数 \(\varepsilon  > 0\) ,都有 \(\left| {a - b}\right|  < \varepsilon\) 。  
\end{Thm}

\begin{proof}
在这个命题的陈述中有两个关键短语需要特别注意。一个是“对于每一个”,稍后会讨论。另一个是“当且仅当”。在数学中说“当且仅当”是一种简洁的方式,表示该命题在两个方向上都是成立的。在正向方向上,我们必须证明以下陈述:

\(\left(  \Rightarrow  \right)\) 如果 \(a = b\) ,那么对于每一个实数 \(\varepsilon  > 0\) ,都有 \(\left| {a - b}\right|  < \varepsilon\) 。我们还必须证明其逆命题:

($\Leftarrow$)如果对于每一个实数 \(\varepsilon  > 0\) ,都有 \(\left| {a - b}\right|  < \varepsilon\) ,那么我们必须有 \(a = b\) 。

对于第一个命题的证明,确实没有太多可说的。如果 \(a = b\) ,那么 \(\left| {a - b}\right|  = 0\) ,因此无论选择什么 \(\varepsilon  > 0\) ,都肯定有 \(\left| {a - b}\right|  < \varepsilon\) 。

对于第二个陈述,我们给出一个反证法证明。该命题在这个方向上的结论表明 \(a = b\) ,因此我们假设 \(a \neq  b\) 。为了寻找矛盾,我们开始考虑短语“对于每一个 \(\varepsilon  > 0\) ”。一些等价的表述假设的方式可以是说“对于 \(\varepsilon  > 0\) 的所有可能选择”或“无论 \(\varepsilon  > 0\) 如何选择,总是 \(\left| {a - b}\right|  < \varepsilon\) 的情况”。但假设 \(a \neq  b\) (正如我们目前所做的),选择

\[
{\varepsilon }_{0} = \left| {a - b}\right|  > 0
\]

带来了一个严重的问题。我们假设 \(\left| {a - b}\right|  < \varepsilon\) 对于每一个 \(\varepsilon  > 0\) 都成立,因此这对于刚刚定义的特定 \({\varepsilon }_{0}\) 也必然成立。然而,命题

\[
\left| {a - b}\right|  < {\varepsilon }_{0}, \quad\left| {a - b}\right|  = {\varepsilon }_{0}
\]

不可能同时为真。这一矛盾意味着我们最初关于 \(a \neq  b\) 的假设是不可接受的。因此, \(a = b\) ,间接证明完成。
\end{proof}

阅读和撰写分析证明所需的最基本技能之一是能够自信地操作量化短语“对于所有($\forall$)”和“存在($\exists$)”。在接下来的许多讨论中,将更加关注这一问题。

\subsection{归纳法}

最后一种常见的技巧是使用归纳法论证。归纳法与自然数 \(\mathbb{N}\) (有时与集合 \(\mathbb{N} \cup  \{ 0\}\) )结合使用。归纳法的基本原理是,如果 \(S\) 是 \(\mathbb{N}\) 的某个子集,且具有以下性质:

\begin{enumerate}[label = (\roman*)]
\item \(S\) 包含$1$
\item 每当 \(S\) 包含一个自然数 \(n\) 时,它也包含 \(n + 1\) 
\end{enumerate}

那么必定有 \(S = \mathbb{N}\) 。正如下一个例子所示,这一原则既可用于定义序列,也可用于证明关于它们的事实。

\begin{Eg}
  \label{eg:1.2.7}
设 \({x}_{1} = 1\) ,并对每个 \(n \in  \mathbb{N}\) 定义

\[
{x}_{n + 1} = \left( {1/2}\right) {x}_{n} + 1
\]

使用此规则,我们可以计算 \({x}_{2} = \left( {1/2}\right) \left( 1\right)  + 1 = 3/2,{x}_{3} = 7/4\) ,并且立即可以看出这如何导致对所有 \(n \in  \mathbb{N}\) 的 \({x}_{n}\) 的定义。

刚刚定义的序列起初似乎是递增的。对于已计算的项,我们有 \({x}_{1} \leq  {x}_{2} \leq  {x}_{3}\) 。让我们用归纳法来证明这一趋势持续下去;也就是说,让我们证明 $\forall n\in \mathbb{N}$

\begin{equation}
\label{eq:1.5}
{x}_{n} \leq  {x}_{n + 1}
\end{equation}

对于 \(n = 1,{x}_{1} = 1\) 和 \({x}_{2} = 3/2\) , \({x}_{1} \leq  {x}_{2}\) 是显然的。现在,我们想要证明:如果我们有 \({x}_{n} \leq  {x}_{n + 1}\) ,那么我们可以得出 \({x}_{n + 1} \leq  {x}_{n + 2}\) 。

将 \(S\) 视为方程\eqref{eq:1.5}中命题为真的自然数集合。我们已经证明了 \(1 \in  S\) 。我们现在感兴趣的是证明如果 \(n \in  S\) ,那么 \(n + 1 \in  S\) 也成立。从归纳假设 \({x}_{n} \leq  {x}_{n + 1}\) 出发,我们可以将不等式两边乘以 \(1/2\) 并加1得到

\[
\frac{1}{2}{x}_{n} + 1 \leq  \frac{1}{2}{x}_{n + 1} + 1
\]

这正是我们想要的结论 \({x}_{n + 1} \leq  {x}_{n + 2}\) 。通过归纳法,该命题对所有 \(n \in  \mathbb{N}\) 都得到了证明。
  
\end{Eg}

任何关于为什么归纳法是一种有效的论证技巧的讨论,都会立即引发一系列关于我们如何理解自然数的问题。早些时候,在第\ref{sec:1.1}节中,我们通过引用 Kronecker 的著名评论——自然数在某种程度上是神赐的——来回避这个问题。尽管我们在这里不会改进这个解释,但应该指出的是,从公理集合论的角度来看,对 \(\mathbb{N}\) 采取一种更无神论且数学上更令人满意的方法是可能的。这让我们回到了本章的一个反复出现的主题。从教学的角度来说,数学基础最好以一种逆向的顺序来学习和理解。虽然自然数与集合论的深入研究值得提倡,但这必须建立在透彻掌握实数系统复杂特性的基础之上。后者正是实分析的研究内容。

\subsection{练习}

练习1.2.1. (a) 证明 \(\sqrt{3}\) 是无理数。类似的论证是否适用于证明 \(\sqrt{6}\) 是无理数?

(b) 如果我们尝试用定理1.1.1的证明来证明 \(\sqrt{4}\) 是无理数,证明在何处会失效?

练习1.2.2. 判断以下哪些陈述正确地描述了集合的性质。对于任何错误的陈述,提供一个具体的例子说明该陈述不成立。

(a) 如果 \({A}_{1} \supseteq  {A}_{2} \supseteq  {A}_{3} \supseteq  {A}_{4}\cdots\) 都是包含无限多个元素的集合,那么它们的交集 \({ \cap  }_{n = 1}^{\infty }{A}_{n}\) 也是无限的。

(b) 如果 \({A}_{1} \supseteq  {A}_{2} \supseteq  {A}_{3} \supseteq  {A}_{4}\cdots\) 都是有限的、非空的实数集,那么它们的交集 \({ \cap  }_{n = 1}^{\infty }{A}_{n}\) 是有限且非空的。

(c) \(A \cap  \left( {B \cup  C}\right)  = \left( {A \cap  B}\right)  \cup  C\) .

(d) \(A \cap  \left( {B \cap  C}\right)  = \left( {A \cap  B}\right)  \cap  C\) .

(e) \(A \cap  \left( {B \cup  C}\right)  = \left( {A \cap  B}\right)  \cup  \left( {A \cap  C}\right)\) .

练习 1.2.3 (德摩根定律). 设 \(A\) 和 \(B\) 是 \(\mathbb{R}\) 的子集。

(a) 如果 \(x \in  {\left( A \cap  B\right) }^{c}\) ,解释为什么 \(x \in  {A}^{c} \cup  {B}^{c}\) 。这表明 \({\left( A \cap  B\right) }^{c} \subseteq\)  \({A}^{c} \cup  {B}^{c}\) 。

(b) 证明反向包含 \({\left( A \cap  B\right) }^{c} \supseteq  {A}^{c} \cup  {B}^{c}\) ,并得出结论 \({\left( A \cap  B\right) }^{c} = {A}^{c} \cup  {B}^{c}\) 。

(c) 通过双向包含证明 \({\left( A \cup  B\right) }^{c} = {A}^{c} \cap  {B}^{c}\) 。

练习 1.2.4。验证在以下特殊情况下的三角不等式:(a) \(a\) 和 \(b\) 具有相同的符号;

(b) \(a \geq  0,b < 0\) ,以及 \(a + b \geq  0\) 。

练习 1.2.5。使用三角不等式来建立不等式 (a) \(\left| {a - b}\right|  \leq  \left| a\right|  + \left| b\right|\) ;

(b) \(\left| \right| a\left| -\right| b\left| \right|  \leq  \left| {a - b}\right|\) .

练习 1.2.6。给定一个函数 \(f\) 及其定义域的一个子集 \(A\) ,让 \(f\left( A\right)\) 表示 \(f\) 在集合 \(A\) 上的范围;即 \(f\left( A\right)  = \{ f\left( x\right)  : x \in  A\}\) 。

(a) 设 \(f\left( x\right)  = {x}^{2}\) 。若 \(A = \left\lbrack  {0,2}\right\rbrack\) (闭区间 \(\{ x \in  \mathbb{R} : 0 \leq  x \leq  2\}\) )且 \(B = \left\lbrack  {1,4}\right\rbrack\) ,求 \(f\left( A\right)\) 和 \(f\left( B\right)\) 。在这种情况下 \(f\left( {A \cap  B}\right)  = f\left( A\right)  \cap  f\left( B\right)\) 是否成立? \(f\left( {A \cup  B}\right)  = f\left( A\right)  \cup  f\left( B\right)\) 是否成立?

(b) 找到两个集合 \(A\) 和 \(B\) ,使得 \(f\left( {A \cap  B}\right)  \neq  f\left( A\right)  \cap  f\left( B\right)\) 。

(c) 证明对于任意函数 \(g : \mathbb{R} \rightarrow  \mathbb{R}\) ,对于所有集合 \(A,B \subseteq  \mathbb{R}\) , \(g\left( {A \cap  B}\right)  \subseteq  g\left( A\right)  \cap  g\left( B\right)\) 总是成立。

(d) 形成并证明一个关于任意函数 \(g\) 下 \(g\left( {A \cup  B}\right)\) 和 \(g\left( A\right)  \cup  g\left( B\right)\) 之间关系的猜想。

练习 1.2.7。给定一个函数 \(f : D \rightarrow  \mathbb{R}\) 和一个子集 \(B \subseteq  \mathbb{R}\) ,令 \({f}^{-1}\left( B\right)\) 为定义域 \(D\) 中所有被映射到 \(B\) 的点的集合;即 \({f}^{-1}\left( B\right)  = \{ x \in  D : f\left( x\right)  \in  B\}\) 。这个集合被称为 \(B\) 的原像。

(a) 设 \(f\left( x\right)  = {x}^{2}\) 。如果 \(A\) 是闭区间 \(\left\lbrack  {0,4}\right\rbrack\) 且 \(B\) 是闭区间 \(\left\lbrack  {-1,1}\right\rbrack\) ,求 \({f}^{-1}\left( A\right)\) 和 \({f}^{-1}\left( B\right)\) 。在这种情况下, \({f}^{-1}\left( {A \cap  B}\right)  = {f}^{-1}\left( A\right)  \cap  {f}^{-1}\left( B\right)\) 成立吗? \({f}^{-1}\left( {A \cup  B}\right)  = {f}^{-1}\left( A\right)  \cup  {f}^{-1}\left( B\right)\) 成立吗?

(b) 在(a)中展示的原像的良好行为是完全普遍的。证明对于任意函数 \(g : \mathbb{R} \rightarrow  \mathbb{R}\) ,对于所有集合 \(A,B \subseteq  \mathbb{R}\) , \({g}^{-1}\left( {A \cap  B}\right)  = {g}^{-1}\left( A\right)  \cap  {g}^{-1}\left( B\right)\) 和 \({g}^{-1}\left( {A \cup  B}\right)  = {g}^{-1}\left( A\right)  \cup  {g}^{-1}\left( B\right)\) 总是成立。

练习1.2.8。形成每个命题的逻辑否定。一种方法是在每个断言前简单地加上“情况并非如此……”,但对于每个陈述,尝试将“不”这个词尽可能深入地嵌入到结果句子中(或完全避免使用它)。

(a) 对于所有满足 \(a < b\) 的实数,存在一个 \(n \in  \mathbb{N}\) 使得 \(a + 1/n < b\) 。

(b) 在每两个不同的实数之间,存在一个有理数。

(c) 对于所有自然数 \(n \in  \mathbb{N},\sqrt{n}\) ,它要么是自然数,要么是无理数。

(d) 给定任何实数 \(x \in  \mathbb{R}\) ,存在 \(n \in  \mathbb{N}\) 满足 \(n > x\) 。

练习1.2.9。证明在例1.2.7中定义的序列 \(\left( {{x}_{1},{x}_{2},{x}_{3},\ldots }\right)\) 以2为上界;即,证明对于每个 \(n \in  \mathbb{N}\) , \({x}_{n} \leq  2\) 成立。

练习 1.2.10. 设 \({y}_{1} = 1\) ,并且对于每个 \(n \in  \mathbb{N}\) 定义 \({y}_{n + 1} = \left( {3{y}_{n} + 4}\right) /4\) 。

(a) 使用归纳法证明该序列对所有 \(n \in  \mathbb{N}\) 满足 \({y}_{n} < 4\) 。

(b) 使用另一个归纳论证来证明序列 \(\left( {{y}_{1},{y}_{2},{y}_{3},\ldots }\right)\) 是递增的。

练习 1.2.11. 如果一个集合 \(A\) 包含 \(n\) 个元素,证明 \(A\) 的不同子集的数量等于 \({2}^{n}\) 。(请记住,空集 \(\varnothing\) 被认为是每个集合的子集。)

练习 1.2.12. 对于本练习,假设练习 1.2.3 已成功完成。

(a) 展示如何使用归纳法得出结论

\[
{\left( {A}_{1} \cup  {A}_{2} \cup  \cdots  \cup  {A}_{n}\right) }^{c} = {A}_{1}^{c} \cap  {A}_{2}^{c} \cap  \cdots  \cap  {A}_{n}^{c}
\]

对于任何有限的 \(n \in  \mathbb{N}\) 。

(b) 解释为什么不能使用归纳法得出结论

\[
{\left( \mathop{\bigcup }\limits_{{n = 1}}^{\infty }{A}_{n}\right) }^{c} = \mathop{\bigcap }\limits_{{n = 1}}^{\infty }{A}_{n}^{c}
\]

考虑练习1.2.2的(a)部分可能有所帮助。

(c) 部分(b)中的陈述是否有效?如果有效,请写出一个不使用归纳法的证明。

\section{完备性公理}
\label{sec:1.3}

实数究竟是什么?在\ref{sec:1.1}节中,我们提到实数集 \(\mathbb{R}\) 是有理数集 \(\mathbb{Q}\) 的扩展,其中没有空洞或间隙。我们希望数轴上的每一个长度——例如 \(\sqrt{2}\) ——都对应一个实数,反之亦然。

我们将改进这个定义。但在这样做时,务必要记住我们早先承认的事实:无论我们制定出多么精确的陈述,都必然依赖于其他未经证实的假设或未定义的术语。在某个时刻,我们必须划清界限,并承认这是我们决定接受的一个合理的起点。自然,关于这条界限应该划在哪里存在一些争议。看待19世纪和20世纪数学的一种方式是将其视为一种坚定的尝试,试图将这条界限不断向后推移,朝着某种不可动摇的基础前进。本书所涵盖的大部分内容可归功于19世纪早期和中期的数学家们。Augustin Louis Cauchy(1789-1857)、Bernhard Bolzano(1781-1848)、Niels Henrik Abel(1802-1829)、Peter Lejeune Dirichlet、Karl Weierstrass (1815-1897)和 Bernhard Riemann (1826-1866)都在后续定理的发现中占据了重要地位。但这里有一个有趣的观点。几乎所有这项工作都是基于对 \(\mathbb{R}\) 性质的直观假设完成的,这些假设与我们此时非正式的理解非常相似。最终,对 \(\mathbb{R}\) 详细结构的足够审查使得在19世纪70年代,提出了几种从 \(\mathbb{Q}\) 严格构造 \(\mathbb{R}\) 的方法。

遵循这一历史模型,我们自己对 \(\mathbb{R}\) 从 \(\mathbb{Q}\) 的严格构建将推迟到第\ref{sec:8.4}节。到那时,这种构建的必要性将更加合理且更容易理解。与此同时,我们有许多证明要写,因此尽可能明确地列出我们打算对实数做出的假设是非常重要的。

\subsection{\(\mathbb{R}\) 的初步定义}

首先, \(\mathbb{R}\) 是一个包含 \(\mathbb{Q}\) 的集合。加法和乘法运算在 \(\mathbb{Q}\) 上扩展到 \(\mathbb{R}\) 的所有元素,使得 \(\mathbb{R}\) 中的每个元素都有一个加法逆元,且 \(\mathbb{R}\) 中的每个非零元素都有一个乘法逆元。呼应第\ref{sec:1.1}节的讨论,我们假设 \(\mathbb{R}\) 是一个域,这意味着实数的加法和乘法是交换的、结合的,并且分配律成立。这使我们能够执行所有对我们来说习以为常的标准代数操作。我们还假设 \(\mathbb{Q}\) 上的序关系的熟悉性质可以扩展到 \(\mathbb{R}\) 的所有元素。因此,例如,像“如果 \(a < b\) 且 \(c > 0\) ,那么 \({ac} < {bc}\) ”这样的推论将自由地进行,无需过多解释。用该学科的正式术语总结这一情况,我们假设 \(\mathbb{R}\) 是一个有序域,它包含 \(\mathbb{Q}\) 作为子域。(“有序域”的严格定义在第\ref{sec:8.4}节中给出。)

这让我们来到了关于实数系统的最后一个,也是最独特的假设。我们必须找到某种方式来清晰地表达我们坚持认为 \(\mathbb{R}\) 不包含 \(\mathbb{Q}\) 中的间隙的含义。因为这是有理数和实数之间的决定性差异,我们将非常精确地表述这一假设,以下称之为完备性公理。

\begingroup
\renewcommand{\theThm}{} % 清空定理编号
\begin{Axm}[完备性公理]
  每一个非空的、有上界的实数集合都有一个最小上界。
\end{Axm}
\endgroup

\addtocounter{Thm}{-1}

那么,这到底是什么意思呢?

\subsection{最小上界与最大下界}

让我们首先陈述相关的定义,然后看一些例子。
\begin{Def}
  \label{def:1.3.1}
称集合 \(A \subseteq  \mathbb{R}\) 为上有界的,如果存在一个数 \(b \in  \mathbb{R}\) ,使得\( \forall a \in  A\) ,都有 \(a \leq  b\) 。这个数 \(b\) 被称为 \(A\) 的一个上界。

类似地,称集合 \(A\) 为下有界的,如果存在一个下界 \(l \in  \mathbb{R}\) 满足 \(\forall a \in  A\) 都有 \(l \leq  a\)。  
\end{Def}

\begin{Def}
  \label{def:1.3.2}
称实数 \(s\) 是集合 \(A \subseteq  \mathbb{R}\) 的最小上界,如果它满足以下两个条件:
\begin{enumerate}[label = (\roman*)]
\item\label{item:1.3.1}  \(s\) 是 \(A\) 的一个上界;
\item\label{item:1.3.2} 如果 \(b\) 是 \(A\) 的任意一个上界,则 \(s \leq  b\) 。
\end{enumerate}
最小上界也常被称为集合 \(A\) 的上确界。尽管符号 \(s = \operatorname{lub}A\) 仍然常见,但我们将始终使用 \(s = \sup A\) 来表示最小上界。  
\end{Def}

最大下界或下确界对于 \(A\) 的定义方式类似(练习1.3.2),并用\(\inf A\) 表示(图\ref{fig:1.3})。

\begin{figure}[h]
  \centering
  \includegraphics[width=0.7\textwidth]{images/01955a90-3fc5-700a-bdb0-df1c9e2a30b7_14_364_414_909_190_0.jpg}
  \caption{ \(\sup A\) 和 \(\inf A\) 的定义}
  \label{fig:1.3}
\end{figure}

虽然一个集合可以有多个上界,但它只能有一个最小上界。如果 \({s}_{1}\) 和 \({s}_{2}\) 都是集合 \(A\) 的最小上界,那么根据定义~\ref{def:1.3.2}中的性质~\ref{item:1.3.2},我们可以断言 \({s}_{1} \leq  {s}_{2}\) 和 \({s}_{2} \leq  {s}_{1}\) 。这就导致 \({s}_{1} = {s}_{2}\) ,即最小上界是唯一的。

\begin{Eg}
  \label{eg:1.3.3}
设

\[
A = \left\{  {\frac{1}{n} : n \in  N}\right\}   = \left\{  {1,\frac{1}{2},\frac{1}{3},\ldots }\right\}  .
\]

集合 \(A\) 上下都有界。其上界有$3$、$2$和 \(3/2\) 等。对于最小上界,我们断言 \(\sup A = 1\) 。为了使用定义~\ref{def:1.3.2}严谨地论证这一点,我们需要验证性质~\ref{item:1.3.1}和~\ref{item:1.3.2}是否成立。对于性质~\ref{item:1.3.1},我们只需观察到 \( \forall n \in  \mathbb{N}\) ,都有, \(1 \geq  1/n\) 成立。为了验证性质~\ref{item:1.3.2},我们首先假设我们拥有某个其他上界 \(b\) 。由于 \(1 \in  A\) 且 \(b\) 是 \(A\) 的上界,我们必须有 \(1 \leq  b\) 。这正是性质~\ref{item:1.3.2}要求我们展示的内容。

尽管我们还没有完全掌握进行严谨证明所需的工具(见定理~\ref{thm:1.4.2}),但应该有些明显的是 \(\inf A = 0\) 。
  
\end{Eg}

从例~\ref{eg:1.3.3}中得出的一个重要教训是, \(\sup A\) 和 \(\inf A\) 可能是也可能不是集合 \(A\) 的元素。这个问题与理解给定集合的最大值和上确界(或最小值和下确界)之间的关键区别密切相关。


\begin{Def}
  \label{def:1.3.4}
称实数 \({a}_{0}\) 是集合 \(A\) 的最大值,如果 \({a}_{0}\in A\) ,且 \(\forall a \in  A\) , \({a}_{0} \geq  a\) 。类似地,称实数 \({a}_{1}\) 是 \(A\) 的最小值,如果 \({a}_{1} \in  A\) 并且 \(\forall a \in  A\) , \({a}_{1} \leq  a\) 。  
\end{Def}

\begin{Eg}
  \label{eg:1.3.5}
为了强调这一点,考虑开区间

\[
\left( {0,2}\right)  = \{ x \in  \mathbb{R} : 0 < x < 2\} ,
\]

和闭区间

\[
\left\lbrack  {0,2}\right\rbrack   = \{ x \in  \mathbb{R} : 0 \leq  x \leq  2\} .
\]

两个集合都有上界(和下界),并且它们有相同的最小上界,即$2$。然而,这两个集合并不都有最大值。最大值是一种特定类型的上界,它必须是所讨论集合的元素,而开区间$(0,2)$并不具备这样的元素。因此,上确界可以存在但不是最大值,但当最大值存在时,它也是上确界。  
\end{Eg}

让我们把注意力转回到完备性公理。虽然我们现在可以看到并非每个非空有界集合都包含最大值,但完备性公理断言每个这样的集合都有一个最小上界。我们不会证明这一点。数学中的公理是一个被接受的假设,无需证明即可使用。公理应该是关于所讨论系统的基本陈述,基础到似乎不需要任何理由。也许完备性公理符合这一描述,也许不符合。在决定之前,让我们提醒自己为什么它不是关于 \(\mathbb{Q}\) 的有效陈述。

\begin{Eg}
  \label{eg:1.3.6}
再次考虑集合

\[
S = \left\{  {r \in  \mathbb{Q} : {r}^{2} < 2}\right\}  ,
\]

暂且假设我们的世界仅由有理数组成。集合 \(S\) 显然是有上界的——可以取 \(b = 2\) ,也可以取 \(b = 3/2\) 。但请注意,当我们寻找最小上界时会发生什么。(这里知道 \(\sqrt{2}\) 的小数展开以 \({1.4142}\ldots\) 开头可能有用。)我们可能会尝试 \(b = {142}/{100}\) ,它确实是一个上界,但随后我们发现 \(b = {1415}/{1000}\) 是一个更小的上界。是否存在一个最小的上界?

在有理数中,不存在。在实数中则存在。回到 \(\mathbb{R}\) ,完备性公理指出我们可以设定 \(\alpha  = \sup S\) ,并确信这样的数存在。在下一节中,我们将证明 \({\alpha }^{2} = 2\) 。但根据定理~\ref{thm:1.1.1},这意味着 \(\alpha\) 不是有理数。如果我们只关注有理数,那么 \(\alpha\) 不是 \(\sup S\) 的可选选项,寻找最小上界的过程将无限进行下去。无论发现什么有理数上界,我们总能找到一个更小的上界。
\end{Eg}

执行例~\ref{eg:1.3.6}中描述的计算所需的工具依赖于一些关于 \(\mathbb{Q}\) 和 \(\mathbb{N}\) 如何嵌入 \(\mathbb{R}\) 的结果。这些将在下一节中讨论。

我们现在给出一种等价且有用的方式来描述最小上界。回想一下,定义~\ref{eg:1.3.6}中的上确界有两部分。第~\ref{item:1.3.1}部分说明 \(\sup A\) 必须是一个上界,第~\ref{item:1.3.2}部分则说明它必须是最小的一个。以下引理提供了一种重新表述第~\ref{item:1.3.2}部分的替代方式。

\begin{Lem}
  \label{lem:1.3.7}
设 \(s \in  \mathbb{R}\) 是集合 \(A \subseteq  \mathbb{R}\) 的一个上界。那么, \(s = \sup A\) 当且仅当 \(\forall \varepsilon  > 0\) , \(\exists a \in  A\) 满足 \(s - \varepsilon  < a\) 。  
\end{Lem}

\begin{proof}
以下是该引理的简短重述:假设 \(s\) 是一个上界,那么 \(s\) 是最小上界当且仅当任何小于 \(s\) 的数都不是上界。这样表述几乎可以视为证明,但我们将详细阐述每个方向的具体含义。

\(\left(  \Rightarrow  \right)\) 对于正向方向,我们假设 \(s = \sup A\) 并考虑 \(s - \varepsilon\) ,其中 \(\varepsilon  > 0\) 已被任意选择。由于 \(s - \varepsilon  < s\) ,定义~\ref{def:1.3.2}的第~\ref{item:1.3.2}部分意味着 \(s - \varepsilon\) 不是 \(A\) 的上界。如果是这种情况,那么必须存在某个元素 \(a \in  A\) ,使得 \(s - \varepsilon  < a\) (否则 \(s - \varepsilon\) 将是上界)。这在一个方向上证明了引理。

$\left( \Leftarrow \right)$假设 \(s\) 是一个上界,且具有无论 \(\varepsilon  > 0\) 如何选择, \(s - \varepsilon\) 都不再是 \(A\) 的上界的性质。注意,这意味着如果 \(b\) 是任何小于 \(s\) 的数,那么 \(b\) 不是上界(只需令 \(\varepsilon  = s - b\) )。为了证明 \(s = \sup A\) ,我们必须验证定义~\ref{def:1.3.2}的第~\ref{item:1.3.2}部分。因为我们刚刚论证了任何小于 \(s\) 的数都不能是上界,所以如果 \(b\) 是 \(A\) 的另一个上界,那么 \(b \geq  s\) 。  
\end{proof}

可以肯定的是,我们到目前为止关于最小上界的所有结论都有相应的最大下界版本。完备性公理并没有明确断言一个有下界的非空集合存在下确界,但这是因为我们不需要将这一事实作为公理的一部分来假设。使用完备性公理,有几种方法可以证明有界集合存在最大下界。其中一个证明在练习1.3.3中进行了探讨。

\subsection{练习}

练习1.3.1。设 \({\mathbb{Z}}_{5} = \{ 0,1,2,3,4\}\) 并定义加法和乘法模5。换句话说,计算 \(a + b\) 和 \({ab}\) 除以5时的整数余数,并分别将其作为和与积的值。

(a) 证明,给定任何元素 \(z \in  {\mathbb{Z}}_{5}\) ,存在一个元素 \(y\) ,使得 \(z + y = 0\) 。元素 \(y\) 称为 \(z\) 的加法逆元。

(b) 证明,给定任意 \(z \neq  0\) 在 \({\mathbb{Z}}_{5}\) 中,存在一个元素 \(x\) 使得 \({zx} = 1\) 。该元素 \(x\) 被称为 \(z\) 的乘法逆元。

(c) 加法和乘法逆元的存在性是域定义的一部分。研究集合 \({\mathbb{Z}}_{4} = \{ 0,1,2,3\}\) (其中加法和乘法定义为模4)中加法和乘法逆元的存在性。对 \(n\) 的值在 \({\mathbb{Z}}_{n}\) 中存在加法逆元的情况提出一个猜想,然后对乘法逆元的存在性提出另一个猜想。

练习 1.3.2. (a) 以定义 1.3.2 的风格为集合的下确界或最大下界写一个正式定义。

(b) 现在,陈述并证明一个关于最大下界的引理 1.3.7 的版本。

练习1.3.3. (a) 设 \(A\) 有下界,并定义 \(B = \{ b \in  \mathbb{R}\) : \(b\) 是 \(A\}\) 的下界。证明 \(\sup B = \inf A\) 。

(b) 使用(a)解释为什么在完备性公理中不需要断言最大上界的存在。

(c) 提出另一种利用完备性公理证明有下界的集合存在最大下界的方法。

练习1.3.4. 假设 \(A\) 和 \(B\) 非空、有上界,并满足 \(B \subseteq  A\) 。证明 \(\sup B \leq  \sup A\) 。

练习1.3.5。设 \(A \subseteq  \mathbb{R}\) 有上界,且设 \(c \in  \mathbb{R}\) 。定义集合 \(c + A\) 和 \({cA}\) 为 \(c + A = \{ c + a : a \in  A\}\) 和 \({cA} = \{ {ca} : a \in  A\}\) 。

(a) 证明 \(\sup \left( {c + A}\right)  = c + \sup A\) 。

(b) 如果 \(c \geq  0\) ,证明 \(\sup \left( {cA}\right)  = c\sup A\) 。

(c) 对于 \(c < 0\) 的情况,假设一个类似 \(\sup \left( {cA}\right)\) 的陈述。

练习1.3.6。在不证明的情况下,计算以下集合的上确界和下确界:

(a) \(\left\{  {n \in  \mathbb{N} : {n}^{2} < {10}}\right\}\) .

(b) \(\{ n/\left( {m + n}\right)  : m,n \in  \mathbb{N}\}\) .

(c) \(\{ n/\left( {{2n} + 1}\right)  : n \in  \mathbb{N}\}\) .

(d) \(\{ n/m : m,n \in  \mathbb{N}\) 与 \(m + n \leq  {10}\}\) 。

练习 1.3.7. 证明如果 \(a\) 是 \(A\) 的上界,并且 \(a\) 也是 \(A\) 的元素,那么必有 \(a = \sup A\) 。

练习 1.3.8. 如果 \(\sup A < \sup B\) ,则证明存在一个元素 \(b \in  B\) ,它是 \(A\) 的上界。

练习 1.3.9. 暂时不考虑形式证明,判断以下关于上确界和下确界的陈述是真还是假。对于任何错误的陈述,提供一个例子说明该主张似乎不成立。

(a) 一个有限的非空集总是包含它的上确界。

(b) 如果对于集合 \(A\) 中的每个元素 \(a\) ,都有 \(a < L\) ,那么 \(\sup A < L\) 。

(c) 如果 \(A\) 和 \(B\) 是具有以下性质的集合:对于每个 \(a \in  A\) 和每个 \(b \in  B\) ,都有 \(a < b\) ,那么可以得出 \(\sup A < \inf B\) 。

如果 \(\sup A = s\) 且 \(\sup B = t\) ,则 \(\sup \left( {A + B}\right)  = s + t\) 。集合 \(A + B\) 定义为 \(A + B = \{ a + b : a \in  A\) 和 \(b \in  B\}\) 。

如果 \(\sup A \leq  \sup B\) ,则存在一个元素 \(b \in  B\) ,它是 \(A\) 的上界。

\section{完备性的推论}
\label{sec:1.4}

完备性公理的第一个应用闭区间套定理。它以一种更自然的方式,用数学表达了实数轴上没有间隙的观点。

\begin{Thm}[闭区间套定理]
  \label{thm:1.4.1}
   \(\forall n \in  \mathbb{N}\) ,给定一列闭区间 \({I}_{n} = \left\lbrack  {{a}_{n},{b}_{n}}\right\rbrack   = \left\{  {x \in  \mathbb{R} : {a}_{n} \leq  x \leq  {b}_{n}}\right\}\) ,并设 \({I}_{n}\supseteq {I}_{n + 1}\) 。那么,由此产生的闭区间套序列

\[
{I}_{1} \supseteq  {I}_{2} \supseteq  {I}_{3} \supseteq  {I}_{4} \supseteq  \cdots
\]

具有非空交集;即, \(\mathop{\bigcap }\limits_{{n = 1}}^{\infty }{I}_{n} \neq  \varnothing\) 。
\end{Thm}

\begin{proof}
为了证明 \(\mathop{\bigcap }\limits_{{n = 1}}^{\infty }{I}_{n}\) 非空,我们将使用完备性公理来生成一个实数 \(x\) , \(\forall n \in  \mathbb{N}\) , \(x \in  {I}_{n}\) 成立。完备性公理是关于有界集的陈述,我们要考虑的是下列集合的左端点:

\[
A = \left\{  {{a}_{n} : n \in  \mathbb{N}}\right\}
\]


\begin{figure}[h]
  \centering
  \includegraphics[width=0.6\textwidth]{images/01955a90-3fc5-700a-bdb0-df1c9e2a30b7_18_398_652_840_141_0.jpg}
\end{figure}

由于区间是嵌套的,我们看到每一个 \({b}_{n}\) 都是 \(A\) 的上界。因此,我们有理由设定

\[
x = \sup A
\]

现在,考虑一个特定的 \({I}_{n} = \left\lbrack  {{a}_{n},{b}_{n}}\right\rbrack\) 。因为 \(x\) 是 \(A\) 的上界,我们有 \({a}_{n} \leq  x\) 。每个 \({b}_{n}\) 都是 \(A\) 的上界,且 \(x\) 是最小上界,这意味着 \(x \leq  {b}_{n}\) 。

综上所述,我们有 \({a}_{n} \leq  x \leq  {b}_{n}\) ,这意味着 \(\forall n \in  \mathbb{N}\) ,都有 \(x \in  {I}_{n}\) 。因此, \(x \in  \mathop{\bigcap }\limits_{{n = 1}}^{\infty }{I}_{n}\) ,即交集不为空。  
\end{proof}

\subsection{\(\mathbb{Q}\) 在 \(\mathbb{R}\) 中的密度}

集合 \(\mathbb{Q}\) 是 \(\mathbb{N}\) 的扩展,而 \(\mathbb{R}\) 又是 \(\mathbb{Q}\) 的扩展。接下来的几个结果展示了 \(\mathbb{N}\) 和 \(\mathbb{Q}\) 如何位于 \(\mathbb{R}\) 内部。

\begin{Thm}[Archimedes性]
  \label{thm:1.4.2}
  \begin{enumerate}[label = (\roman*)]
  \item\label{item:1.4.1} \(\forall x \in  \mathbb{R}\) , \(\exists n \in  \mathbb{N}\) 满足 \(n > x\) 。
  \item \label{item:1.4.2 } \(\forall y > 0\) , \(\exists n \in  \mathbb{N}\) 满足 \(1/n < y\) 。
  \end{enumerate}
\end{Thm}

\begin{proof}
命题的第~\ref{item:1.4.1}部分指出 \(\mathbb{N}\) 不是有上界的。这一点是没有疑问的,并且可以合理地认为我们根本不需要证明它。这是一个合理的观点,特别是考虑到我们已经决定假设 \(\mathbb{N},\mathbb{Z}\) 和 \(\mathbb{Q}\) 的其他熟悉属性为已知。

然而,我们将证明它,因为我们能够做到。一个集合可以具有 Archimides 性而不一定是完备的, \(\mathbb{Q}\) 就是一个很好的例子,但证明这一事实需要对所讨论的有序域的公理构造进行大量的审查。在 \(\mathbb{R}\) 的情况下,完备性公理为我们提供了一个非常简短的论证。许多深刻的结果最终都依赖于 \(\mathbb{R}\) 和 \(\mathbb{N}\) 之间的这种关系,因此为其提供一个证明为这些即将到来的论证增加了一点额外的确定性。

我们来进行证明。反设 \(\mathbb{N}\) 是有上界的。根据完备性公理, \(\mathbb{N}\) 应该有一个最小上界,我们可以设其为 \(\alpha  = \sup \mathbb{N}\) 。考虑 \(\alpha  - 1\) ,它不是上界。(参见引理~\ref{lem:1.3.7}),因此存在一个 \(n \in  \mathbb{N}\) 满足 \(\alpha  - 1 < n\) 。但这等价于 \(\alpha  < n + 1\) 。由于 \(n + 1 \in  \mathbb{N}\) ,我们得到了与 \(\alpha\) 应该是 \(\mathbb{N}\) 的上界这一事实相矛盾的结果。(注意,这里的矛盾仅依赖于完备性公理以及 \(\mathbb{N}\) 在加法下封闭的事实。)

第二部分通过令 \(x = 1/y\) 从第一部分得出。  
\end{proof}

\(\mathbb{N}\) 的这一熟悉特性是关于 \(\mathbb{Q}\) 如何嵌入 \(\mathbb{R}\) 的一个极其重要事实的关键。


\begin{Thm}[\(\mathbb{Q}\) 在 \(\mathbb{R}\) 中的稠密性]
  \label{thm:1.4.3}
  $\forall a,b\in \mathbb{R}$,若 \(a < b\) , 则 \(\exists r\in \mathbb{Q}\) 满足 \(a < r < b\) 。
\end{Thm}

\begin{proof}
为了简化问题,我们假设 \(0 \leq  a < b\) 。 \(a < 0\) 的情况可以迅速由此得出(练习1.4.1)。有理数是整数的比值,因此我们必须构造 \(m,n \in  \mathbb{N}\) 使得
\begin{equation}
\label{eq:1.6}
a < \frac{m}{n} < b.
\end{equation}

第一步是选择足够大的分母 \(n\) ,使得大小为 \(1/n\) 的连续增量过于接近,以至于无法“跨越”区间$(a, b)$。


\begin{figure}[h]
  \centering
\includegraphics[width=0.6\textwidth]{images/01955a90-3fc5-700a-bdb0-df1c9e2a30b7_19_585_1128_785_117_0.jpg}  
\end{figure}


利用定理~\ref{thm:1.4.2},我们可以选择足够大的 \(n \in  \mathbb{N}\) ,使得

\begin{equation}
\label{eq:1.7}
\frac{1}{n} < b - a
\end{equation}

将不等式\eqref{eq:1.6}乘以 \(n\) 得到 \({na} < m < {nb}\) 。由于 \(n\) 已经选定,现在的思路是选择 \(m\) 为大于 \({na}\) 的最小自然数。换句话说,选择 \(m \in  \mathbb{N}\) 使得

\[
m - 1\le {na}  < m\text{ . }
\]

由 $na<m $立即得出 \(a < m/n\) ,这是成功的一半。记住不等式~\eqref{eq:1.7}等价于 \(a < b - 1/n\) ,我们可以利用$m-1\le na$写出
\begin{align*}
m \leq  &{na} + 1\\
< &n\left( {b - \frac{1}{n}}\right)  + 1\\
=& {nb}
\end{align*}

因为 \(m < {nb}\Rightarrow m/n < b\) ,所以我们有 \(a < m/n < b\) ,得证。  
\end{proof}

定理~\ref{thm:1.4.3}可以解释为 \(\mathbb{Q}\) 在 \(\mathbb{R}\) 中是稠密的。无需过多努力,我们可以利用这一结果证明无理数在 \(\mathbb{R}\) 中也是稠密的。


\begin{Cor}
  \label{cor:1.4.4}
  给定任意两个实数 \(a < b\) ,存在一个无理数 \(t\) 满足 \(a < t < b\) 。
\end{Cor}

\begin{proof}
  因为\(a < b\),由有理数的稠密性,存在有理数\(r\)满足\(a - \sqrt{2} < r < b - \sqrt{2}\)。令\(t = r + \sqrt{2}\),则显然有\(a < t < b\)。由于\(r\)是有理数且\(\sqrt{2}\)是无理数,\(t = r + \sqrt{2}\)必为无理数,得证。
\end{proof}


\subsection{平方根的存在性}

是时候处理一些在例~\ref{eg:1.3.6}和本章开头讨论中遗留的未完成事项了。

\begin{Thm}
  \label{thm:1.4.5}
 \(\exists \alpha  \in  \mathbb{R}\) 满足 \({\alpha }^{2} = 2\) 。
\end{Thm}

\begin{proof}
在回顾了例~\ref{eg:1.3.6}之后,考虑集合

\[
T = \left\{  {t \in  \mathbb{R} : {t}^{2} < 2}\right\}
\]

并设 \(\alpha  = \sup T\) 。我们将通过排除可能性 \({\alpha }^{2} < 2\) 和 \({\alpha }^{2} > 2\) 来证明 \({\alpha }^{2} = 2\) 。请记住, \(\sup T\) 的定义有两部分,它们都很重要。(当在论证中使用上确界时,这种情况总是会发生。)策略是证明 \({\alpha }^{2} < 2\) 违反了 \(\alpha\) 是 \(T\) 的上界这一事实,而 \({\alpha }^{2} > 2\) 违反了它是最小上界这一事实。

首先,我们看看如果假设 \({\alpha }^{2} < 2\) 会发生什么。为在 $T$ 中寻找一个大于 \(\alpha\) 的元素时,我们写下
\begin{align*}
{\left( \alpha  + \frac{1}{n}\right) }^{2} =& {\alpha }^{2} + \frac{2\alpha }{n} + \frac{1}{{n}^{2}}\\
<& {\alpha }^{2} + \frac{2\alpha }{n} + \frac{1}{n}\\
= & {\alpha }^{2} + \frac{{2\alpha } + 1}{n}\text{ . }
\end{align*}

但现在假设 \({\alpha }^{2} < 2\) 给了我们一点空间来容纳 \(\left( {{2\alpha } + 1}\right) /n\) 这项,并保持总和小于$2$。具体来说,只要选择足够大的 \({n}_{0} \in  \mathbb{N}\) 使得

\[
\frac{1}{{n}_{0}} < \frac{2 - {\alpha }^{2}}{{2\alpha } + 1}
\]

便有 \(\left( {{2\alpha } + 1}\right) /{n}_{0} < 2 - {\alpha }^{2}\) 。此时

\[
{\left( \alpha  + \frac{1}{{n}_{0}}\right) }^{2} < {\alpha }^{2} + \left( {2 - {\alpha }^{2}}\right)  = 2.
\]

因此, \(\alpha  + 1/{n}_{0} \in  T\) ,这与 \(\alpha\) 是 \(T\) 的上界的事实相矛盾。我们得出结论, \({\alpha }^{2} < 2\) 不可能发生。

现在,关于 \({\alpha }^{2} > 2\) 的情况呢?这次,写下
\begin{align*}
{\left( \alpha  - \frac{1}{n}\right) }^{2} = & {\alpha }^{2} - \frac{2\alpha }{n} + \frac{1}{{n}^{2}}\\
> & {\alpha }^{2} - \frac{2\alpha }{n}.
\end{align*}
\end{proof}


对这个证明稍作修改,可以用来证明 \(\sqrt{x}\) 对于任何 \(x \geq  0\) 都存在。利用二项式定理展开 \({\left( \alpha  + 1/n\right) }^{m}\) 可以用来证明 \(\sqrt[m]{x}\) 对于任意 \(m \in  \mathbb{N}\) 的值都存在。

\subsection{可数集与不可数集}

到目前为止,完备性公理的应用基本上恢复了我们对实数系统的直观性质的信心。我们即将提出的完备性的最后一个结果则性质迥异,并且本身代表了一项惊人的智力发现。传统上,数学进步是通过数学家修改和扩展前人的工作来完成的。这一模型似乎并不适用于 Georg Cantor (1845-1918),至少在他关于无限集理论的工作上是如此。

目前,我们将 \(\mathbb{R}\) 视为由有理数和无理数组成,沿着实数轴连续紧密排列。我们已经看到, \(\mathbb{Q}\) 和 \(\mathbb{R}\setminus\mathbb{Q}\) 在 \(\mathbb{R}\) 中都是稠密的,这意味着在每一个区间$(a, b)$中,都存在有理数和无理数。在心理上,我们倾向于认为 \(\mathbb{Q}\) 和 \(\mathbb{R}\setminus\mathbb{Q}\) 以相等的比例错综复杂地混合在一起,但事实并非如此。正如 Cantor 精确指出的那样,在构成实数线的过程中,无理数的数量远远超过有理数。

\subsection{基数}

基数(cardinality)这一术语在数学中用于指代集合的大小。有限集合的基数可以通过为每个集合附加一个自然数来简单比较。《白雪公主》中的小矮人的集合比美国最高法院大法官集合小,因为7小于9。但如果不借助任何数字,我们如何得出同样的结论呢?Cantor 的想法是尝试将集合彼此建立一一对应关系。小矮人的数量少于大法官,因为如果所有小矮人同时被任命到法官席上,仍然会有两个空位需要填补。另一方面,最高法院的基数与棒球队的守场员集合的基数相同。这是因为当法官们上场时,可以安排他们使得每个位置恰好有一名法官。

这种比较集合大小的方法的优势在于,它同样适用于无限集合。

\begin{Def}
  \label{def:1.4.6}
称一个函数 \(f : A \rightarrow  B\) 是单射,如果 \(\forall a_1, a_2\in A, {a}_{1} \neq  {a}_{2}\Rightarrow f\left( {a}_{1}\right)  \neq  f\left( {a}_{2}\right)\) 。称函数 \(f\) 是满射,如果 \(\forall b \in  B\) ,都 \(\exists a \in  A\) 使得 \(f\left( a\right)  = b\) 。
\end{Def}

一个既是单射又是满射的函数 \(f : A \rightarrow  B\),正是我们所说的两个集合之间的一一对应关系。单射的性质意味着 \(A\) 中的任何两个元素都不会对应到 \(B\) 中的同一个元素(没有两个法官担任相同的职位),而满射的性质则确保 \(B\) 中的每一个元素都对应到 \(A\) 中的某个元素(每个职位上都有一位法官)。

\begin{Def}
  \label{def:1.4.7}
  如果存在一个既是单射又是满射的函数 \(f : A \rightarrow  B\) ,则两个集合 \(A\) 和 \(B\) 具有相同的基数。在这种情况下,我们记作 \(A \sim  B\) 。
\end{Def}

\begin{Eg}
  \label{eg:1.4.8}
  \begin{enumerate}[label = (\roman*)]
  \item 如果我们让 \(E = \{ 2,4,6,\ldots \}\) 表示偶数自然数的集合,那么我们可以证明 \(\mathbb{N} \sim  E\)。为了理解这一点,让 \(f : \mathbb{N} \rightarrow  E\) 由 \(f\left( n\right)  = {2n}\) 给出。

    \begin{figure}[h]
      \centering
      \includegraphics[width = 0.6\linewidth]{images/1.png}
    \end{figure}

确实, \(E\) 是 \(\mathbb{N}\) 的一个真子集,因此似乎有理由说 \(E\) 是一个“较小”的集合。这是一种看待问题的方式,但它代表了一种由于过度接触有限集而产生的强烈偏见观点。基数的定义非常具体,从这个角度来看, \(E\) 和 \(\mathbb{N}\) 是等价的。
\item 为了再次强调这一点,请注意,尽管 \(\mathbb{N}\) 作为真子集包含在 \(\mathbb{Z}\) 中,我们可以证明 \(\mathbb{N} \sim  \mathbb{Z}\) 。这次让

\[
f\left( n\right)  = \left\{  \begin{array}{ll} \left( {n - 1}\right) /2 & n\text{是奇数} \\   - n/2 & n\text{是偶数} \end{array}\right.
\]

需要验证的重要细节是, \(f\) 不会将任何两个自然数映射到 \(\mathbb{Z}\) 的同一元素(即 $f$ 是单射),并且 \(\mathbb{Z}\) 的每个元素都会被 \(\mathbb{N}\) 中的某个元素“击中”( 即\(f\) 是满射)。

\begin{figure}[h]
  \centering
  \includegraphics[width = 0.6\linewidth]{images/2.png}
\end{figure}


  \end{enumerate}
\end{Eg}

\begin{figure}[h]
  \centering
  \includegraphics[width=0.3\textwidth]{images/01955a90-3fc5-700a-bdb0-df1c9e2a30b7_23_713_350_482_610_0.jpg}
  \caption{ \(\left( {-1,1}\right)  \sim  \mathbb{R}\) 使用 \(f\left( x\right)  = x/\left( {{x}^{2} - 1}\right)\) }
  \label{fig:1.4}
\end{figure}

\begin{Eg}
  \label{eg:1.4.9}
一些微积分(我们不会提供细节)表明,函数 \(f\left( x\right)  = x/\left( {{x}^{2} - 1}\right)\) 将区间$(-1,1)$以单射的方式映射到 \(\mathbb{R}\) (见图1.4)。因此 \(\left( {-1,1}\right)  \sim  \mathbb{R}\) 。事实上, \(\left( {a,b}\right)  \sim  \mathbb{R}\) 对于任何区间$(a, b)$都成立。  
\end{Eg}

\subsection{可数集}
\begin{Def}
  \label{def:1.4.10}
称集合 \(A\) 是可数的,如果 \(\mathbb{N} \sim  A\) 。不可数的无限集称为不可数集。  
\end{Def}

从例~\ref{eg:1.4.8}中,我们看到 \(E\) 和 \(\mathbb{Z}\) 都是可数集。将一个集合与 \(\mathbb{N}\) 建立一一对应关系,实际上意味着将所有元素放入一个无限长的列表或序列中。观察例~\ref{eg:1.4.8},我们可以看出这对于 \(E\) 来说相当容易,而对于集合 \(\mathbb{Z}\) 只需要稍微调整一下顺序。一个自然的问题是,是否所有无限集都是可数的。给定一些无限集,如 \(\mathbb{Q}\) 或 \(\mathbb{R}\) ,似乎只要我们足够聪明,就应该能够将集合中的所有元素放入一个单一的列表中(即与 \(\mathbb{N}\) 建立对应关系)。毕竟,这个列表是无限长的,所以应该有足够的空间。但遗憾的是,正如Hardy所言,“[数学家的]学科是所有学科中最奇特的——没有哪个学科中的真理会如此捉弄人。”

\begin{Thm}
  \label{thm:1.4.11}
\begin{enumerate}[label = (\roman*)]
\item\label{item:1.4.3} 集合 \(\mathbb{Q}\) 是可数的。 
\item\label{item:1.4.4} 集合 \(\mathbb{R}\) 是不可数的。
\end{enumerate}  
\end{Thm}

\begin{proof}
\ref{item:1.4.3} \(\forall n \in  \mathbb{N}\) ,令 \({A}_{n}\) 为由以下给出的集合

\[
  {A}_{n} = \left\{  \pm \frac{p}{q} \mid p,q\in \mathbb{N}, p+q = n, \gcd(p,q) = 1\right\}  .
\]

这些集合的前几个形如

\[
{A}_{1} = \left\{  \frac{0}{1}\right\}  ,\;{A}_{2} = \left\{  {\frac{1}{1},\frac{-1}{1}}\right\}  ,\;{A}_{3} = \left\{  {\frac{1}{2},\frac{-1}{2},\frac{2}{1},\frac{-2}{1}}\right\}  ,
\]

\[
{A}_{4} = \left\{  {\frac{1}{3},\frac{-1}{3},\frac{3}{1},\frac{-3}{1}}\right\}  ,{A}_{5} = \left\{  {\frac{1}{4},\frac{-1}{4},\frac{2}{3},\frac{-2}{3},\frac{3}{2},\frac{-3}{2},\frac{4}{1},\frac{-4}{1}}\right\}  .
\]

关键观察是每个 \({A}_{n}\) 都是有限的,并且每个有理数恰好出现在这些集合中的一个。我们与 \(\mathbb{N}\) 的一一对应关系是通过依次列出每个 \({A}_{n}\) 中的元素来实现的。

\begin{figure}[h]
  \centering
  \includegraphics[width=0.6\textwidth]{images/01955a90-3fc5-700a-bdb0-df1c9e2a30b7_24_394_758_844_216_0.jpg}
\end{figure}


诚然,为这种对应关系写出一个明确的公式将是一项笨拙的任务,尝试这样做并不是好的消磨时间的方式。重要的是我们理解为什么每个有理数在对应关系中恰好出现一次。例如,给定 \({22}/7\in \mathbb{Q}\) ,我们有 \({22}/7 \in  {A}_{29}\) 。由于 \({A}_{1},\ldots ,{A}_{28}\) 中的元素集是有限的,我们可以确信 \({22}/7\) 最终会被包含在序列中。这一推理适用于任何有理数 \(p/q\) ,从而证明了对应关系是满射的。为了验证它是单射 ,我们观察到集合 \({A}_{n}\) 被构造为互不相交的,因此没有有理数会重复出现。这就完成了~\ref{item:1.4.3}的证明。

~\ref{item:1.4.4}定理~\ref{thm:1.4.11}的第二个陈述才是真正出人意料的部分。对其证明,我们采用反证法。假设确实存在一个双射 \(f : \mathbb{N} \rightarrow  \mathbb{R}\) 。再次强调,这意味着可以枚举 \(\mathbb{R}\) 的元素。如果我们设 \({x}_1 = f\left( 1\right) ,{x}_2 = f\left( 2\right)\) ……依此类推,那么我们关于 \(f\) 是满射的假设意味着我们可以写出

\begin{equation}
\label{eq:1.8}
\mathbb{R} = \left\{  {{x}_{1},{x}_{2},{x}_{3},{x}_{4},\ldots }\right\}
\end{equation}

并且确信每个实数都会出现在列表的某个位置。我们现在将使用闭区间套定理(定理~\ref{thm:1.4.1})来生成一个不在列表中的实数。

设 \({I}_{1}\) 为一个不包含 \({x}_{1}\) 的闭区间。接着,设 \({I}_{2}\) 为一个包含在 \({I}_{1}\) 中且不包含 \({x}_{2}\) 的闭区间。这样的 \({I}_{2}\) 的存在性很容易验证。显然, \({I}_{1}\) 包含两个较小的不相交闭区间,而 \({x}_{2}\) 只能位于其中一个。一般情况下,给定一个区间 \({I}_{n}\) ,构造 \({I}_{n + 1}\) 以满足
\begin{enumerate}[label = (\roman*)]
\item  \({I}_{n + 1} \subseteq  {I}_{n}\)
\item \({x}_{n + 1} \notin  {I}_{n + 1}\) 
\end{enumerate}


\begin{figure}[h]
  \centering
  \includegraphics[width=0.3\textwidth]{images/01955a90-3fc5-700a-bdb0-df1c9e2a30b7_25_745_362_463_167_0.jpg}
\end{figure}


我们现在考虑交集 \(\mathop{\bigcap }\limits_{{n = 1}}^{\infty }{I}_{n}\) 。如果 \({x}_{{n}_{0}}\) 是列表\eqref{eq:1.8}中的某个实数,那么我们有 \({x}_{{n}_{0}} \notin  {I}_{{n}_{0}}\) ,由此可得

\[
{x}_{{n}_{0}} \notin  \mathop{\bigcap }\limits_{{n = 1}}^{\infty }{I}_{n}
\]

现在,我们假设列表\eqref{eq:1.8}包含所有实数,这导致结论

\[
\mathop{\bigcap }\limits_{{n = 1}}^{\infty }{I}_{n} = \varnothing
\]

然而,闭区间套定理断言 \(\mathop{\bigcap }\limits_{{n = 1}}^{\infty }{I}_{n} \neq  \varnothing\) 。根据闭区间套定理,至少存在一个 \(x \in  \mathop{\bigcap }\limits_{{n = 1}}^{\infty }{I}_{n}\) ,因此,它不可能在列表\eqref{eq:1.8}中。这一矛盾意味着对 \(\mathbb{R}\) 进行枚举是不可能的,我们得出结论 \(\mathbb{R}\) 是一个不可数集。  
\end{proof}

我们究竟该如何理解这一发现?事实上,任何可数集的子集必须是可数的或有限的。这并不应令人感到过于惊讶。如果一个集合可以被排列成一个单一的列表,那么从这个列表中删除一些元素将得到另一个(更短且可能终止的)列表。这意味着可数集是最小的无限集类型。任何比它更小的集合要么仍然是可数的,要么是有限的。

非正式地说,定理~\ref{thm:1.4.11}的核心在于, \(\mathbb{R}\) 的基数是一种更大的无穷类型。实数在数量上远超自然数,以至于无法将 \(\mathbb{N}\) 映射到 \(\mathbb{R}\) 上。无论我们如何尝试,总会有多余的实数存在。另一方面,集合 \(\mathbb{Q}\) 是可数的。就无限集而言,这是最小的规模。这对无理数集 \(\mathbb{R}\setminus\mathbb{Q}\) 意味着什么?通过模仿 \(\mathbb{N} \sim  \mathbb{Z}\) 的证明,我们可以证明两个可数集的并集必须是可数的。由于 \(\mathbb{R} = \mathbb{Q} \cup  (\mathbb{R}\setminus\mathbb{Q})\) ,因此 \(\mathbb{R}\setminus\mathbb{Q}\) 不可能是可数的,否则 \(\mathbb{R}\) 也会是可数的。不可避免的结论是,尽管我们遇到的无理数很少,但它们在 \(\mathbb{R}\) 中形成的子集远比 \(\mathbb{Q}\) 大得多。

本讨论中描述的可数集性质对于后续章节中的一些练习非常有用。为便于参考,我们将它们陈述为一些最终命题,并在接下来的练习中概述其证明。

\begin{Thm}
  \label{thm:1.4.12}
  如果 \(A \subseteq  B\) 且 \(B\) 是可数的,那么 \(A\) 要么是可数的,要么是有限的,要么是空的。
\end{Thm}


\begin{Thm}
  \label{thm:1.4.13}
  \begin{enumerate}[label = (\roman*)]
  \item 如果 \({A}_{1},{A}_{2},\ldots {A}_{m}\) 都是可数集,那么并集 \({A}_{1} \cup  {A}_{2} \cup  \cdots  \cup  {A}_{m}\) 是可数的。
  \item 如果对于每个 \(n \in  \mathbb{N}\) , \({A}_{n}\) 都是可数集,那么 \(\mathop{\bigcup }\limits_{{n = 1}}^{\infty }{A}_{n}\) 是可数的。
  \end{enumerate}
\end{Thm}


\subsection{练习}

练习1.4.1。在不做太多工作的情况下,展示如何通过将此情况转换为已证明的情况来证明定理1.4.3。

练习1.4.2。回想一下, \(\mathbb{R}\setminus\mathbb{Q}\) 代表无理数集。

(a) 证明如果 \(a,b \in  \mathbb{Q}\) ,那么 \({ab}\) 和 \(a + b\) 也是 \(\mathbb{Q}\) 的元素。

(b) 证明如果 \(a \in  \mathbb{Q}\) 和 \(t \in  \mathbb{R}\setminus\mathbb{Q}\) ,那么只要 \(a \neq  0\) , \(a + t \in  \mathbb{R}\setminus\mathbb{Q}\) 和 \({at} \in  \mathbb{R}\setminus\mathbb{Q}\) 也成立。

(c) 部分 (a) 可以总结为 \(\mathbb{Q}\) 在加法和乘法下是封闭的。 \(\mathbb{R}\setminus\mathbb{Q}\) 在加法和乘法下是否封闭?给定两个无理数 \(s\) 和 \(t\) ,我们可以对 \(s + t\) 和 \({st}\) 说些什么?

练习 1.4.3。使用练习 1.4.2,通过将定理 1.4.3 应用于实数 \(a - \sqrt{2}\) 和 \(b - \sqrt{2}\) ,为推论 1.4.4 提供一个证明。

练习 1.4.4。使用 \(\mathbb{R}\) 的阿基米德性质来严格证明 \(\inf \{ 1/n : n \in  \mathbb{N}\}  = 0\) 。

练习1.4.5。证明 \(\mathop{\bigcap }\limits_{{n = 1}}^{\infty }\left( {0,1/n}\right)  = \varnothing\) 。注意,这表明闭区间套定理中的区间必须是闭的,以便定理的结论成立。

练习1.4.6。(a) 通过展示假设 \({\alpha }^{2} > 2\) 导致与事实 \(\alpha  = \sup T\) 矛盾的结论,完成定理1.4.5的证明。

(b) 修改此论证以证明对于任何实数 \(b \geq  0\) , \(\sqrt{b}\) 的存在性。

练习1.4.7。完成定理1.4.12的以下证明。

假设 \(B\) 是一个可数集。因此,存在 \(f : \mathbb{N} \rightarrow  B\) ,它是 \(1 - 1\) 且满射。设 \(A \subseteq  B\) 是 \(B\) 的一个无限子集。我们必须证明 \(A\) 是可数的。

设 \({n}_{1} = \min \{ n \in  \mathbb{N} : f\left( n\right)  \in  A\}\) 。作为 \(g : \mathbb{N} \rightarrow  A\) 定义的开端,设 \(g\left( 1\right)  = f\left( {n}_{1}\right)\) 。展示如何归纳地继续这个过程,以生成一个从 \(\mathbb{N}\) 到 \(A\) 的一对一函数 \(g\) 。

练习1.4.8。使用以下大纲为定理1.4.13中的陈述提供证明。

首先,证明两个可数集合 \({A}_{1}\) 和 \({A}_{2}\) 的陈述(i)。例1.4.8(ii)可能是一个有用的参考。通过首先将 \({A}_{2}\) 替换为集合 \({B}_{2} = {A}_{2} \smallsetminus  {A}_{1} = \left\{  {x \in  {A}_{2} : x \notin  {A}_{1}}\right\}\) ,可以避免一些技术性问题。这样做的目的是使并集 \({A}_{1} \cup  {B}_{2}\) 等于 \({A}_{1} \cup  {A}_{2}\) ,并且集合 \({A}_{1}\) 和 \({B}_{2}\) 是不相交的。(如果 \({B}_{2}\) 是有限的,会发生什么?)

现在,解释(i)中更一般的陈述是如何得出的。

解释为什么不能使用归纳法从(i)部分证明定理1.4.13的(ii)部分。

(c) 展示如何将 \(\mathbb{N}\) 排列成二维数组

\[
\begin{array}{llllll} 1 & 3 & 6 & {10} & {15} & \cdots  \end{array}
\]

\[
\begin{array}{lllll} 2 & 5 & 9 & {14} & \cdots  \end{array}
\]

\[
\begin{array}{llll} 4 & 8 & {13} & \cdots  \end{array}
\]

\[
\begin{array}{lll} 7 & {12} & \cdots  \end{array}
\]

11 ...

\(\vdots\)

从而证明定理1.4.13 (ii)。

练习1.4.9. (a) 给定集合 \(A\) 和 \(B\) ,解释为什么 \(A \sim  B\) 等价于断言 \(B \sim  A\) 。

(b) 对于三个集合 \(A,B\) 、 \(C\) ,证明 \(A \sim  B\) 和 \(B \sim  C\) 蕴含 \(A \sim  C\) 。这两个性质意味着 \(\sim\) 是一个等价关系。

练习1.4.10. 证明 \(\mathbb{N}\) 的所有有限子集的集合是一个可数集。(事实上, \(\mathbb{N}\) 的所有子集的集合不是一个可数集。这是1.5节的主题。)

练习1.4.11。考虑开区间(0,1),并设 \(S\) 为开单位正方形中的点集;即 \(S = \{ \left( {x,y}\right)  : 0 < x,y < 1\}\) 。

(a) 找到一个将(0,1)映射到 \(S\) 的一对一函数,但不一定满射。(这很容易。)

(b) 利用每个实数都有小数展开的事实,构造一个将 \(S\) 映射到(0,1)的一对一函数。讨论所构造的函数是否为满射。(请记住,任何终止小数展开,如.235,与.234999...表示相同的实数。)

接下来在练习1.4.13中讨论的施罗德-伯恩斯坦定理现在可以应用于得出结论 \(\left( {0,1}\right)  \sim  S\) 。

练习 1.4.12. 一个实数 \(x \in  \mathbb{R}\) 被称为代数数,如果存在不全为零的整数 \({a}_{0},{a}_{1},{a}_{2},\ldots ,{a}_{n} \in  \mathbb{Z}\) ,使得

\[
{a}_{n}{x}^{n} + {a}_{{n}_{1}}{x}^{n - 1} + \cdots  + {a}_{1}x + {a}_{0} = 0.
\]

换句话说,如果一个实数是具有整数系数的多项式的根,那么它就是代数数。不是代数数的实数被称为超越数。重读第1.1节的最后一段。这里提出的最后一个问题与超越数是否存在密切相关。

(a) 证明 \(\sqrt{2},\sqrt[3]{2}\) 和 \(\sqrt{3} + \sqrt{2}\) 是代数数。

(b) 固定 \(n \in  \mathbb{N}\) ,并令 \({A}_{n}\) 为作为具有整数系数的多项式根所得到的代数数,这些多项式的次数为 \(n\) 。利用每个多项式都有有限个根的事实,证明 \({A}_{n}\) 是可数的。(对于每个 \(m \in  \mathbb{N}\) ,考虑满足 \(\left. {\left| {a}_{n}\right|  + \left| {a}_{n - 1}\right|  + \cdots  + \left| {a}_{1}\right|  + \left| {a}_{0}\right|  \leq  m.}\right)\) 的多项式 \({a}_{n}{x}^{n} + {a}_{{n}_{1}}{x}^{n - 1} + \cdots  + {a}_{1}x + {a}_{0}\) )

(c) 现在,论证所有代数数的集合是可数的。关于超越数集合,我们可以得出什么结论?

练习 1.4.13(施罗德-伯恩斯坦定理)。假设存在一个一对一函数 \(f : X \rightarrow  Y\) 和另一个 \(1 - 1\) 函数 \(g : Y \rightarrow  X\) 。按照步骤证明存在一个 \(1 - 1\) 且满射的函数 \(h : X \rightarrow  Y\) ,因此 \(X \sim  Y\) 。

(a) \(f\) 的范围由 \(f\left( X\right)  = \{ y \in  Y : y = f\left( x\right)\) 定义,其中 \(x \in  X\}\) 。设 \(y \in  f\left( X\right)\) 。(因为 \(f\) 不一定是满射,范围 \(f\left( X\right)\) 可能不是 \(Y\) 的全部。)解释为什么存在唯一的 \(x \in  X\) 使得 \(f\left( x\right)  = y\) 。现在定义 \({f}^{-1}\left( y\right)  = x\) ,并证明 \({f}^{-1}\) 是从 \(f\left( X\right)\) 到 \(X\) 的一对一函数。

以类似的方式,我们也可以定义 \(1 - 1\) 函数 \({g}^{-1} : g\left( X\right)  \rightarrow  Y\) 。

(b) 设 \(x \in  X\) 为任意元素。令链 \({C}_{x}\) 为由所有形如以下元素的集合组成

(1)
\[
\ldots ,{f}^{-1}\left( {{g}^{-1}\left( x\right) }\right) ,{g}^{-1}\left( x\right) ,x,f\left( x\right) ,g\left( {f\left( x\right) }\right) ,f\left( {g\left( {f\left( x\right) }\right) }\right) ,\ldots .
\]

解释为什么在上述链中, \(x\) 左侧的元素数量可能为零、有限或无限。

(c) 证明任意两条链要么完全相同,要么完全不相交。

(d) 注意在 (1) 中的链的项在 \(X\) 的元素和 \(Y\) 的元素之间交替。给定一条链 \({C}_{x}\) ,我们希望关注 \({C}_{x} \cap  Y\) ,它只是链中位于 \(Y\) 的部分。

定义集合 \(A\) 为所有满足 \({C}_{x} \cap  Y \subseteq  f\left( X\right)\) 的链 \({C}_{x}\) 的并集。令 \(B\) 由不在 \(A\) 中的剩余链的并集组成。证明任何包含在 \(B\) 中的链必须具有以下形式

\[
y,g\left( y\right) ,f\left( {g\left( y\right) }\right) ,g\left( {f\left( {g\left( y\right) }\right) }\right) ,\ldots ,
\]

其中 \(y\) 是 \(Y\) 的一个元素,且不在 \(f\left( X\right)\) 中。

(e) 设 \({X}_{1} = A \cap  X,{X}_{2} = B \cap  X,{Y}_{1} = A \cap  Y\) ,且 \({Y}_{2} = B \cap  Y\) 。证明 \(f\) 将 \({X}_{1}\) 映射到 \({Y}_{1}\) ,且 \(g\) 将 \({Y}_{2}\) 映射到 \({X}_{2}\) 。利用此信息证明 \(X \sim  Y\) 。

\section{Cantor 定理}
\label{sec:1.5}
Cantor 在无限集理论中的工作远远超出了定理~\ref{thm:1.4.11}的结论。尽管最初遭到抵制,他在这一领域的创造性和不懈努力最终引发了集合论的一场革命,并改变了数学家对无限的理解方式。

\subsection{Cantor 的对角线方法}

定理~\ref{thm:1.4.11} \ref{item:1.4.4}的证明与 Cantor 为这一结果提供的任何论证都不同。之所以选择这个证明,是因为它直接揭示了不可数性与完备性概念之间的联系,并且因为使用嵌套区间的技术将在我们后续的工作中多次使用。

Cantor 最初在1874年发表了 \(\mathbb{R}\) 不可数的发现,但在1891年,他提供了另一个证明,其简洁性令人惊讶。这个证明依赖于实数的十进制表示,我们将不加任何正式定义地接受并使用它。

\begin{Thm}
\label{thm:1.5.1}
  开区间 \(\left( {0,1}\right)  = \{ x \in  \mathbb{R} : 0 < x < 1\}\) 是不可数的。
\end{Thm}

练习1.5.1. 证明$(0,1)$不可数当且仅当 \(\mathbb{R}\) 不可数。这表明定理~\ref{thm:1.5.1}与定理~\ref{thm:1.4.11}是等价的。

\begin{proof}
与定理~\ref{thm:1.4.11}类似,我们采用反证法,假设确实存在一个函数 \(f : \mathbb{N} \rightarrow  \left( {0,1}\right)\) ,它是单射且满射的。对于每个 \(m \in  \mathbb{N}\) , \(f\left( m\right)\) 是一个介于$0$和$1$之间的实数,我们使用十进制表示法来表示它。

\[
f\left( m\right)  =  \cdot  {a}_{m1}{a}_{m2}{a}_{m3}{a}_{m4}{a}_{m5}\ldots
\]

这里的意思是,对于每个 \(m,n \in  \mathbb{N},{a}_{mn}\) 是来自集合 \(\{ 0,1,2,\ldots ,9\}\) 的数字,表示 \(f\left( m\right)\) 的十进制展开中的第 \(n\) 位数字。 \(\mathbb{N}\) 与$(0,1)$之间的一一对应关系可以用这样的双索引数组来表示。

\begin{figure}[h]
  \centering
  \includegraphics[width = 0.8\linewidth]{images/3.png}
\end{figure}

关于这一对应的关键假设是,假设$(0,1)$区间内的每一个实数都会出现在列表中的某个位置。

现在进入论证的核心部分。定义一个实数 \(x \in  \left( {0,1}\right)\) ,其小数展开 \(x = {b}_{1}{b}_{2}{b}_{3}{b}_{4}\ldots\) 遵循以下规则

\[
{b}_{n} = \left\{  \begin{array}{ll} 2 & {a}_{nn} \neq  2 \\  3 & {a}_{nn} = 2. \end{array}\right.
\]

让我们明确这一点。为了计算数字 \({b}_{1}\) ,我们查看数组左上角的数字 \({a}_{11}\) 。如果 \({a}_{11} = 2\) ,则选择 \({b}_{1} = 3\) ;否则,设定 \({b}_{1} = 2\) 。

练习1.5.2. (a) 解释为什么实数 \(x = {.b}_{1}{b}_{2}{b}_{3}{b}_{4}\ldots\) 不能是 \(f\left( 1\right)\) 。

(b) 现在,解释为什么 \(x \neq  f\left( 2\right)\) ,以及一般来说为什么对于任何 \(n \in  \mathbb{N}\) , \(x \neq  f\left( n\right)\) 成立。

(c) 指出这些观察中产生的矛盾,并得出结论:$(0,1)$是不可数的。
  
\end{proof}

练习 1.5.3. 对以下关于定理 1.5.1 证明的质疑提供反驳。

(a) 每个有理数都有一个小数展开,因此我们可以应用相同的论点来证明0到1之间的有理数集是不可数的。然而,因为我们知道 \(\mathbb{Q}\) 的任何子集都必须是可数的,所以定理~\ref{thm:1.5.1}的证明一定有缺陷。

(b) 一些数字有两种不同的小数表示。具体来说,任何终止的小数展开也可以用重复的9表示。例如, \(1/2\) 可以写成.5或.4999.... 这不会导致一些问题吗?

练习1.5.4。设 \(S\) 为由所有0和1的序列组成的集合。注意, \(S\) 不是一个特定的序列,而是一个包含序列作为元素的大集合;即,

\[
S = \left\{  {\left( {{a}_{1},{a}_{2},{a}_{3},\ldots }\right)  : {a}_{n} = 0\text{ or }1}\right\}  .
\]

例如,序列 \(\left( {1,0,1,0,1,0,1,0,\ldots }\right)\) 是 \(S\) 的一个元素,序列 \(\left( {1,1,1,1,1,1,\ldots }\right)\) 也是。

给出一个严格的论证,证明 \(S\) 是不可数的。

在区分了可数无穷( \(\mathbb{N}\) )和不可数无穷( \(\mathbb{R}\) )之后,Cantor 面临的一个新问题是,是否存在一种“高于” \(\mathbb{R}\) 的无穷。这在逻辑上是一个危险的领域。我们在定义“具有相同基数”的关系时所给予的同样谨慎,也需要用于定义诸如“基数大于”或“基数小于或等于”的关系。然而,即使不陷入形式定义的泥潭,从下一个结果中我们可以非常清楚地感受到,存在一个超越 \(\mathbb{R}\) 连续统的无穷集合的层级结构。

\subsection{幂集与 Cantor 定理}

给定一个集合 \(A\) ,幂集 \(P\left( A\right)\) 指的是 \(A\) 的所有子集的集合。注意, \(P\left( A\right)\) 本身被视为一个集合,其元素是 \(A\) 的不同可能子集。

练习1.5.5. (a) 设 \(A = \{ a,b,c\}\) 。列出 \(P\left( A\right)\) 的八个元素。(不要忘记 \(\varnothing\) 被视为每个集合的子集。)

(b) 如果 \(A\) 是有限的,且有 \(n\) 个元素,证明 \(P\left( A\right)\) 有 \({2}^{n}\) 个元素。(构造 \(A\) 的一个特定子集可以解释为关于是否包含 \(A\) 的每个元素的一系列决策。)

练习1.5.6. (a) 使用特定集合 \(A = \{ a,b,c\}\) ,展示从 \(A\) 到 \(P\left( A\right)\) 的两个不同的1-1映射。

(b) 设 \(B = \{ 1,2,3,4\}\) ,生成一个 \(1 - 1\) 映射 \(g : B \rightarrow  P\left( B\right)\) 的例子。

解释为什么在(a)和(b)部分中,不可能构造出满射映射。

Cantor 定理指出,练习1.5.6中的现象不仅适用于有限集,也适用于无限集。尽管将 \(A\) 映射到 \(P\left( A\right)\) 相当容易,但找到一个满射映射是不可能的。

\begin{Thm}[Cantor]
  \label{thm:1.5.2}
  给定任何集合 \(A\) ,不存在一个满射函数 \(f : A \rightarrow  P\left( A\right)\) 。
\end{Thm}

\begin{proof}
与同类证明一样,此证明采用间接法。反设 \(f : A \rightarrow  P\left( A\right)\) 是满射的,以期引出矛盾。与通常我们处理数字集合作为定义域和值域的情况不同, \(f\) 是一个集合与其幂集之间的对应关系。对于每个元素 \(a \in  A,f\left( a\right)\) ,它是 \(A\) 的一个特定子集。假设 \(f\) 是满射的,意味着 \(A\) 的每一个子集都作为 \(f\left( a\right)\) 出现,对应于某个 \(a \in  A\) 。为了引出矛盾,我们将构造一个子集 \(B \subseteq  A\) ,它不等于任何 \(a \in  A\) 对应的 \(f\left( a\right)\) 。

按照以下规则构造 \(B\) 。对于每个元素 \(a \in  A\) ,考虑子集 \(f\left( a\right)\) 。这个 \(A\) 的子集可能包含元素 \(a\) ,也可能不包含。这取决于函数 \(f\) 。如果 \(f\left( a\right)\) 不包含 \(a\) ,那么我们将 \(a\) 包含在我们的集合 \(B\) 中。更准确地说,设

\[
B = \{ a \in  A : a \notin  f\left( a\right) \} .
\]

练习1.5.7。回到练习1.5.6中构造的特定函数,并使用前面的规则构造得到的子集 \(B\) 。在每种情况下,注意 \(B\) 不在所用函数的值域内。

我们现在关注一般性论证。由于我们假设函数 \(f : A \rightarrow  P\left( A\right)\) 是满射的,因此对于某个 \({a}^{\prime } \in  A\) ,必须有 \(B = f\left( {a}^{\prime }\right)\) 。当我们考虑 \({a}^{\prime }\) 是否是 \(B\) 的元素时,矛盾就出现了。

练习1.5.8。(a) 首先,证明情况 \({a}^{\prime } \in  B\) 会导致矛盾。

(b) 现在,通过证明情况 \({a}^{\prime } \notin  B\) 同样不可接受来完成论证。  
\end{proof}

练习1.5.9。作为最后一个练习,通过建立与已知基数集合的一一对应关系来回答以下每个问题。

(a) 从 \(\{ 0,1\}\) 到 \(\mathbb{N}\) 的所有函数集合是可数的还是不可数的?

(b) 从 \(\mathbb{N}\) 到 \(\{ 0,1\}\) 的所有函数的集合是可数的还是不可数的?

(c) 给定一个集合 \(B\) , \(P\left( B\right)\) 的一个子集 \(\mathcal{A}\) 被称为一个反链,如果 \(\mathcal{A}\) 的任何元素都不是 \(\mathcal{A}\) 中任何其他元素的子集。 \(P\left( \mathbb{N}\right)\) 是否包含一个不可数的反链?

\section{结语}
\label{sec:1.6}
具有相同基数的关系是一种等价关系(参见练习1.4.9),大致意味着宇宙中的所有集合可以根据它们的大小组织成不相交的组。两个集合出现在同一个组或等价类中,当且仅当它们具有相同的基数。因此, \(\mathbb{N},\mathbb{Z}\) 和 \(\mathbb{Q}\) 与所有其他可数集合一起被分在同一类中,而 \(\mathbb{R}\) 则在另一个类中,该类包括区间$(0,1)$以及其他不可数集合。Cantor定理的一个含义是, \(P\left( \mathbb{R}\right)\) —— \(\mathbb{R}\) 的所有子集的集合——与 \(\mathbb{R}\) 不在同一类中,而且没有理由在此停止。 \(P\left( \mathbb{R}\right)\) 的子集集合——即 \(P\left( {P\left( \mathbb{R}\right) }\right)\) ——又在另一个类中,这个过程无限延续。

将集合的宇宙划分为不相交的组后,为每个集合附加一个“数字”将非常方便,这个数字可以像自然数一样用来表示有限集的大小。给定一个集合 \(X\) ,存在一个称为 \(X\) 的基数(cardinal number)的东西,记作card \(X\) ,它的行为非常符合这种模式。例如,两个集合 \(X\) 和 \(Y\) 满足 \(\operatorname{card}X = \operatorname{card}Y\) 当且仅当 \(X \sim  Y\) 。(严格定义card \(X\) 需要一些重要的集合论知识。一种方法是定义card \(X\) 为一个非常特定的集合,该集合总是可以在与 \(X\) 相同的等价类中唯一找到。)

回顾 Cantor 定理,我们强烈地感觉到,无穷集合的大小存在一种顺序,这种顺序应该在我们新的基数系统中得到体现。具体来说,如果能够以一一对应的方式将集合 \(X\) 映射到 \(Y\) 中,那么我们希望 \(\operatorname{card}X \leq  \operatorname{card}Y\) 。写成严格不等式 \(\operatorname{card}X < \operatorname{card}Y\) 应该表明,可以将 \(X\) 映射到 \(Y\) 中,但无法证明 \(X \sim  Y\) 。用这种符号重新表述,Cantor定理指出,对于每个集合 \(A\) ,有基数 \(A < \operatorname{card}P\left( A\right)\) 。

有一些重要的细节需要解决。当我们意识到 Cantor 定理的一个含义是不存在“最大”集合时,一种形而上学的问题便出现了。诸如“设 \(U\) 为所有可能事物的集合”这样的声明是自相矛盾的,因为我们立即得到 \(\operatorname{card}U < \operatorname{card}P\left( U\right)\) ,因此集合 \(U\) 并不包含它声称要包含的所有内容。这类问题最终通过施加一些限制条件来解决,即规定什么才能算作一个集合。随着集合论的形式化,公理必须精心设计,以确保像 \(U\) 这样的对象根本不被允许。一个更实际且需要关注的问题是证明我们对基数之间“ \(\leq\) ”的定义确实是一个序关系。这涉及到证明基数具有类似于实数的性质,即如果 \(\operatorname{card}X \leq  \operatorname{card}Y\) 且 \(\operatorname{card}Y \leq  \operatorname{card}X\) ,则 \(\operatorname{card}X = \operatorname{card}Y\) 。最终,这归结为证明如果存在 \(f : X \rightarrow  Y\) 是单射,且存在 \(g : Y \rightarrow  X\) 是单射,那么可以找到一个既是单射又是满射的函数 \(h : X \rightarrow  Y\) 。Cantor未能证明这一事实,但最终由 Ernst Schröder (1896年)和 Felix Bernstein (1898年)独立提供。Schröder-Bernstein 定理的论证在练习1.4.13中概述。

Cantor 所关注的、源自新兴基数理论的另一个深层次问题,在他有生之年并未得到解决。由于可数集的重要性,符号 \({\aleph }_{0}\) 常被用来表示基数 \(\mathbb{N}\) 。当我们记住可数集是最小的无限集类型时,下标“0”是合适的。就基数而言,如果基数 \(X < {\aleph }_{0}\) ,那么 \(X\) 是有限的。因此, \({\aleph }_{0}\) 是最小的无限基数。 \(\mathbb{R}\) 的基数也足够重要,值得特别命名为 \(\aleph_1 = \operatorname{card}\mathbb{R} = \operatorname{card}\left( {0,1}\right)\) 。定理~\ref{thm:1.4.11} 和 \ref{thm:1.5.1}的内容是 \({\aleph }_{0} < \aleph_1\) 。困扰Cantor的问题是,是否存在严格介于这两者之间的基数。换句话说,是否存在一个集合 \(A \subseteq  \mathbb{R}\) ,使得 \(\operatorname{card}\mathbb{N} < \operatorname{card}A < \operatorname{card}\mathbb{R}\) ?Cantor认为这样的集合不存在。在基数的排序中,他推测 \(\aleph_1\) 是 \({\aleph }_{0}\) 的直接后继。

Cantor 的“连续统假设”是过去一个世纪中最著名的数学难题之一。它的意外解决分为两部分。1940年,德国逻辑学家和数学家 Kurt Gödel 证明,仅使用集合论中公认的公理集,无法否定连续统假设。1963年,Paul Cohen 成功证明,在相同的规则下,也无法证明这一猜想。综合来看,这两项发现意味着连续统假设是不可判定的。它可以被接受或拒绝作为关于无限集性质的陈述,无论哪种情况都不会产生任何逻辑矛盾。

提到 Kurt Gödel,不禁让人对 Cantor 工作的意义作最后的评论。Gödel 以其“不完备性定理”最为著名,这些定理涉及一般公理系统的强度。Gödel 所证明的是,任何为研究算术而创建的一致公理系统必然注定是“不完备的”,即总会存在一些命题,使公理系统因过于薄弱而无法证明。在 Gödel 极其复杂的证明的核心,有一种与定理~\ref{thm:1.5.1}和~\ref{thm:1.5.2}证明中发生的事情密切相关的操作类型。Cantor 证明方法的变体也可以在计算机科学的限制性结果中找到。“停机问题”大致询问是否存在某种通用算法,可以查看每个程序并决定该程序是否最终终止。证明这种算法不存在的核心论证使用了对角线化类型的构造。主要观点是,不仅Cantor定理的推论深远,其论证技术也同样重要。作为这一现象的一个更直接的例子,对角线化方法在第\ref{chap:6}章中再次被使用——以一种建设性的方式——作为 Arzela-Ascoli 定理证明的关键步骤。

%%% Local Variables:
%%% mode: latex
%%% TeX-master: "main"
%%% End:
