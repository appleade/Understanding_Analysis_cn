\chapter{函数极限与连续性}
\label{chap:4}
\section{讨论: Dirichlet 和 Thomae 的例子}
\label{sec:4.1}

尽管在微积分课程中通常先讨论连续性再讨论微分,但历史上数学家对连续性概念的关注远在导数被广泛使用之后。Pierre de Fermat (1601-1665)早在1629年就使用切线来解决优化问题。另一方面,直到大约1820年,Cauchy、Bolzano、Weierstrass 等人才开始用比“无断裂曲线”或“无跳跃或间隙的函数”等流行的直观概念更为严谨的方式来描述连续性。这两百年等待期的根本原因在于,在这段时间的大部分时间里,函数的概念本身并不真正允许不连续性。函数是诸如多项式、正弦和余弦等实体,在其相关域内总是平滑且连续的。函数这一术语逐渐解放到其现代理解——将唯一输出与给定输入相关联的规则——与19世纪对无穷级数行为的研究同时发生。微积分能力的扩展与将函数 \(f\left( x\right)\) 表示为多项式的极限(称为幂级数)或正弦和余弦的和的极限(称为三角级数或Fourier级数)的能力密切相关。Cauchy及其同时代人面临的一个典型问题是,极限多项式或极限三角函数的连续性是否必然意味着极限 \(f\) 也是连续的。

函数列与级数是第\ref{chap:6}章的主题。此刻相关的是,我们意识到为什么“为连续性寻找严格定义”的问题最终走到了前台。关于连续函数的极限是否连续的问题(对Cauchy和我们来说)的任何重大进展,必然依赖于一个不依赖于“无洞”或“间隙”等不精确概念的连续性定义。有了序列极限的数学明确定义,我们正朝着严格理解连续性的方向迈进。

给定一个定义域为 \(A \subseteq  \mathbb{R}\) 的函数 \(f\) ,我们希望在点 \(c \in  A\) 处定义连续性。这意味着如果 \(x \in  A\) 选择在 \(c\) 附近,那么 \(f\left( x\right)\) 将在 \(f\left( c\right)\) 附近。用符号语言,称 \(f\) 在 \(c\) 处连续,若

\[
\mathop{\lim }\limits_{{x \rightarrow  c}}f\left( x\right)  = f\left( c\right) .
\]

问题在于,目前我们只有数列极限的定义,而 \(\mathop{\lim }\limits_{{x \rightarrow  c}}f\left( x\right)\) 的含义并不完全清楚。当我们尝试构建这样一个定义时,所出现的微妙之处通过一系列例子得到了很好的说明,这些例子都基于德国著名数学家 Peter Lejeune Dirichlet 的想法。Dirichlet的想法是根据输入变量 \(x\) 是有理数还是无理数,以分段方式定义一个函数 \(g\) 。具体来说,设

\[
g\left( x\right)  = \left\{  \begin{array}{ll} 1 & x \in  \mathbb{Q} \\  0 & x \notin  \mathbb{Q}. \end{array}\right.
\]

\(\mathbb{Q}\) 和 \(\mathbb{R}\setminus \mathbb{Q}\) 在 \(\mathbb{R}\) 内部错综复杂的结合方式使得 \(g\) 的精确图形在技术上无法绘制,但图\ref{fig:4.1}展示了其大致想法。


\begin{figure}[h]
  \centering
  \includegraphics[width=0.4\textwidth]{images/01955a91-1881-740a-ab38-d94129e5d318_1_675_381_611_284_0.jpg}
  \caption{Dirichlet函数 \(g\left( x\right)\) }
  \label{fig:4.1}
\end{figure}


为表达式 \(\mathop{\lim }\limits_{{x \rightarrow  1/2}}g\left( x\right)\) 赋予一个值是否有意义?一种思路是考虑一个序列 \(\left( {x}_{n}\right)  \rightarrow  1/2\) 。利用我们对序列极限的概念,我们可能会尝试将 \(\mathop{\lim }\limits_{{x \rightarrow  1/2}}g\left( x\right)\) 简单地定义为序列 \(g\left( {x}_{n}\right)\) 的极限。但需要注意的是,这个极限取决于序列 \(\left( {x}_{n}\right)\) 的选择方式。如果每个 \({x}_{n}\) 都是有理数,那么

\[
\mathop{\lim }\limits_{{n \rightarrow  \infty }}g\left( {x}_{n}\right)  = 1
\]

另一方面,如果每个 \({x}_{n}\) 都是无理数,那么

\[
\mathop{\lim }\limits_{{n \rightarrow  \infty }}g\left( {x}_{n}\right)  = 0.
\]



这种不可接受的情况要求我们在函数极限的定义上更加努力。一般来说,我们希望 \(\mathop{\lim }\limits_{{x \rightarrow  c}}g\left( x\right)\) 的值与我们如何接近 \(c\) 无关。在这种特定情况下,我们一致认同的函数极限定义应得出以下结论:

\[
\mathop{\lim }\limits_{{x \rightarrow  1/2}}g\left( x\right) \;\text{不存在}
\]

我们暂时放下寻找正式定义的努力。至少现在我们知道了Dirichlet函数在 \(c = 1/2\) 处不连续。事实上,这个函数在 \(c = 1/2\) 点并没有什么独特之处。因为 \(\mathbb{Q}\) 和 \(\mathbb{R}\setminus\mathbb{Q}\) (无理数集)都在实数轴上稠密,故 \(\forall z \in  \mathbb{R}\) ,我们都能找到序列 \(\left( {x}_{n}\right)  \subseteq  \mathbb{Q}\) 和 \(\left( {y}_{n}\right)  \subseteq  \mathbb{R}\setminus\mathbb{Q}\) ,使得

\[
\lim {x}_{n} = \lim {y}_{n} = z.
\]

(参见例\ref{eg:3.2.9}\ref{item:3.2.7}。)因为

\[
\lim g\left( {x}_{n}\right)  \neq  \lim g\left( {y}_{n}\right) ,
\]

同样的推理表明 \(g\left( x\right)\) 在 \(z\) 处不连续。用分析的术语来说,Dirichlet函数是 \(\mathbb{R}\) 上的无处连续函数。


\begin{figure}[h]
  \centering
  \includegraphics[width=0.4\textwidth]{images/01955a91-1881-740a-ab38-d94129e5d318_2_512_401_614_326_0.jpg}
  \caption{修正的Dirichlet函数, \(h\left( x\right)\) }
  \label{fig:4.2}
\end{figure}

如果我们以下列方式调整 \(g\left( x\right)\) 的定义会发生什么?通过在 \(\mathbb{R}\) 上设置一个新函数 \(h\) (图\ref{fig:4.2})来定义

\[
h\left( x\right)  = \left\{  \begin{array}{ll} x & x \in  \mathbb{Q} \\  0 & x \notin  \mathbb{Q}. \end{array}\right.
\]

如果我们取 \(c\ne 0\),那么就像之前一样,我们可以构造有理数序列 \(\left( {x}_{n}\right)  \rightarrow  c\) 和无理数序列 \(\left( {y}_{n}\right)  \rightarrow  c\) ,使得

\[
\lim h\left( {x}_{n}\right)  = c, \quad \lim h\left( {y}_{n}\right)  = 0.
\]

因此, \(h\) 在每一点 \(c \neq  0\) 处都不连续。

如果 \(c = 0\) ,那么这两个极限都等于 \(h\left( 0\right)  = 0\) 。事实上,无论我们如何构造一个收敛于零的序列 \(\left( {z}_{n}\right)\) ,似乎总是有 \(\lim h\left( {z}_{n}\right)  = 0\) 。这一观察触及了我们所希望的函数极限概念所蕴含的核心思想。我们断言

\[
\mathop{\lim }\limits_{{x \rightarrow  c}}h\left( x\right)  = L
\]

应意味着

\[
h\left( {z}_{n}\right)  \rightarrow  L \quad \forall z_n \text{ s.t. } \left( {z}_{n}\right)  \rightarrow  c.
\]

出于尚不明显的原因,用围绕 \(c\) 和 \(L\) 构造的邻域来定义函数极限是有益的。然而,我们将很快看到,这种拓扑公式化与我们在此得出的序列特征是等价的。



\begin{figure}[h]
  \centering
  \includegraphics[width=0.4\textwidth]{images/01955a91-1881-740a-ab38-d94129e5d318_3_710_398_563_250_0.jpg}
  \caption{Thomae函数, \(t\left( x\right)\) }
  \label{fig:4.3}
\end{figure}


到目前为止,我们一直在讨论函数在其定义域中某一点的连续性。这与将连续函数视为无需抬笔即可绘制的曲线有显著不同,并引发了一些引人入胜的问题。1875年,K.J. Thomae 发现了函数

\[
t\left( x\right)  = \left\{  \begin{array}{ll} 1 & x = 0 \\  1/n & x = m/n \in  \mathbb{Q} \smallsetminus  \{ 0\} , n > 0, \gcd(m,n) = 1 \\  0 & x \notin  \mathbb{Q}. \end{array}\right.
\]

如果 \(c \in  \mathbb{Q}\) ,那么 \(t\left( c\right)  > 0\) 。因为无理数集在 \(\mathbb{R}\) 中稠密,我们可以在 \(\mathbb{R}\setminus\mathbb{Q}\) 中找到一个序列 \(\left( {y}_{n}\right)\) 收敛到 \(c\) 。于是

\[
\lim t\left( {y}_{n}\right)  = 0 \neq  t\left( c\right) ,
\]

故Thomae函数(图\ref{fig:4.3})在任何有理点处都不连续。

当我们尝试在定义域中的某些无理点(如 \(c = \sqrt{2}\) )上使用这个论证时,情况就变得复杂了。所有无理值都被 \(t\) 映为零,因此自然要考虑一个收敛到 \(\sqrt{2}\)  的有理数序列 \(\left( {x}_{n}\right)\) 。现在, \(\sqrt{2} \approx  {1.414213}\ldots\) 所以对于 \(\sqrt{2}\) 的一个特定有理逼近序列,一个好的起点可能是

\[
\left( {1,\frac{14}{10},\frac{141}{100},\frac{1414}{1000},\frac{14142}{10000},\frac{141421}{100000},\ldots }\right) .
\]

但请注意,这些分数的分母越来越大。在这种情况下,序列 \(t\left( {x}_{n}\right)\) 开始,

\[
\left( {1,\frac{1}{5},\frac{1}{100},\frac{1}{500},\frac{1}{5000},\frac{1}{100000},\ldots }\right)
\]

并且迅速接近 \(0 = t\left( \sqrt{2}\right)\) 。我们将看到这种情况总是发生。有理数越接近固定的无理数,其分母必然越大。因此,Thomae函数具有在 \(\mathbb{R}\) 上的每个无理点连续而在每个有理点不连续的奇特性质。

是否存在一个具有相反性质的函数示例?换句话说,是否存在一个定义在整个 \(\mathbb{R}\) 上的函数,在 \(\mathbb{Q}\) 上连续但在$\mathbb{R}\setminus \mathbb{Q}$上不连续?特定函数的不连续点集可以是任意的吗?如果我们给定某个集合 \(A \subseteq  \mathbb{R}\) ,是否总能找到一个仅在集合 \({A}^{c}\) 上连续的函数?在本节的两个示例中,我们所研究的函数在定义域中的点周围具有不规则的振荡。如果我们把注意力限制在波动较小的函数上,我们能得出什么结论?其中一类是所谓的单调函数,它们在给定定义域上要么递增要么递减。关于 \(\mathbb{R}\) 上单调函数的不连续点集,我们能说些什么?

\section{函数极限}
\label{sec:4.2}
考虑一个函数 \(f : A \rightarrow  \mathbb{R}\) 。回想一下, \(A\) 的极限点 \(c\) 是指具有以下性质的点:每个 \(\varepsilon\) -邻域 \({V}_{\varepsilon }\left( c\right)\) 都与 \(A\) 交于 \(c\) 以外的某个点。等价地, 称 \(c\) 是 \(A\) 的极限点,当且仅当存在序列 \(\left( {x}_{n}\right)  \subseteq  A\) 且 \({x}_{n} \neq  c\) ,使得 \(c = \lim {x}_{n}\) 成立。重要的是要记住, \(A\) 的极限点不一定属于集合 \(A\) ,除非 \(A\) 是闭集。

如果 \(c\) 是 \(f\) 定义域内的极限点,那么直观上,命题

\[
\mathop{\lim }\limits_{{x \rightarrow  c}}f\left( x\right)  = L
\]

旨在传达当 \(x\) 被选择得越来越接近 \(c\) 时, \(f\left( x\right)\) 的值会任意接近 \(L\) 。从函数极限的角度来看, \(x = c\) 处发生什么是无关紧要的。事实上, \(c\) 甚至不必在 \(f\) 的定义域内。

函数极限定义的结构遵循了序列极限定义中建立的“挑战-响应”模式。回想一下,给定一个序列 \(\left( {a}_{n}\right)\) ,断言 \(\lim {a}_{n} = L\) 意味着对于每一个以 \(L\) 为中心的 \(\varepsilon\) -邻域 \({V}_{\varepsilon }\left( L\right)\) ,序列中存在一个点——称之为 \({a}_{N} \) ,在此之后所有的项 \({a}_{n}\) 都落在 \({V}_{\varepsilon }\left( L\right)\) 中。每一个 \(\varepsilon\) -邻域代表一个特定的挑战,而每一个 \(N\) 则是相应的响应。对于诸如 \(\mathop{\lim }\limits_{{x \rightarrow  c}}f\left( x\right)  = L\) 这样的函数极限陈述,挑战仍然以围绕 \(L\) 的任意 \(\varepsilon\) -邻域的形式提出,但这次作为响应的是以 \(c\) 为中心的 \(\delta\) -邻域。


\begin{figure}[h]
  \centering
  \includegraphics[width=0.4\textwidth]{images/01955a91-1881-740a-ab38-d94129e5d318_5_692_381_513_431_0.jpg}
  \caption{函数极限的定义}
  \label{fig:4.4}
\end{figure}



\begin{Def}
  \label{def:4.2.1}
  设 \(f : A \rightarrow  \mathbb{R}\) ,并设 \(c\) 为定义域 \(A\) 的一个极限点。我们说 \(\mathop{\lim }\limits_{{x \rightarrow  c}}f\left( x\right)  = L\) 成立,当且仅当对于所有 \(\varepsilon  > 0\) ,存在一个 \(\delta  > 0\) ,使得每当 \(0 < \left| {x - c}\right|  < \delta\) (且 \(x \in  A\) )时,就有 \(\left| {f\left( x\right)  - L}\right|  < \varepsilon\) 。

\end{Def}

这通常被称为函数极限定义的“ \(\varepsilon  - \delta\) 语言”。回想一下,以下两个命题是等价的

\[
\left| {f\left( x\right)  - L}\right|  < \varepsilon  \Leftrightarrow f\left( x\right)  \in  {V}_{\varepsilon }\left( L\right) .
\]

同样地,以下两个命题也是等价的

\[
\left| {x - c}\right|  < \delta  \Leftrightarrow x \in  {V}_{\delta }\left( c\right) \text{ . }
\]

附加限制 \(0 < \left| {x - c}\right|\) 只是 \(x \neq  c\) 的另一种表达方式。将定义\ref{def:4.2.1}重新表述为邻域的形式——正如我们在第\ref{sec:2.2}节中对序列收敛定义所做的那样——几乎只是符号的变化,但它确实有助于强调所发生现象的几何性质(图\ref{fig:4.4})。

\addtocounter{Thm}{-1}


\begin{Def}[拓扑版本]
设 \(c\) 为 \(f : A \rightarrow  \mathbb{R}\) 定义域的一个极限点。称 \(\mathop{\lim }\limits_{{x \rightarrow  c}}f\left( x\right)  = L\) 成立,若对于 \(L\) 的每一个 \(\varepsilon\) -邻域 \({V}_{\varepsilon }\left( L\right)\) ,都存在 \(c\) 的一个 \(\delta\) -邻域 \({V}_{\delta }\left( c\right)\) ,使得\(\forall x \in  {V}_{\delta }\left( c\right)\) (且 \(x \in  A\) ),都有 \(f\left( x\right)  \in  {V}_{\varepsilon }\left( L\right)\) 成立。  
\end{Def}


定义的两个版本中都包含括号内的提醒“ \(\left( {x \in  A}\right)\) ”,以确保 \(x\) 是所讨论函数的合法输入。当不太可能引起混淆时,我们可以省略此提醒,将“ \(f\left( x\right)\) 的出现意味着 \(x\) 在 \(f\) 的定义域中”的隐含假设留给读者。同样地,没有必要在定义域的孤立点上讨论函数极限。因此,我们只讨论 \(x\) 趋近于函数定义域的极限点时的函数极限。



\begin{Eg}
  \label{eg:4.2.2}
  \begin{enumerate}[label = (\roman*)]
  \item\label{item:4.2.1}为了熟悉定义\ref{def:4.2.1},让我们证明如果 \(f\left( x\right)  = {3x} + 1\) ,那么

\[
\mathop{\lim }\limits_{{x \rightarrow  2}}f\left( x\right)  = 7
\]

设 \(\varepsilon  > 0\) 。定义\ref{def:4.2.1} 要求我们取一个 \(\delta  > 0\) ,使得只要 \(0 < \left| {x - 2}\right|  < \delta\) 便有 \(\left| {f\left( x\right)  - 7}\right|  < \varepsilon\) 。注意到

\[
\left| {f\left( x\right)  - 7}\right|  = \left| {\left( {{3x} + 1}\right)  - 7}\right|  = \left| {{3x} - 6}\right|  = 3\left| {x - 2}\right| .
\]

因此,如果我们选择 \(\delta  = \varepsilon /3\) ,那么 \(0 < \left| {x - 2}\right|  < \delta\) 意味着 \(\left| {f\left( x\right)  - 7}\right|  < 3\left( {\varepsilon /3}\right)  = \varepsilon\) 。
  \item\label{item:4.2.2}让我们证明

\[
\mathop{\lim }\limits_{{x \rightarrow  2}}g\left( x\right)  = 4
\]

其中 \(g\left( x\right)  = {x}^{2}\) 。给定任意 \(\varepsilon  > 0\) ,我们这次的目标是通过限制 \(\left| {x - 2}\right|\) 小于某个精心选择的 \(\delta\) 来使 \(\left| {g\left( x\right)  - 4}\right|  < \varepsilon\) 成立。与之前的问题一样,代数运算表明:

\[
\left| {g\left( x\right)  - 4}\right|  = \left| {{x}^{2} - 4}\right|  = \left| {x + 2}\right| \left| {x - 2}\right| .
\]

我们可以使 \(\left| {x - 2}\right|\) 尽可能小,但我们需要 \(\left| {x + 2}\right|\) 的上界,以便知道选择 \(\delta\) 的大小。变量 \(x\) 的存在最初会引起一些困惑,但请记住,我们讨论的是 \(x\) 趋近于2时的极限。如果我们同意 \(c = 2\) 周围的 \(\delta\) 邻域的半径不能大于 \(\delta  = 1\) ,那么我们就得到了对所有 \(x \in  {V}_{\delta }\left( c\right)\) 都成立的上界 \(\left| {x + 2}\right|  \leq  \left| {3 + 2}\right|  = 5\) 。

现在,选择 \(\delta  = \min \{ 1,\varepsilon /5\}\) 。如果 \(0 < \left| {x - 2}\right|  < \delta\) ,那么可以得出

\[
\left| {{x}^{2} - 4}\right|  = \left| {x + 2}\right| \left| {x - 2}\right|  < \left( 5\right) \frac{\varepsilon }{5} = \varepsilon ,
\]

极限得证。
  \end{enumerate}
\end{Eg}




\subsection{函数极限的序列准则}

我们在第二章中非常努力地推导出了一系列关于序列极限的重要性质。特别是,代数极限定理(定理\ref{thm:2.3.3})和序极限定理(定理\ref{thm:2.3.4})在后续的许多论证中证明是非常重要的。不出所料,我们将需要类似的陈述来处理函数极限。尽管为这些陈述生成独立的证明并不困难,但一旦我们从本章开头的讨论中推导出函数极限的序列准则,所有这些陈述都将自然而然地从其序列类比中得出。

\begin{Thm}[函数极限的序列准则]
  \label{thm:4.2.3}
  给定一个函数 \(f : A \rightarrow  \mathbb{R}\) 和 \(A\) 的一个极限点 \(c\) ,以下两个陈述是等价的:
  \begin{enumerate}[label = (\roman*)]
  \item\label{item:4.2.3} \(\mathop{\lim }\limits_{{x \rightarrow  c}}f\left( x\right)  = L\) .
  \item\label{item:4.2.4}  \(\forall{x}_{n} \neq  c, \left( {x}_{n}\right)  \rightarrow  c, f\left( {x}_{n}\right)  \rightarrow  L\) 。
  \end{enumerate}
\end{Thm}

\begin{proof}
\( \Rightarrow :\) 我们首先假设 \(\mathop{\lim }\limits_{{x \rightarrow  c}}f\left( x\right)  = L\) 。为了证明~\ref{item:4.2.3},我们考虑一个任意的序列 \(\left( {x}_{n}\right)\) ,使其收敛于 \(c\) 并满足 \({x}_{n} \neq  c\) 。我们的目标是证明序列 \(f\left( {x}_{n}\right)\) 收敛于 \(L\) 。这最容易通过拓扑版的定义来理解。

设 \(\varepsilon  > 0\) 。因为我们假设\ref{item:4.2.3},定义\ref{def:4.2.1}意味着存在一个 \({V}_{\delta }\left( c\right)\) ,其性质是所有不同于 \(c\) 的 \(x \in  {V}_{\delta }\left( c\right)\) 都满足 \(f\left( x\right)  \in  {V}_{\varepsilon }\left( L\right)\) 。我们只需要论证我们的特定序列 \(\left( {x}_{n}\right)\) 最终在 \({V}_{\delta }\left( c\right)\) 中。但我们假设 \(\left( {x}_{n}\right)  \rightarrow  c\) 。这意味着存在一个点 \({x}_{N}\) ,在此之后 \({x}_{n} \in  {V}_{\delta }\left( c\right)\) 。因此, \(n \geq  N\) 意味着 \(f\left( {x}_{n}\right)  \in  {V}_{\varepsilon }\left( L\right)\) ,得证。

$\Leftarrow:$ 施反证法,设命题~\ref{item:4.2.4}为真,并(谨慎地)否定命题~\ref{item:4.2.3}。也就是说

\[
\mathop{\lim }\limits_{{x \rightarrow  c}}f\left( x\right)  \neq  L
\]

这意味着至少存在一个特定的 \({\varepsilon }_{0} > 0\) ,对于它来说,没有任何 \(\delta\) 是合适的响应。换言之, \(\forall \delta  > 0\) ,总会存在至少一个点

\[
x \in  {V}_{\delta }\left( c\right), x \neq  c, f\left( x\right)  \notin  {V}_{{\varepsilon }_{0}}\left( L\right) .
\]

\(\forall n \in  \mathbb{N}\) ,令 \({\delta }_{n} = 1/n\) 。从前面的讨论可以得出,\(\forall n \in  \mathbb{N}\) ,我们可以选择一个 \({x}_{n} \in  {V}_{{\delta }_{n}}\left( c\right)\) ,使得 \({x}_{n} \neq  c\) 且 \(f\left( {x}_{n}\right)  \notin  {V}_{{\varepsilon }_{0}}\left( L\right)\) 。但现在注意到,这样做的结果是产生了一个序列 \(\left( {x}_{n}\right)  \rightarrow  c\) ,其中 \({x}_{n} \neq  c\) ,而序列 \(f\left( {x}_{n}\right)\) 显然不收敛于 \(L\) 。

因为这与我们假设为真的\ref{item:4.2.4}相矛盾,所以我们可以得出结论,\ref{item:4.2.3}也必然成立。  
\end{proof}

定理~\ref{thm:4.2.3}有几个有用的推论。除了之前提到的好处(为我们提供了一些关于函数极限如何与函数的代数组合相互作用的简短证明)外,我们还获得了一种经济的方法来确定某些极限不存在。

\begin{Cor}[函数极限的代数极限定理]
  \label{cor:4.2.4}
  设 \(f\) 和 \(g\) 为定义在 \(A \subseteq  \mathbb{R}\) 上的函数,并假设对于 \(A\) 的某个极限点 \(c\) ,有 \(\mathop{\lim }\limits_{{x \rightarrow  c}}f\left( x\right)  = L\) 和 \(\mathop{\lim }\limits_{{x \rightarrow  c}}g\left( x\right)  = M\) 。则
\begin{enumerate}
\item  \(\forall k \in  \mathbb{R}\) , \(\mathop{\lim }\limits_{{x \rightarrow  c}}{kf}\left( x\right)  = {kL}\) ,
\item  \(\mathop{\lim }\limits_{{x \rightarrow  c}}\left\lbrack  {f\left( x\right)  + g\left( x\right) }\right\rbrack   = L + M\) ,
\item  \(\mathop{\lim }\limits_{{x \rightarrow  c}}\left\lbrack  {f\left( x\right) g\left( x\right) }\right\rbrack   = {LM}\) ,
\item $M\ne 0$时,\(\mathop{\lim }\limits_{{x \rightarrow  c}}f\left( x\right) /g\left( x\right)  = L/M\) 。
\end{enumerate}
\end{Cor}

\begin{proof}
  详见习题4.2.5。
\end{proof}


\begin{Cor}
  \label{cor:4.2.5}
  设 \(f\) 是定义在 \(A\) 上的函数,且 \(c\) 是 \(A\) 的极限点。如果存在 \(A\) 中的两个序列 \(\left( {x}_{n}\right)\) 和 \(\left( {y}_{n}\right)\) ,满足 \({x}_{n} \neq  c\) 和 \({y}_{n} \neq  c\) ,且
\[
\lim {x}_{n} = \lim {y}_{n} = c, \quad\lim f\left( {x}_{n}\right)  \neq  \lim f\left( {y}_{n}\right) ,
\]
则函数极限 \(\mathop{\lim }\limits_{{x \rightarrow  c}}f\left( c\right)\) 不存在。
\end{Cor}


\begin{Eg}
  \label{eg:4.2.6}
  假设我们已熟知正弦函数的性质,让我们证明 \(\mathop{\lim }\limits_{{x \rightarrow  0}}\sin \left( {1/x}\right)\) 不存在(图\ref{fig:4.5})。

\begin{figure}[h]
  \centering
  \includegraphics[width=0.6\textwidth]{images/01955a91-1881-740a-ab38-d94129e5d318_8_420_1443_799_362_0.jpg}
  \caption{函数 \(\sin \left( {1/x}\right)\) 在零附近的情况}\label{fig:4.5}
\end{figure}

设 \({x}_{n} = 1/{2n\pi }\) 且 \({y}_{n} = 1/\left( {{2n\pi } + \pi /2}\right)\) ,则 \(\lim \left( {x}_{n}\right)  = \lim \left( {y}_{n}\right)  = 0\) 。然而 $\forall n\in \mathbb{N}$, \(\sin \left( {1/{x}_{n}}\right)  = 0\) , \(\sin \left( {1/{y}_{n}}\right)  = 1\) 。因此,

\[
\lim \sin \left( {1/{x}_{n}}\right)  \neq  \lim \sin \left( {1/{y}_{n}}\right) ,
\]

根据推论 \ref{cor:4.2.5}, \(\mathop{\lim }\limits_{{x \rightarrow  0}}\sin \left( {1/x}\right)\) 不存在。
\end{Eg}



\subsection{练习}

练习 4.2.1。使用定义 4.2.1 为以下极限语句提供证明。

(a) \(\mathop{\lim }\limits_{{x \rightarrow  2}}\left( {{2x} + 4}\right)  = 8\) .

(b) \(\mathop{\lim }\limits_{{x \rightarrow  0}}{x}^{3} = 0\) .

(c) \(\mathop{\lim }\limits_{{x \rightarrow  2}}{x}^{3} = 8\) .

(d) \(\mathop{\lim }\limits_{{x \rightarrow  \pi }}\left\lbrack  \left\lbrack  x\right\rbrack  \right\rbrack   = 3\) ,其中 \(\left\lbrack  \left\lbrack  x\right\rbrack  \right\rbrack\) 表示小于或等于 \(x\) 的最大整数。

练习 4.2.2. 假设某个 \(\delta  > 0\) 已被构造为对特定 \(\varepsilon\) 挑战的适当响应。那么,任何更大/更小(选择一个)的 \(\delta\) 也将足够。

练习 4.2.3. 使用推论 4.2.5 证明以下每个极限不存在。

(a) \(\mathop{\lim }\limits_{{x \rightarrow  0}}\left| x\right| /x\)

(b) \(\mathop{\lim }\limits_{{x \rightarrow  1}}g\left( x\right)\) ,其中 \(g\) 是来自第4.1节的Dirichlet函数。

练习4.2.4。回顾第4.1节中Thomae函数(Thomae’s function) \(t\left( x\right)\) 的定义。

(a) 构造三个不同的序列 \(\left( {x}_{n}\right) ,\left( {y}_{n}\right)\) 、 \(\left( {z}_{n}\right)\) ,每个序列都收敛到1,但不使用数字1作为序列中的项。

(b) 现在,计算 \(\lim t\left( {x}_{n}\right) ,\lim t\left( {y}_{n}\right)\) 和 \(\lim t\left( {z}_{n}\right)\) 。

(c) 对 \(\mathop{\lim }\limits_{{x \rightarrow  1}}t\left( x\right)\) 做出一个有根据的猜想,并使用定义 4.2.1B 来验证该主张。(给定 \(\varepsilon  > 0\) ,考虑点集 \(\{ x \in  \mathbb{R} : t\left( x\right)  \geq  \varepsilon \}\) 。论证该集合中的所有点都是孤立的。)

练习 4.2.5. (a) 详细说明推论 4.2.4 第 (ii) 部分如何从定理 4.2.3 中函数极限的序列准则和第 2 章中证明的序列代数极限定理得出。

(b) 现在,直接从定义 4.2.1 出发,不使用定理 4.2.3 中的序列准则,写出推论 4.2.4 第 (ii) 部分的另一个证明。

(c) 对推论 4.2.4 第 (iii) 部分重复 (a) 和 (b)。

练习 4.2.6。设 \(g : A \rightarrow  \mathbb{R}\) 并假设 \(f\) 是 \(A \subseteq  \mathbb{R}\) 上的有界函数(即存在 \(M > 0\) 满足对于所有 \(x \in  A\) 有 \(\left| {f\left( x\right) }\right|  \leq  M\) )。证明如果 \(\mathop{\lim }\limits_{{x \rightarrow  c}}g\left( x\right)  = 0\) ,则 \(\mathop{\lim }\limits_{{x \rightarrow  c}}g\left( x\right) f\left( x\right)  = 0\) 也成立。

练习 4.2.7. (a) 陈述 \(\mathop{\lim }\limits_{{x \rightarrow  0}}1/{x}^{2} = \infty\) 在直觉上显然是有意义的。为形式为 \(\mathop{\lim }\limits_{{x \rightarrow  c}}f\left( x\right)  = \infty\) 的极限陈述构建一个类似于定义 4.2.1 的“挑战-回应”风格的严格定义,并用它来证明之前的陈述。

(b) 现在,为陈述 \(\mathop{\lim }\limits_{{x \rightarrow  \infty }}f\left( x\right)  = L\) 构建一个定义。展示 \(\mathop{\lim }\limits_{{x \rightarrow  \infty }}1/x = 0\) 。

(c) \(\mathop{\lim }\limits_{{x \rightarrow  \infty }}f\left( x\right)  = \infty\) 的严格定义会是什么样?给出一个这样的极限的例子。

练习 4.2.8. 假设对于某个集合 \(A\) 中的所有 \(x\) , \(f\left( x\right)  \geq  g\left( x\right)\) 成立,其中 \(f\) 和 \(g\) 在该集合上定义。证明对于 \(A\) 的任何极限点 \(c\) ,我们必须有

\[
\mathop{\lim }\limits_{{x \rightarrow  c}}f\left( x\right)  \geq  \mathop{\lim }\limits_{{x \rightarrow  c}}g\left( x\right) .
\]

习题 4.2.9(夹逼定理)。设 \(f,g\) 、 \(h\) 满足 \(f\left( x\right)  \leq  g\left( x\right)  \leq\)  \(h\left( x\right)\) 对于所有在某个共同定义域 \(A\) 中的 \(x\) 。如果 \(\mathop{\lim }\limits_{{x \rightarrow  c}}f\left( x\right)  = L\) 且 \(\mathop{\lim }\limits_{{x \rightarrow  c}}h\left( x\right)  =\)  \(L\) 在 \(A\) 的某个极限点 \(c\) 处,证明 \(\mathop{\lim }\limits_{{x \rightarrow  c}}g\left( x\right)  = L\) 也成立。

\section{连续函数的运算}
\label{sec:4.3}
\begin{Def}
  \label{def:4.3.1}
称一个函数 \(f : A \rightarrow  \mathbb{R}\) 在点 \(c \in  A\) 处连续,若 \(\forall \varepsilon  > 0\) , \(\exists \delta  > 0\) ,使得每当 \(\left| {x - c}\right|  < \delta\) (且 \(x \in  A\) )时,就有 \(\left| {f\left( x\right)  - f\left( c\right) }\right|  < \varepsilon\) 。

如果 \(f\) 在定义域 \(A\) 中的每一点都连续,那么我们说 \(f\) 在 \(A\) 上连续。
\end{Def}


连续性的定义看起来与函数极限的定义非常相似,但有一些细微的差别。最重要的是,我们要求点 \(c\) 在 \(f\) 的定义域内。然后,值 \(f\left( c\right)\) 变成了 \(\mathop{\lim }\limits_{{x \rightarrow  c}}f\left( x\right)\) 的值。考虑到这一点,人们很容易将定义~\ref{def:4.3.1}简化为:称 \(f\) 在 \(c \in  A\) 处连续,若

\[
\mathop{\lim }\limits_{{x \rightarrow  c}}f\left( x\right)  = f\left( c\right) .
\]

这很好——除了一个非常小的技术细节以外:函数极限的定义要求点 \(c\) 是 \(A\) 的极限点,而这一点在定义\ref{def:4.3.1}中并没有技术上的假设。同时,定义~\ref{def:4.3.1}的一个结果是,任何函数在其定义域的孤立点处都是连续的(练习4.3.4),但这并不深刻。正如其名称所示,孤立点离定义域的其他点太远,无法对任何有趣的现象做出贡献。

我们将注意力牢牢集中在定义域的极限点上,总结出连续性的几种等价表征。


\begin{Thm}[连续性的表征]
  \label{thm:4.3.2}
  设 \(f : A \rightarrow  \mathbb{R}\) ,且 \(c \in  A\) 是 \(A\) 的极限点。函数 \(f\) 在 \(c\) 处连续,当且仅当满足以下任一条件:

  \begin{enumerate}[label = (\roman*)]
  \item\label{item:4.3.1}  \(\forall \varepsilon  > 0\) ,\(\exists \delta  > 0\) ,使得只要 \(\left| {x - c}\right|  < \delta\) (且 \(x \in  A\) )就有 \(\left| {f\left( x\right)  - f\left( c\right) }\right|  < \varepsilon\);
  \item\label{item:4.3.2}\(\mathop{\lim }\limits_{{x \rightarrow  c}}f\left( x\right)  = f\left( c\right)\) ;
  \item \label{item:4.3.3} \(\forall{V}_{\varepsilon }\left( {f\left( c\right) }\right)\) , \(\exists {V}_{\delta }\left( c\right)\) ,使得只要 \(x \in  {V}_{\delta }\left( c\right)\) (且 \(x \in  A)\) 便有 \(f\left( x\right)  \in  {V}_{\varepsilon }\left( {f\left( c\right) }\right)\) ;
  \item \label{item:4.3.4}若 \(\left( {x}_{n}\right)  \rightarrow  c\) (且 \({x}_{n} \in  A\) ),则 \(f\left( {x}_{n}\right)  \rightarrow  f\left( c\right)\) 。
  \end{enumerate}
\end{Thm}

\begin{proof}
  命题~\ref{item:4.3.1}即为定义~\ref{def:4.3.1}。通过应用定义~\ref{def:4.2.1}并观察到 \(x = c\) 的情况(这在函数极限的定义中被排除)导致 \(f\left( c\right)  \in  {V}_{\varepsilon }\left( {f\left( c\right) }\right)\) 恒成立,因此陈述~\ref{item:4.3.2}与~\ref{item:4.3.1}等价。陈述~\ref{item:4.3.3}是使用拓扑邻域代替绝对值符号对~\ref{item:4.3.1}的标准重述。最后,陈述~\ref{item:4.3.4}的证明几乎与定理~\ref{thm:4.2.3}的论证相同,只需要对 \({x}_{n} = c\) 的情况稍作修改即可得到。
\end{proof}

这个列表的长度有些误导人。命题~\ref{item:4.3.1}、~\ref{item:4.3.2}和~\ref{item:4.3.3}密切相关,本质上提醒我们函数极限既有 \(\varepsilon  - \delta\) 的表述,也有拓扑表述。然而,陈述~\ref{item:4.3.4}与前三个在性质上有所不同。一般来说,连续性的序列表征通常对于证明函数在某点不连续最为有用。


\begin{Cor}[不连续性的判定准则]
  \label{cor:4.3.3}
  设 \(f : A \rightarrow  \mathbb{R}\) ,且 \(c \in  A\) 是 \(A\) 的极限点。若存在一个序列 \(\left( {x}_{n}\right)  \subseteq  A\) ,使得 \(\left( {x}_{n}\right)  \rightarrow  c\) ,但 \(f\left( {x}_{n}\right)\) 不收敛于 \(f\left( c\right)\) ,则 \(f\) 在 \(c\) 处不连续。
\end{Cor}


连续性的序列表征对于函数极限的重要性同样重要。特别是,它使我们能够将关于序列行为的成果目录应用于连续函数的研究中。下一个定理应与推论~\ref{thm:4.2.3}以及定理\ref{thm:2.3.3}进行比较。


\begin{Thm}[代数连续性定理]
  \label{thm:4.3.4}
  设 \(f : A \rightarrow  \mathbb{R}\) 和 \(g : A \rightarrow  \mathbb{R}\) 在点 \(c \in  A\) 处连续。那么,
\begin{enumerate}[label = (\roman*)]
\item \({kf}\left( x\right)\) 在 \(c\) 处对所有 \(k \in  \mathbb{R}\) 连续;
\item \(f\left( x\right)  + g\left( x\right)\) 在 \(c\) 处连续;
\item \(f\left( x\right) g\left( x\right)\) 在 \(c\) 处连续;并且
\item \(f\left( x\right) /g\left( x\right)\) 在 \(c\) 处连续,前提是该商有定义。
\end{enumerate}

\end{Thm}

\begin{proof}
  所有这些陈述都可以从推论~\ref{cor:4.2.4}和定理~\ref{thm:4.3.2}中快速得出。
\end{proof}

这些结果为我们提供了在本章开头部分关于Dirichlet函数和Thomae函数行为论证所需的工具。细节详见练习4.3.6。下面,我们就一些熟悉的函数讨论其连续性与否。


\begin{Eg}
  \label{eg:4.3.5}
  所有多项式在 \(\mathbb{R}\) 上都是连续的。实际上,有理函数(即多项式的商)在它们有定义的地方都是连续的。

  要理解这一点,我们从基本命题开始,即 \(g\left( x\right)  = x\) 和 \(f\left( x\right)  = k\) 在 \(\mathbb{R}\) 上是连续的(练习4.3.3),其中 \(k \in  \mathbb{R}\) 。因为任意多项式

\[
p\left( x\right)  = {a}_{0} + {a}_{1}x + {a}_{2}{x}^{2} + \cdots  + {a}_{n}{x}^{n}
\]

由 \(g\left( x\right)\) 与不同常数函数的和与积组成,我们可以根据定理~\ref{thm:4.3.4}得出结论, \(p\left( x\right)\) 是连续的。

同样,定理~\ref{thm:4.3.4}表明,只要分母不为零,多项式的商也是连续的。
\end{Eg}



\begin{Eg}
  \label{eg:4.3.6}
在例~\ref{eg:4.2.6}中,我们看到 \(\sin \left( {1/x}\right)\) 的振荡过大,导致 \(\mathop{\lim }\limits_{{x \rightarrow  0}}\sin \left( {1/x}\right)\) 不存在。现在,考虑函数

\[
g\left( x\right)  = \left\{  \begin{array}{ll} x\sin \left( {1/x}\right) & x \neq  0 \\  0 & x = 0. \end{array}\right.
\]


\begin{figure}[h]
  \centering
  \includegraphics[width=0.6\textwidth]{images/01955a91-1881-740a-ab38-d94129e5d318_12_424_393_791_367_0.jpg}
  \caption{函数 \(x\sin \left( {1/x}\right)\) 在零附近的行为}
  \label{fig:4.6}
\end{figure}
为了研究 \(g\) 在 \(c = 0\) 处的连续性(图\ref{fig:4.6}),我们可以进行估计

\[
\left| {g\left( x\right)  - g\left( 0\right) }\right|  = \left| {x\sin \left( {1/x}\right)  - 0}\right|  \leq  \left| x\right| .
\]

给定 \(\varepsilon  > 0\) ,只要取 \(\delta  = \varepsilon\) ,便可使得每当 \(\left| {x - 0}\right|  = \left| x\right|  < \delta\) 时,就有 \(\left| {g\left( x\right)  - g\left( 0\right) }\right|  < \varepsilon\) 。因此, \(g\) 在原点处是连续的。
\end{Eg}



\begin{Eg}
  \label{eg:4.3.7}
(向下)取整函数 \( \lfloor x\rfloor\)  定义在 $\mathbb{R}$ 上:令 \(\lfloor x\rfloor\) 等于满足 \(n \leq  x\) 的最大整数 \(n \in  \mathbb{Z}\) 。这个熟悉的阶梯函数在其定义域的每个整数值处肯定有不连续的“跳跃”,但尝试用分析的语言来阐述这一观察是一个很好的练习。

给定 \(m \in  \mathbb{Z}\) ,定义序列 \(\left( {x}_{n}\right)\) 为 \({x}_{n} = m - 1/n\) 。由此可得 \(\left( {x}_{n}\right)  \rightarrow  m\) ,但

\[
h\left( {x}_{n}\right)  \rightarrow  \left( {m - 1}\right) ,
\]

这不等于 \(m = h\left( m\right)\) 。根据推论~\ref{cor:4.3.3},我们看到 \(h\) 在每个 \(m \in  \mathbb{Z}\) 处都不连续。

现在让我们看看为什么 \(h\) 在点 \(c \notin  \mathbb{Z}\) 处是连续的。给定 \(\varepsilon  > 0\) ,我们必须找到一个 \(\delta\) -邻域 \({V}_{\delta }\left( c\right)\) ,使得只要 \(x \in  {V}_{\delta }\left( c\right)\),就有 \(h\left( x\right)  \in  {V}_{\varepsilon }\left( {h\left( c\right) }\right)\) 。我们知道 \(c \in  \mathbb{R}\) 落在某个 \(n \in  \mathbb{Z}\) 的连续整数 \(n < c < n + 1\) 之间。如果我们取 \(\delta  = \min \{ c - n,\left( {n + 1}\right)  - c\}\) ,那么根据 \(h\) 的定义, \(\forall x \in  {V}_{\delta }\left( c\right)\) , \(h\left( x\right)  = h\left( c\right)\) 都成立。因此,我们确实有 $\forall x \in  {V}_{\delta }\left( c\right)$

\[
h\left( x\right)  \in  {V}_{\varepsilon }\left( {h\left( c\right) }\right)
\]

这里的证明与一般情况大不相同,因为这里 \(\delta\) 的值实际上并不依赖于 \(\varepsilon\) 的选择。通常,较小的 \(\varepsilon\) 需要较小的 \(\delta\) 来响应,但在这里,无论 \(\varepsilon\) 选择得多小,相同的 \(\delta\) 值都适用。
\end{Eg}


\begin{Eg}
  \label{eg:4.3.8}
考虑定义在 \(A = \{ x \in  \mathbb{R} : x \geq  0\}\) 上的 \(f\left( x\right)  = \sqrt{x}\) 。练习 2.3.2 概述了一个 \(f\) 在 \(A\) 上连续性的证明。在这里,我们用 \(\varepsilon  - \delta\) 语言证明来证明同一事实。

设 \(\varepsilon  > 0\) 。我们需要证明,对于 \(c\) 周围的某个 \(\delta\) 邻域内的所有 \(x\) 值, \(\left| {f\left( x\right)  - f\left( c\right) }\right|\) 可以小于 \(\varepsilon\) 。如果 \(c = 0\) ,这简化为陈述 \(\sqrt{x} < \varepsilon\) ,只要 \(x < {\varepsilon }^{2}\) 成立。因此,选择 \(\delta  = {\varepsilon }^{2}\) ,便有 \(\forall \left| {x - 0}\right|  < \delta, \left| {f\left( x\right)  - 0}\right|  < \varepsilon\) 。

对于一个不同于零的点 \(c \in  A\) ,我们需要估计 \(\left| {\sqrt{x} - \sqrt{c}}\right|\) 。这次,写成

\[
\left| {\sqrt{x} - \sqrt{c}}\right|  = \left| {\sqrt{x} - \sqrt{c}}\right| \left( \frac{\sqrt{x} + \sqrt{c}}{\sqrt{x} + \sqrt{c}}\right)  = \frac{\left| x - c\right| }{\sqrt{x} + \sqrt{c}} \leq  \frac{\left| x - c\right| }{\sqrt{c}}.
\]

为了使这个量小于 \(\varepsilon\) ,只需选择 \(\delta  = \varepsilon \sqrt{c}\) 。然后, 只要 \(\left| {x - c}\right|  < \delta\) 便有

\[
\left| {\sqrt{x} - \sqrt{c}}\right|  < \frac{\varepsilon \sqrt{c}}{\sqrt{c}} = \varepsilon
\]
 得证。
\end{Eg}


尽管我们现在已经证明了多项式和平方根函数都是连续的,但代数连续性定理并未提供所需的理由来断定诸如 \(h\left( x\right)  = \sqrt{3{x}^{2} + 5}\) 这样的函数是连续的。为此,我们必须证明连续函数的复合也是连续的。

\begin{Thm}[连续函数的复合]
  给定 \(f : A \rightarrow \mathbb{R}\) 和 \(g : B \rightarrow  \mathbb{R}\) ,设 \(f\left( A\right)  = \{ f\left( x\right)  : x \in  A\}\) 包含在 \(B\) 中,使得复合函数 \(g \circ  f\left( x\right)  = g\left( {f\left( x\right) }\right)\) 在 \(A\) 上良定义。

如果 \(f\) 在 \(c \in  A\) 处连续,且 \(g\) 在 \(f\left( c\right)  \in  B\) 处连续,则 \(g \circ  f\) 在 \(c\) 处连续。
\end{Thm}



\begin{proof}
  
   由于 \( g \) 在 \( f(c) \) 处连续,对任意 \( \varepsilon > 0 \),存在 \( \gamma > 0 \),使得当 \( |y - f(c)| < \gamma \) 时,有  
   \[
   |g(y) - g(f(c))| < \varepsilon.
   \]

   因为 \( f \) 在 \( c \) 处连续,对上述 \( \gamma > 0 \),存在 \( \delta > 0 \),使得当 \( |x - c| < \delta \) 时,有  
   \[
   |f(x) - f(c)| < \gamma.
   \]

   当 \( |x - c| < \delta \) 时,由 \( f \) 的连续性得 \( |f(x) - f(c)| < \gamma \)。此时令 \( y = f(x) \),则 \( |y - f(c)| < \gamma \),代入 \( g \) 的连续性条件得  
   \[
   |g(f(x)) - g(f(c))| < \varepsilon.
   \]  
   因此,\( g \circ f \) 在 \( c \) 处连续。
\end{proof}


\subsection{练习}

练习4.3.1。设 \(g\left( x\right)  = \sqrt[3]{x}\) 。

(a) 证明 \(g\) 在 \(c = 0\) 处连续。

(b) 证明 \(g\) 在点 \(c \neq  0\) 处连续。(恒等式 \({a}^{3} - {b}^{3} =\)  \(\left( {a - b}\right) \left( {{a}^{2} + {ab} + {b}^{2}}\right)\) 将有所帮助。)

练习 4.3.2. (a) 使用连续性的 \(\varepsilon  - \delta\) 特征为定理 4.3.9 提供证明。

(b) 使用连续性的序列特征(来自定理 4.3.2 (iv))给出该定理的另一个证明。

练习 4.3.3. 使用连续性的 \(\varepsilon  - \delta\) 特征(因此不使用关于序列的先前结果),证明线性函数 \(f\left( x\right)  = {ax} + b\) 在 \(\mathbb{R}\) 的每一点都连续。

练习 4.3.4. (a) 使用定义 4.3.1 证明,任何定义域为 \(\mathbb{Z}\) 的函数 \(f\) 在其定义域的每一点都必然连续。

(b) 证明一般情况下,如果 \(c\) 是 \(A \subseteq  \mathbb{R}\) 的孤立点,那么 \(f : A \rightarrow  \mathbb{R}\) 在 \(c\) 处连续。

练习 4.3.5. 在定理 4.3.4 中,陈述 (iv) 指出,如果 \(f\) 和 \(g\) 都连续,且商有定义,则 \(f\left( x\right) /g\left( x\right)\) 在 \(c\) 处连续。证明如果 \(g\) 在 \(c\) 和 \(g\left( c\right)  \neq  0\) 处连续,则存在一个包含 \(c\) 的开区间,在该区间上 \(f\left( x\right) /g\left( x\right)\) 始终有定义。

习题4.3.6. (a) 参考相关定理,给出一个正式论证,证明第4.1节中的Dirichlet函数(Dirichlet’s function)在 \(\mathbb{R}\) 上处处不连续。

(b) 回顾第4.1节中Thomae函数(Thomae's function)的定义,并证明其在每个有理点处都不连续。

(c) 使用定理4.3.2 (iii)中的连续性特征,证明Thomae函数在 \(\mathbb{R}\) 中的每个无理点处连续。(给定 \(\varepsilon  > 0\) ,考虑点集 \(\{ x \in  \mathbb{R} : t\left( x\right)  \geq  \varepsilon \}\) 。论证该集合中的所有点都是孤立的。)

练习 4.3.7. 假设 \(h : \mathbb{R} \rightarrow  \mathbb{R}\) 在 \(\mathbb{R}\) 上连续,且令 \(K = \{ x\) : \(h\left( x\right)  = 0\}\) 。证明 \(K\) 是一个闭集。

练习 4.3.8. (a) 证明如果一个函数在整个 \(\mathbb{R}\) 上连续,并且在每个有理点上都等于 0,那么它必须在整个 \(\mathbb{R}\) 上恒等于 0。

(b) 如果 \(f\) 和 \(g\) 在整个 \(\mathbb{R}\) 上定义,并且在每个有理点上都满足 \(f\left( r\right)  = g\left( r\right)\) ,那么 \(f\) 和 \(g\) 必须是同一个函数吗?

练习 4.3.9(压缩映射定理)。设 \(f\) 是定义在 \(\mathbb{R}\) 上的函数,并假设存在常数 \(c\) 使得 \(0 < c < 1\) 且

\[
\left| {f\left( x\right)  - f\left( y\right) }\right|  \leq  c\left| {x - y}\right|
\]

对于所有 \(x,y \in  \mathbb{R}\) 。

(a) 证明 \(f\) 在 \(\mathbb{R}\) 上是连续的。

(b) 选取某点 \({y}_{1} \in  \mathbb{R}\) 并构造序列

\[
\left( {{y}_{1},f\left( {y}_{1}\right) ,f\left( {f\left( {y}_{1}\right) }\right) ,\ldots }\right) \text{ . }
\]

一般情况下,如果 \({y}_{n + 1} = f\left( {y}_{n}\right)\) ,证明生成的序列 \(\left( {y}_{n}\right)\) 是Cauchy序列。因此我们可以令 \(y = \lim {y}_{n}\) 。

(c) 证明 \(y\) 是 \(f\) 的不动点(即 \(f\left( y\right)  = y\) ),并且在此方面是唯一的。

最后,证明如果 \(x\) 是 \(\mathbb{R}\) 中的任意一点,则序列 \(\left( {x,f\left( x\right) ,f\left( {f\left( x\right) }\right) ,\ldots }\right)\) 收敛到(b)中定义的 \(y\) 。

习题4.3.10。设 \(f\) 是定义在 \(\mathbb{R}\) 上的函数,满足对所有 \(x,y \in  \mathbb{R}\) 的加性条件 \(f\left( {x + y}\right)  = f\left( x\right)  + f\left( y\right)\) 。

(a) 证明 \(f\left( 0\right)  = 0\) 且对所有 \(x \in  \mathbb{R}\) 有 \(f\left( {-x}\right)  =  - f\left( x\right)\) 。

(b) 证明如果 \(f\) 在 \(x = 0\) 处连续,则 \(f\) 在 \(\mathbb{R}\) 中的每一点都连续。

(c) 设 \(k = f\left( 1\right)\) 。证明对于所有 \(n \in  \mathbb{N}\) , \(f\left( n\right)  = {kn}\) 成立,然后证明对于所有 \(z \in  \mathbb{Z}\) , \(f\left( z\right)  = {kz}\) 成立。现在,证明对于任何有理数 \(r\) , \(f\left( r\right)  = {kr}\) 成立。

(d) 使用 (b) 和 (c) 得出结论,对于所有 \(x \in  \mathbb{R}\) , \(f\left( x\right)  = {kx}\) 成立。因此,任何在 \(x = 0\) 处连续的加性函数必然是通过原点的线性函数。

练习 4.3.11. 对于以下每个 \(A\) 的选择,构造一个函数 \(f : \mathbb{R} \rightarrow  \mathbb{R}\) ,该函数在 \(A\) 中的每个点 \(x\) 处具有不连续性,并且在 \({A}^{c}\) 上连续。

(a) \(A = \mathbb{Z}\) .

(b) \(A = \{ x : 0 < x < 1\}\) .

(c) \(A = \{ x : 0 \leq  x \leq  1\}\) .

(d) \(A = \left\{  {\frac{1}{n} : n \in  \mathbb{N}}\right\}\) .

练习 4.3.12. 设 \(C\) 为第 3.1 节中构造的Cantor集(Cantor set)。定义 \(g : \left\lbrack  {0,1}\right\rbrack   \rightarrow  \mathbb{R}\) 为

\[
g\left( x\right)  = \left\{  \begin{array}{ll} 1 & \text{ if }x \in  C \\  0 & \text{ if }x \notin  C. \end{array}\right.
\]

(a) 证明 \(g\) 在任何点 \(c \in  C\) 处都不连续。

(b) 证明 \(g\) 在每个点 \(c \notin  C\) 处都连续。

\section{紧集上的连续函数}
\label{sec:4.4}
\begin{Def}
  \label{def:4.4.1}
  给定一个函数 \(f : A \rightarrow  \mathbb{R}\) 和一个子集 \(B \subseteq  A\) ,令 \(f\left( B\right)\) 表示 \(f\) 在集合 \(B\) 上的范围;即 \(f\left( B\right)  = \{ f\left( x\right)  : x \in  B\}\) 。如果 \(f\left( A\right)\) 在定义\ref{def:2.3.1}的意义上有界,则我们说 \(f\) 是有界的。对于给定的子集 \(B \subseteq  A\) ,如果 \(f\left( B\right)\) 有界,则我们说 \(f\) 在 \(B\) 上有界。
\end{Def}


开集、闭集、有界集、紧集、完备集和连通集都用于描述实数集的子集。一个有趣的问题是,当一个特定集合 \(A \subseteq  \mathbb{R}\) 通过连续函数映射到 \(f\left( A\right)\) 时,这些性质中哪些(如果有的话)会被保留?例如,如果 \(A\) 是开集且 \(f\) 是连续的,那么 \(f\left( A\right)\) 是否必然也是开集?这个问题的答案是否定的。如果 \(f\left( x\right)  = {x}^{2}\) 且 \(A\) 是开区间$(-1,1)$,那么 \(f\left( A\right)\) 是区间 \(\lbrack 0,1)\) ,它不是开集。

对于闭集的相应猜想也被证明是错误的,尽管构造一个反例需要更多的思考。考虑函数

\[
g\left( x\right)  = \frac{1}{1 + {x}^{2}}
\]

以及闭集 \(A = \lbrack 0,\infty ) = \{ x : x \geq  0\}\) 。由于 \(g\left( A\right)  = (0,1\rbrack\) 不是闭集,我们只得得出结论:连续函数通常不会将闭集映射为闭集。然而,请注意,我们的特定反例需要使用一个无界闭集 \(A\) 。这并非偶然。闭且有界的集合——即紧集——总是被连续函数映射为闭且有界的子集。


\begin{Thm}[紧集的保持性]
  \label{thm:4.4.2}
  设 \(f : A \rightarrow  \mathbb{R}\) 在 \(A\) 上连续。如果 \(K \subseteq  A\) 是紧集,那么 \(f\left( K\right)\) 也是紧集。
\end{Thm}


\begin{proof}
设 \(\left( {y}_{n}\right)\) 为包含在值域 \(f\left( K\right)\) 中的任意序列。为了证明这一结果,我们必须找到一个子序列 \(\left( {y}_{{n}_{k}}\right)\) ,使它收敛到一个也在 \(f\left( K\right)\) 中的极限。我们的策略是利用定义域 \(K\) 是紧集的假设,将值域序列 \(\left( {y}_{n}\right)\) 转换回定义域 \(K\) 中的序列。

断言 \(\left( {y}_{n}\right)  \subseteq  f\left( K\right)\) 意味着, \(\forall n \in  \mathbb{N}\) ,我们可以找到(至少一个) \({x}_{n} \in  K\) 满足 \(f\left( {x}_{n}\right)  = {y}_{n}\) 。这便产生了一个序列 \(\left( {x}_{n}\right)  \subseteq  K\) 。由于 \(K\) 是紧的,存在一个收敛的子序列 \(\left( {x}_{{n}_{k}}\right)\) ,其极限 \(x = \lim {x}_{{n}_{k}}\) 也在 \(K\) 中。最后,我们利用 \(f\) 在 \(A\) 上连续的事实,因此特别在 \(x\) 处连续。鉴于 \(\left( {x}_{{n}_{k}}\right)  \rightarrow  x\) ,我们得出结论 \(\left( {y}_{{n}_{k}}\right)  \rightarrow  f\left( x\right)\) 。由于 \(x \in  K\) ,我们有 \(f\left( x\right)  \in  f\left( K\right)\) ,因此 \(f\left( K\right)\) 是紧的。  
\end{proof}


通过将这一结果与紧集有界且包含其上确界和下确界的观察相结合,可以得到一个极其重要的推论。

\begin{Thm}[极值定理]
  \label{thm:4.4.3}
  如果 \(f : K \rightarrow  \mathbb{R}\) 在紧集 \(K \subseteq  \mathbb{R}\) 上连续,则 \(f\) 达到最大值和最小值。换句话说,存在 \({x}_{0},{x}_{1} \in  K\) 使得对于所有 \(x \in  K\) 有 \(f\left( {x}_{0}\right)  \leq  f\left( x\right)  \leq  f\left( {x}_{1}\right)\) 。
\end{Thm}

\begin{proof}
  由于 \( K \subseteq \mathbb{R} \) 是紧集且 \( f \) 连续,故 \( f(K) \subseteq \mathbb{R} \) 也是紧集。  
在 \(\mathbb{R}\) 中,紧集等价于闭且有界集。因此,\( f(K) \) 是闭且有界的。  
由有界性,\( f(K) \) 存在上确界 \( M \) 和下确界 \( m \)。因 \( f(K) \) 是闭集,故 \( M, m \in f(K) \)。  
存在 \( x_0, x_1 \in K \) 使得 \( f(x_0) = m \) 和 \( f(x_1) = M \)。从而对所有 \( x \in K \),有  
   \[
   f(x_0) \leq f(x) \leq f(x_1),
   \]
   即 \( f \) 在 \( K \) 上达到最小值和最大值。
\end{proof}



\subsection{一致连续性}

尽管我们已经证明了多项式在 \(\mathbb{R}\) 上总是连续的,但通过直接证明函数 \(f\left( x\right)  = {3x} + 1\) 和 \(g\left( x\right)  = {x}^{2}\) (之前在例~\ref{eg:4.2.2}中研究过)处处连续,我们可以学到重要的一课。


\begin{Eg}
  \label{eg:4.4.4}
  \begin{enumerate}[label = (\roman*)]
  \item\label{item:4.4.1}
    为了直接证明 \(f\left( x\right)  = {3x} + 1\) 在任意点 \(c \in  \mathbb{R}\) 处连续,我们必须论证对于当 $x$ 趋近于 $c$ 时, \(\left| {f\left( x\right)  - f\left( c\right) }\right|\) 可以变得任意小。已知:

\[
\left| {f\left( x\right)  - f\left( c\right) }\right|  = \left| {\left( {{3x} + 1}\right)  - \left( {{3c} + 1}\right) }\right|  = 3\left| {x - c}\right| ,
\]

因此,给定 \(\varepsilon  > 0\) ,我们选择 \(\delta  = \varepsilon /3\) 。然后只要 \(\left| {x - c}\right|  < \delta\) 便有

\[
\left| {f\left( x\right)  - f\left( c\right) }\right|  = 3\left| {x - c}\right|  < 3\left( \frac{\varepsilon }{3}\right)  = \varepsilon .
\]

对于本次讨论特别重要的是,无论我们考虑哪个点 \(c \in  \mathbb{R}\) , \(\delta\) 的选择都是相同的。
\item \label{item:4.4.2}让我们将其与证明 \(g\left( x\right)  = {x}^{2}\) 在 \(\mathbb{R}\) 上连续时的情况进行对比。给定 \(c \in  \mathbb{R}\) ,我们有

\[
\left| {g\left( x\right)  - g\left( c\right) }\right|  = \left| {{x}^{2} - {c}^{2}}\right|  = \left| {x - c}\right| \left| {x + c}\right| .
\]

如例~\ref{eg:4.2.2}所讨论的,我们需要 \(\left| {x + c}\right|\) 的一个上界,这是通过坚持我们选择的 \(\delta\) 不超过 $1$ 来获得的。这保证了所有考虑的 \(x\) 值必然落在区间 \(\left( {c - 1,c + 1}\right)\) 内。因此

\[
\left| {x + c}\right|  \leq  \left| x\right|  + \left| c\right|  \leq  \left( {\left| c\right|  + 1}\right)  + \left| c\right|  = 2\left| c\right|  + 1.
\]

现在,令 \(\varepsilon  > 0\) 。如果我们选择 \(\delta  = \min \{ 1,\varepsilon /\left( {2\left| c\right|  + 1}\right) \}\) ,那么 \(\left| {x - c}\right|  < \delta\) 保证了

\[
\left| {f\left( x\right)  - f\left( c\right) }\right|  = \left| {x - c}\right| \left| {x + c}\right|  < \left( \frac{\varepsilon }{2\left| c\right|  + 1}\right) \left( {2\left| c\right|  + 1}\right)  = \varepsilon .
\]

这个论证没有任何缺陷,但需要注意的是,在第二个证明中,选择响应 \(\delta\) 的算法取决于 \(c\) 的值。命题

\[
\delta  = \frac{\varepsilon }{2\left| c\right|  + 1}
\]

意味着更大的 \(c\) 将需要更小的 \(\delta\) 。这一事实从 \(g\left( x\right)  = {x}^{2}\) 的图像(图\ref{fig:4.7})中应该可以明显看出。例如给定 \(\varepsilon  = 1\) , \(\delta  = 1/3\) 的响应对于 \(c = 1\) 是足够的,因为 \(2/3 < x < 4/3\) 肯定意味着 \(0 < {x}^{2} < 2\) 。然而,如果 \(c = {10}\) ,那么 \(g\left( x\right)\) 图的陡峭意味着需要更小的 \(\delta\) 来强制 \({99} < {x}^{2} < {101}\) 。

\begin{figure}[h]
  \centering
  \includegraphics[width=0.5\textwidth]{images/01955a91-1881-740a-ab38-d94129e5d318_18_487_379_664_616_0.jpg}
  \caption{ \(g\left( x\right)  = {x}^{2}\) ;较大的 \(c\) 需要较小的 \(\delta\) }
  \label{fig:4.7}
\end{figure}

  \end{enumerate}
\end{Eg}

下一个定义旨在区分这两个例子。

\begin{Def}
  \label{def:4.4.5}
   函数 \(f : A \rightarrow  \mathbb{R}\) 在 \(A\) 上一致连续,如果对于每一个 \(\varepsilon  > 0\) ,存在一个 \(\delta  > 0\) ,使得 \(\left| {x - y}\right|  < \delta\Rightarrow \left| {f\left( x\right)  - f\left( y\right) }\right|  < \varepsilon\) 。
\end{Def}


回忆:称“ \(f\) 在 \(A\) 上连续”意味着 \(f\) 在每个单独的点 \(c \in  A\) 上连续。换句话说,给定 \(\varepsilon  > 0\) 和 \(c \in  A\) ,我们可以找到一个可能依赖于 \(c\) 的 \(\delta  > 0\) ,使得只要 \(\left| {x - c}\right|  < \delta\) 便有 \(\left| {f\left( x\right)  - f\left( c\right) }\right|  < \varepsilon\) 。一致连续性是一个更强的性质。 \(f\) “在 \(A\) 上一致连续”与仅仅“在 \(A\) 上连续”之间的关键区别在于,给定一个 \(\varepsilon  > 0\) ,可以选择一个单一的 \(\delta  > 0\) ,它同时适用于 \(A\) 中的所有点 \(c\) 。称一个函数在集合 \(A\) 上不一致连续,并不一定意味着它在某一点上不连续。相反,这意味着存在某个 \({\varepsilon }_{0} > 0\) ,对于所有 \(c \in  A\) ,没有一个一致的 \(\delta  > 0\) 是合适的响应。

\begin{Thm}[非一致连续性的序列准则]
  \label{thm:4.4.6}
称函数 \(f : A \rightarrow  \mathbb{R}\) 在 \(A\) 上不一致连续,如果存在特定的 \({\varepsilon }_{0} > 0\) 和 \(A\) 中的两个序列 \(\left( {x}_{n}\right)\) 和 \(\left( {y}_{n}\right)\) 满足

\[
\left| {{x}_{n} - {y}_{n}}\right|  \rightarrow  0, \quad\left| {f\left( {x}_{n}\right)  - f\left( {y}_{n}\right) }\right|  \geq  {\varepsilon }_{0}.
\]  
\end{Thm}

\begin{proof}
  设存在 \( \varepsilon_0 > 0 \) 和序列 \( \{x_n\}, \{y_n\} \subseteq A \),满足 \( |x_n - y_n| \rightarrow 0 \) 但 \( |f(x_n) - f(y_n)| \geq \varepsilon_0 \)。
  
  对任意 \( \delta > 0 \),存在 \( N\in \mathbb{N} \) 使得当 \( n \geq N \) 时,\( |x_n - y_n| < \delta \),但此时仍有 \( |f(x_n) - f(y_n)| \geq \varepsilon_0 \)。
  
这表明对于该 \( \varepsilon_0 \),无论 \( \delta \) 多小,总存在 \( x_n, y_n \) 使得 \( |x_n - y_n| < \delta \) 但 \( |f(x_n) - f(y_n)| \geq \varepsilon_0 \),故 \( f \) 在 \( A \) 上不一致连续。  

反之,若 \( f \) 不一致连续,则存在 \( \varepsilon_0 > 0 \),对任意 \( n \in \mathbb{N} \),存在 \( x_n, y_n \in A \) 使得 \( |x_n - y_n| < \frac{1}{n} \) 但 \( |f(x_n) - f(y_n)| \geq \varepsilon_0 \)。此时 \( |x_n - y_n| \rightarrow 0 \),满足定义条件。  
\end{proof}


\begin{Eg}
  \label{eg:4.4.7}
函数 \(h\left( x\right)  = \sin \left( {1/x}\right)\) (图\ref{fig:4.5})在开区间$(0,1)$内的每一点都连续,但在该区间上不是一致连续的。问题出现在接近零的地方,那里越来越快的振荡使得定义域上非常接近的值被映射到值域上相距$2$的值。为了说明定理~\ref{thm:4.4.6},取 \({\varepsilon }_{0} = 2\) 并设

\[
{x}_{n} = \frac{1}{\pi /2 + {2n\pi }}\;\text{ and }\;{y}_{n} = \frac{1}{{3\pi }/2 + {2n\pi }}.
\]

由于这些序列都趋于零,我们有 \(\left| {{x}_{n} - {y}_{n}}\right|  \rightarrow  0\) ,并且通过简短的计算可以得出 \(\forall n \in  \mathbb{N}\) , \(\left| {h\left( {x}_{n}\right)  - h\left( {y}_{n}\right) }\right|  = 2\) 。
\end{Eg}

虽然连续性是在单点定义的,但一致连续性总是针对特定域进行讨论。在例~\ref{eg:4.4.4}中,我们无法证明 \(g\left( x\right)  = {x}^{2}\) 在 \(\mathbb{R}\) 上是一致连续的,因为越大的 \(x\) 值需要越小的 \(\delta\) 值。(作为定理~\ref{thm:4.4.6}的另一个例证,取 \({x}_{n} = n\) 和 \({y}_{n} = n + 1/n\) 。)然而, \(g\left( x\right)\) 在有界集 \(\left\lbrack  {-{10},{10}}\right\rbrack\) 上是一致连续的。回到例~\ref{eg:4.4.4}~\ref{item:4.4.2}中提出的论点,注意如果我们将注意力限制在定义域 \(\left\lbrack  {-{10},{10}}\right\rbrack\) 上,那么对于任意的 \(x\) 和 \(y\) , \(\left| {x + y}\right|  \leq  {20}\) 都成立。给定 \(\varepsilon  > 0\) ,我们现在可以选择 \(\delta  = \varepsilon /{20}\) ,并验证如果 \(x,y \in  \left\lbrack  {-{10},{10}}\right\rbrack\) 满足 \(\left| {x - y}\right|  < \delta\) ,那么

\[
\left| {f\left( x\right)  - f\left( y\right) }\right|  = \left| {{x}^{2} - {y}^{2}}\right|  = \left| {x - y}\right| \left| {x + y}\right|  < \left( \frac{\varepsilon }{20}\right) {20} = \varepsilon .
\]

事实上,不难看出如何修改这个论证以证明 \(g\left( x\right)\) 在 \(\mathbb{R}\) 中的任何有界集 \(A\) 上是一致连续的。

然而考虑到例~\ref{eg:4.4.7},“有界集上连续的函数必然是一致连续的”并不是正确的结论。然而,如果我们假设定义域是紧的,那么相应的一般性的结果确实成立。

\begin{Thm}
  \label{thm:4.4.8}
  在紧致集 \(K\) 上连续的函数在 \(K\) 上是一致连续的。
\end{Thm}

\begin{proof}
设 \(f : K \rightarrow  \mathbb{R}\) 在紧致集 \(K \subseteq  \mathbb{R}\) 的每一点都连续。为了证明 \(f\) 在 \(K\) 上是一致连续的,我们采用反证法。

根据定理~\ref{thm:4.4.6}中的准则,如果 \(f\) 在 \(K\) 上不是一致连续的,那么存在 \(K\) 中的两个序列 \(\left( {x}_{n}\right)\) 和 \(\left( {y}_{n}\right)\) ,使得 $\exists \varepsilon_0 > 0$

\[
\lim \left| {{x}_{n} - {y}_{n}}\right|  = 0\;\text{ 且 }\;\left| {f\left( {x}_{n}\right)  - f\left( {y}_{n}\right) }\right|  \geq  {\varepsilon }_{0}
\]

由于 \(K\) 是紧的,序列 \(\left( {x}_{n}\right)\) 有一个收敛的子列 \(\left( {x}_{{n}_{k}}\right)\) ,且 \(x = \lim {x}_{{n}_{k}}\) 也在 \(K\) 中。

我们可以再次利用 \(K\) 的紧性来生成 \(\left( {y}_{n}\right)\) 的收敛子序列,但请注意当我们考虑由 \(\left( {y}_{n}\right)\) 中与收敛子序列 \(\left( {x}_{{n}_{k}}\right)\) 对应的项组成的特定子序列 \(\left( {y}_{{n}_{k}}\right)\) 时会发生什么。根据代数极限定理,

\[
\lim \left( {y}_{{n}_{k}}\right)  = \lim \left( {\left( {{y}_{{n}_{k}} - {x}_{{n}_{k}}}\right)  + {x}_{{n}_{k}}}\right)  = 0 + x.
\]

结论是 \(\left( {x}_{{n}_{k}}\right)\) 和 \(\left( {y}_{{n}_{k}}\right)\) 都收敛到 \(x \in  K\) 。据假设, \(f\) 在 \(x\) 处连续,我们有 \(\lim f\left( {x}_{{n}_{k}}\right)  = f\left( x\right)\) 和 \(\lim f\left( {y}_{{n}_{k}}\right)  =\)  \(f\left( x\right)\) ,这意味着

\[
\lim \left( {f\left( {x}_{{n}_{k}}\right)  - f\left( {y}_{{n}_{k}}\right) }\right)  = 0.
\]

当我们回忆起 \(\left( {x}_{n}\right)\) 和 \(\left( {y}_{n}\right)\) 是如何被选出时,矛盾就出现了:$\forall n\in \mathbb{N}$

\[
\left| {f\left( {x}_{n}\right)  - f\left( {y}_{n}\right) }\right|  \geq  {\varepsilon }_{0}
\]

因此,我们得出结论, \(f\) 在 \(K\) 上确实是一致连续的。
  
\end{proof}

\subsection{练习}

练习 4.4.1. (a) 证明 \(f\left( x\right)  = {x}^{3}\) 在 \(\mathbb{R}\) 上连续。

(b) 使用定理 4.4.6 论证 \(f\) 在 \(\mathbb{R}\) 上不是一致连续的。(c) 证明 \(f\) 在 \(\mathbb{R}\) 的任何有界子集上是一致连续的。

练习 4.4.2. 证明 \(f\left( x\right)  = 1/{x}^{2}\) 在集合 \(\lbrack 1,\infty )\) 上是一致连续的,但在集合 \((0,1\rbrack\) 上不是。

练习 4.4.3. 为极值定理(定理 4.4.3)的证明提供细节(包括对练习 3.3.1 的论证,如果尚未完成)。

练习4.4.4. 证明如果 \(f\) 在 \(\left\lbrack  {a,b}\right\rbrack\) 上连续,且对于所有 \(a \leq  x \leq  b\) 有 \(f\left( x\right)  > 0\) ,则 \(1/f\) 在 \(\left\lbrack  {a,b}\right\rbrack\) 上有界。

练习4.4.5. 使用定理4.4.6后的建议,为非一致连续性的这一准则提供一个完整的证明。

练习4.4.6. 给出以下每种情况的例子,或者说明这种请求是不可能的。对于任何不可能的情况,提供一个简短的解释(可能引用适当的定理)说明原因。

(a) 一个连续函数 \(f : \left( {0,1}\right)  \rightarrow  \mathbb{R}\) 和一个Cauchy序列 \(\left( {x}_{n}\right)\) ,使得 \(f\left( {x}_{n}\right)\) 不是Cauchy序列;

(b) 一个连续函数 \(f : \left\lbrack  {0,1}\right\rbrack   \rightarrow  \mathbb{R}\) 和一个Cauchy序列 \(\left( {x}_{n}\right)\) ,使得 \(f\left( {x}_{n}\right)\) 不是Cauchy序列;

(c) 一个连续函数 \(f : \lbrack 0,\infty ) \rightarrow  \mathbb{R}\) 和一个Cauchy序列 \(\left( {x}_{n}\right)\) ,使得 \(f\left( {x}_{n}\right)\) 不是Cauchy序列;

(d) 在(0,1)上的一个有界连续函数 \(f\) ,它在这个开区间上达到最大值但没有达到最小值。

练习 4.4.7. 假设 \(g\) 定义在开区间 (a, c) 上,并且已知在 \((a,b\rbrack\) 和 \(\lbrack b,c)\) 上一致连续,其中 \(a < b < c\) 。证明 \(g\) 在 (a, c) 上一致连续。

练习 4.4.8. (a) 假设 \(f : \lbrack 0,\infty ) \rightarrow  \mathbb{R}\) 在其定义域的每一点都连续。证明如果存在 \(b > 0\) 使得 \(f\) 在集合 \(\lbrack b,\infty )\) 上一致连续,那么 \(f\) 在 \(\lbrack 0,\infty )\) 上一致连续。

(b) 证明 \(f\left( x\right)  = \sqrt{x}\) 在 \(\lbrack 0,\infty )\) 上一致连续。

练习4.4.9。一个函数 \(f : A \rightarrow  \mathbb{R}\) 被称为利普希茨(Lipschitz)函数,如果存在一个

界 \(M > 0\) ,使得

\[
\left| \frac{f\left( x\right)  - f\left( y\right) }{x - y}\right|  \leq  M
\]

对于所有 \(x,y \in  A\) 。从几何上讲,一个函数 \(f\) 是利普希茨函数,如果在函数 \(f\) 的图像上任意两点所画直线的斜率大小有一个统一的界。

(a) 证明如果 \(f : A \rightarrow  \mathbb{R}\) 是利普希茨函数,那么它在 \(A\) 上是一致连续的。

(b) 逆命题是否成立?所有一致连续函数是否必然是利普希茨函数?

练习4.4.10。一致连续函数是否保持有界性?如果 \(f\) 在有界集 \(A\) 上是一致连续的,那么 \(f\left( A\right)\) 是否必然有界?

练习 4.4.11(连续性的拓扑特征)。设 \(g\) 定义在 \(\mathbb{R}\) 的所有点上。如果 \(A\) 是 \(\mathbb{R}\) 的子集,定义集合 \({g}^{-1}\left( A\right)\) 为

\[
{g}^{-1}\left( A\right)  = \{ x \in  \mathbb{R} : g\left( x\right)  \in  A\} .
\]

证明 \(g\) 是连续的当且仅当 \(O \subseteq  \mathbb{R}\) 是开集时 \({g}^{-1}\left( O\right)\) 也是开集。

练习 4.4.12。使用定理 3.3.8 (iii) 中紧性的开覆盖特征,构造定理 4.4.8 的另一种证明。

练习4.4.13(连续延拓定理)。(a) 证明一致连续函数保持Cauchy序列;即,如果 \(f : A \rightarrow  \mathbb{R}\) 是一致连续的且 \(\left( {x}_{n}\right)  \subseteq  A\) 是Cauchy序列,则证明 \(f\left( {x}_{n}\right)\) 是Cauchy序列。

(b) 设 \(g\) 是开区间(a, b)上的连续函数。证明 \(g\) 在(a, b)上一致连续当且仅当可以在端点处定义值 \(g\left( a\right)\) 和 \(g\left( b\right)\) ,使得延拓后的函数 \(g\) 在 \(\left\lbrack  {a,b}\right\rbrack\) 上连续。(在正向方向上,首先为 \(g\left( a\right)\) 和 \(g\left( b\right)\) 生成候选值,然后证明延拓后的 \(g\) 是连续的。)

\section{介值定理}
\label{sec:4.5}
介值定理(Intermediate Value Theorem, IVT)是对一个非常直观的观察结果的命名,即一个在闭区间 \(\left\lbrack  {a,b}\right\rbrack\) 上的连续函数 \(f\) 会取得介于范围值 \(f\left( a\right)\) 和 \(f\left( b\right)\) 之间的每一个值(图\ref{fig:4.8})。

\begin{figure}[h]
  \centering
  \includegraphics[width=0.3\textwidth]{images/01955a91-1881-740a-ab38-d94129e5d318_22_573_379_491_492_0.jpg}
  \caption{介值定理}
  \label{fig:4.8}
\end{figure}


以下是用分析语言表述的这一观察结果。

\begin{Thm}[介值定理]
  \label{thm:4.5.1}
  如果 \(f : \left\lbrack  {a,b}\right\rbrack   \rightarrow  \mathbb{R}\) 是连续的,并且如果 \(L\) 是满足 \(f\left( a\right)  < L < f\left( b\right)\) 或 \(f\left( a\right)  > L >\)  \(f\left( b\right)\) 的实数,那么存在一个点 \(c \in  \left( {a,b}\right)\) ,使得 \(f\left( c\right)  = L\) 。
\end{Thm}


18世纪的数学家(包括Euler和Gauss)自由地使用了这一定理,而并未考虑其有效性。事实上,直到1817年,Bolzano才在一篇论文中首次提供了分析证明,该论文还首次出现了某种现代意义上的连续性定义。这凸显了这一结果的重要性。正如第\ref{sec:4.1}节所讨论的,Bolzano及其同时代人在数学发展的过程中,已经到了一个越来越需要夯实学科基础的阶段。然而,这样做并不仅仅是回溯历史并补上缺失的证明。真正的挑战在于首先获得对相关概念的深入且相互认可的理解。介值定理对我们的重要性与此类似,因为我们对连续性和实数性质的理解已经足够成熟,使得证明成为可能。事实上,对于这一简单结果,有几种令人满意的论证,每一种都以略微不同的方式,隔离了连续性和完备性之间的相互作用。

\subsection{连通集的保持}

理解介值定理最具潜在价值的方式是将其视为“连续函数将连通集映射为连通集”这一事实的特例。在定理~\ref{thm:4.4.2}中,我们看到如果 \(f\) 是紧集 \(K\) 上的连续函数,那么值域集 \(f\left( K\right)\) 也是紧的。类似的观察也适用于连通集。


\begin{Thm}
  \label{thm:4.5.2}
  设 \(f : A \rightarrow  \mathbb{R}\) 为连续函数。如果 \(E \subseteq  A\) 是连通的,那么 \(f\left( E\right)\) 也是连通的。
\end{Thm}

\begin{proof}
为了使用定理\ref{thm:3.4.6}中连通集的刻画,设 \(f\left( E\right)  = A \cup  B\) ,其中 \(A\) 和 \(B\) 是互不相交且非空的。我们的目标是构造一个包含在其中一个集合中的序列,使其收敛到另一个集合中的极限。

设

\[
C = \{ x \in  E : f\left( x\right)  \in  A\} , \quad D = \{ x \in  E : f\left( x\right)  \in  B\} .
\]

集合 \(C\) 和 \(D\) 分别被称为 \(A\) 和 \(B\) 的原像。利用 \(A\) 和 \(B\) 的性质,可以很容易地验证 \(C\) 和 \(D\) 是非空且不相交的,并且满足 \(E = C \cup  D\) 。现在,我们假设 \(E\) 是一个连通集,因此根据定理 \ref{thm:3.4.6},存在一个序列 \(\left( {x}_{n}\right)\) ,使得它包含在 \(C\) 或 \(D\) 中的一个之中, 其极限 \(x = \lim {x}_{n}\) 却包含在另一个之中。由于 \(f\) 在 \(x\) 处连续,我们得到 \(f\left( x\right)  = \lim f\left( {x}_{n}\right)\) 。因此,可以得出 \(f\left( {x}_{n}\right)\) 是一个包含在 \(A\) 或 \(B\) 中的收敛序列,而极限 \(f\left( x\right)\) 是另一个集合的元素。再次引用定理 \ref{thm:3.4.6},得证。
\end{proof}

在 \(\mathbb{R}\) 中,一个集合是连通的当且仅当它是一个(可能无界的)区间。这一事实与定理~\ref{thm:4.5.2}一起,为介值定理(练习4.5.1)提供了一个简短的证明。我们应该指出,定理~\ref{thm:4.5.2}的证明并没有利用 \(\mathbb{R}\) 中连通集与区间之间的等价性,而是仅依赖于一般定义。之前的评论认为这是处理中值定理最有用的方法,原因在于,尽管这里没有讨论,但连续性和连通性的定义可以很容易地适应高维设置。因此,定理~\ref{thm:4.5.2}在高维中仍然是一个有效的结论,而中值定理本质上是一个一维的结果。

\subsection{完备性}

介值定理的一个典型应用是证明根的存在性。例如,给定 \(f\left( x\right)  = {x}^{2} - 2\) ,我们看到 \(f\left( 1\right)  =  - 1\) 和 \(f\left( 2\right)  = 2\) 。因此,存在一个点 \(c \in  \left( {0,1}\right)\) ,其中 \(f\left( c\right)  = 0\) 。

在这种情况下,我们可以轻松计算 \(c = \sqrt{2}\) ,这意味着我们实际上并不需要介值定理来证明 \(f\) 存在根。我们在第一章中花费了大量时间证明 \(\sqrt{2}\) 的存在,这只有在我们坚持将完备性公理作为关于实数的假设之一时才可能实现。介值定理刚刚断言 \(\sqrt{2}\) 存在的事实表明,理解这一结果的另一种方式是通过 \(f\) 的连续性与 \(\mathbb{R}\) 的完备性之间的关系。

第一章中的完备性公理指出“有上界的集合存在最小上界。”后来,我们看到闭区间套定理是断言实数没有“间隙”的等价方式。这些完备性的特征中的任何一个都可以作为定理~\ref{thm:4.5.1}的替代证明的基石。


\begin{figure}[h]
  \centering
  \includegraphics[width=0.4\textwidth]{images/01955a91-1881-740a-ab38-d94129e5d318_24_544_496_548_314_0.jpg}
\end{figure}

\begin{proof}
  法一:使用完备性公理。

  为了简化问题,我们考虑 \(f\) 是满足 \(f\left( a\right)  < 0 < f\left( b\right)\) 的连续函数的特殊情况。要证明 \(\exists c \in  \left( {a,b}\right)\)  使得  \(f\left( c\right)  = 0\) 成立。首先设

\[
K = \{ x \in  \left\lbrack  {a,b}\right\rbrack   : f\left( x\right)  \leq  0\} .
\]

注意到 \(K\) 被上界 $b$ 所限制,且 \(a \in  K\) ,因此 \(K\) 非空。因此我们可以诉诸完备性公理来断言 \(c = \sup K\) 存在。

有三种情况需要考虑:

\begin{itemize}
\item $f(c)>0$:此时由连续性,存在 $ \delta > 0 $,使得当 $ x \in (c - \delta, c + \delta) $ 时,$ f(x) > 0 $。但 $ c $ 是 $ K $ 的上确界,故存在 $ x \in K $ 满足 $ x > c - \delta $,此时 $ f(x) \leq 0 $,与 $ f(x) > 0 $ 矛盾。因此 $ f(c) \not> 0 $。
\item $f(c)<0$:此时由连续性,存在 $ \delta > 0 $,使得当 $ x \in (c - \delta, c + \delta) $ 时,$ f(x) < 0 $。取 $ x = \min(c + \delta/2, b) $,因 $ c < b $,当 $ \delta $ 足够小时,$ x \in (c, b) $。此时 $ f(x) < 0 $,故 $ x \in K $,与 $ c = \sup K $ 矛盾。因此 $ f(c) \not< 0 $。
\item $f(c) = 0$:这是唯一的可能,得证。
\end{itemize}
 

法二:使用闭区间套定理。

为了简化问题,我们考虑 \(f\) 是满足 \(f\left( a\right)  < 0 < f\left( b\right)\) 的连续函数的特殊情况。要证明 \(\exists c \in  \left( {a,b}\right)\)  使得  \(f\left( c\right)  = 0\) 成立。

再次考虑特殊情况,其中 \(L = 0\) 和 \(f\left( a\right)  < 0 < f\left( b\right)\) 。设 \({I}_{0} = \left\lbrack  {a,b}\right\rbrack\) ,并考虑中点

\[
z = \left( {a + b}\right) /2
\]

如果 \(f\left( z\right)  \geq  0\) ,则设 \({a}_{1} = a\) 和 \({b}_{1} = z\) 。如果 \(f\left( z\right)  < 0\) ,则设 \({a}_{1} = z\) 和 \({b}_{1} = b\) 。无论哪种情况,区间 \({I}_{1} = \left\lbrack  {{a}_{1},{b}_{1}}\right\rbrack\) 都具有在左端点处为负且在右端点处为非负的性质。


接下来重复上述二分法,构造区间序列 \(\{I_n\} = [a_n, b_n]\),其中每个区间长度为 \((b - a)/2^n\),且满足 \(f(a_n) < 0\) 和 \(f(b_n) \geq 0\)。根据闭区间套定理,存在唯一一点 \(c \in \bigcap_{n=0}^\infty I_n \subseteq (a, b)\),且 \(\lim_{n \to \infty} a_n = c\) 和 \(\lim_{n \to \infty} b_n = c\)。由于 \(f\) 连续,有  
\[
f(c) = \lim_{n \to \infty} f(a_n) \leq 0 \quad \text{且} \quad f(c) = \lim_{n \to \infty} f(b_n) \geq 0,
\]  
故 \(f(c) = 0\),即 \(c\) 为所求零点。


\begin{figure}[h]
  \centering
  \includegraphics[width=0.4\textwidth]{images/01955a91-1881-740a-ab38-d94129e5d318_24_543_1477_550_353_0.jpg}
\end{figure}


\end{proof}


\subsection{介值性}

介值定理是否有逆定理?

\begin{Def}
  \label{def:4.5.3}
  称一个函数 \(f\) 在区间 \(\left\lbrack  {a,b}\right\rbrack\) 上具有介值性,如果对于 \(\left\lbrack  {a,b}\right\rbrack\) 中的所有 \(x < y\) 和介于 \(f\left( x\right)\) 与 \(f\left( y\right)\) 之间的所有 \(L\) ,总能找到一个点 \(c \in  \left( {x,y}\right)\) ,使得 \(f\left( c\right)  = L\) 。
\end{Def}


另一种总结介值定理的方式是说,每个在 \(\left\lbrack  {a,b}\right\rbrack\) 上的连续函数都具有介值性。一种合理的猜想是:任何具有介值性质的函数必然连续,但事实并非如此。我们已经看到

\[
g\left( x\right)  = \left\{  \begin{array}{ll} \sin \left( {1/x}\right) & x \neq  0 \\  0 & x = 0 \end{array}\right.
\]

在零点不连续(例~\ref{eg:4.2.6}),但它在 \(\left\lbrack  {0,1}\right\rbrack\) 上确实具有介值性质。

如果我们坚持函数是单调的,那么介值性质确实意味着连续性(练习4.5.4)。

\subsection{练习}

练习4.5.1。展示介值定理如何作为定理4.5.2的推论得出。

练习4.5.2。判断以下猜想的有效性。

连续函数将有限开区间映射为有限开区间。

(b) 连续函数将有限开区间映射为开集。

(c) 连续函数将有限闭区间映射为有限闭区间。

练习 4.5.3. 是否存在一个定义在整个 \(\mathbb{R}\) 上的连续函数,其值域 \(f\left( \mathbb{R}\right)\) 等于 \(\mathbb{Q}\) ?

练习 4.5.4. 如果对于 \(A\) 中的所有 \(x < y\) ,都有 \(f\left( x\right)  \leq  f\left( y\right)\) ,则函数 \(f\) 在 \(A\) 上是递增的。证明如果我们假设 \(f\) 在 \(\left\lbrack  {a,b}\right\rbrack\) 上是递增的,则介值定理确实有逆定理。

练习 4.5.5. 使用之前开始的完备性公理完成介值定理的证明。

练习 4.5.6. 使用之前开始的嵌套区间性质完成介值定理的证明。

练习 4.5.7. 设 \(f\) 是闭区间 \(\left\lbrack  {0,1}\right\rbrack\) 上的连续函数,其值域也包含在 \(\left\lbrack  {0,1}\right\rbrack\) 中。证明 \(f\) 必须有一个不动点;也就是说,证明对于至少一个 \(x \in  \left\lbrack  {0,1}\right\rbrack\) 的值, \(f\left( x\right)  = x\) 成立。

练习 4.5.8. 想象一个钟表,其中时针和分针无法区分。假设指针在钟表表面上连续移动,并且假设它们的位置可以精确测量,是否总是可以确定时间?

\section{不连续点集}
\label{sec:4.6}
给定一个函数 \(f : \mathbb{R} \rightarrow  \mathbb{R}\) ,定义 \({D}_{f} \subseteq  \mathbb{R}\) 为函数 \(f\) 不连续的点集。在第~\ref{sec:4.1}节中,我们看到Dirichlet函数 \(g\left( x\right)\) 有 \({D}_{g} = \mathbb{R}\) 。Dirichlet函数的修改 \(h\left( x\right)\) 有 \({D}_{h} = \mathbb{R} \smallsetminus  \{ 0\}\) ,$0$是唯一的连续点。最后,对于Thomae函数 \(t\left( x\right)\) ,我们看到 \({D}_{t} = \mathbb{Q}\) 。

练习 4.6.1。使用这些函数的修改,构造一个函数 \(f : \mathbb{R} \rightarrow  \mathbb{R}\) ,使得

(a) \({D}_{f} = \mathbb{Z}\) .

(b) \({D}_{f} = \{ x : 0 < x \leq  1\}\) .

我们在引言部分以一个关于 \({D}_{f}\) 是否可以取任意实数子集形式的问题作为结尾。事实证明,情况并非如此。在 \(\mathbb{R}\) 上定义的实值函数的间断点集具有特定的拓扑结构,这种结构并非 \(\mathbb{R}\) 的每个子集都具备。具体来说,无论 \(f\) 如何选择, \({D}_{f}\) 总可以表示为闭集的可数并。当 \(f\) 是单调函数时,这些闭集可以取为单点集。

\subsection{单调函数}

对于任意 \(f\) 分类 \({D}_{f}\) 较为复杂。然而对于单调函数类,描述 \({D}_{f}\) 却相当直接。这是很有趣的事实。


\begin{Def}
  \label{def:4.6.1}
  称函数 \(f : A \rightarrow  \mathbb{R}\) 在 \(A\) 上单调递增,如果 \(x < y\Rightarrow f\left( x\right)  \leq  f\left( y\right)\) ;在 \(A\) 上单调递减,如果 \(x < y\Rightarrow f\left( x\right)  \geq  f\left( y\right)\) 。单调函数是指要么递增要么递减的函数。
\end{Def}


函数 \(f\) 在点 \(c\) 处的连续性意味着 \(\mathop{\lim }\limits_{{x \rightarrow  c}}f\left( x\right)  = f\left( c\right)\) 。一种特定的制造不连续性发生的方式是,使得 \(c\) 处的右极限与 \(c\) 处的左极限不同。与所有新术语一样,我们需要精确地定义“从左”和“从右”的含义。

\begin{Def}
  \label{def:4.6.2}
  给定集合 \(A\) 的极限点 \(c\) 和函数 \(f : A \rightarrow  \mathbb{R}\) ,称 $f(x)$ 在 $c$ 处的右极限为 $L$,若 \(\forall \varepsilon  > 0\) , \(\exists \delta  > 0\) ,使得每当 \(0 < x - c < \delta\) 时, \(\left| {f\left( x\right)  - L}\right|  < \varepsilon\) 成立。记作
\[
\mathop{\lim }\limits_{{x \rightarrow  {c}^{ + }}}f\left( x\right)  = L
\]
\end{Def}


等价地,用序列表示。如果对于所有满足 \({x}_{n} > c\) 和 \(\lim \left( {x}_{n}\right)  = c\) 的序列 \(\left( {x}_{n}\right)\) , \(\lim f\left( {x}_{n}\right)  = L\) 成立,则 \(\mathop{\lim }\limits_{{x \rightarrow  {c}^{ + }}}f\left( x\right)  = L\) 成立。


同理:
\addtocounter{Thm}{-1}
\begin{Def}
  给定集合 \(A\) 的极限点 \(c\) 和函数 \(f : A \rightarrow \mathbb{R}\),称 \(f(x)\) 在 \(c\) 处的左极限为 \(L\),若 \(\forall \varepsilon > 0\),\(\exists \delta > 0\),使得每当 \(0 < c - x < \delta\) 时,\(\left| f(x) - L \right| < \varepsilon\) 成立。记作  
\[
\mathop{\lim }\limits_{{x \rightarrow {c}^{-}}} f(x) = L.
\]
\end{Def}


等价地,用序列表示。如果对于所有满足 \({x}_{n} < c\) 和 \(\lim \left( {x}_{n} \right) = c\) 的序列 \(\left( {x}_{n} \right)\),\(\lim f\left( {x}_{n} \right) = L\) 成立,则 \(\mathop{\lim }\limits_{{x \rightarrow {c}^{-}}} f(x) = L\) 成立。


\begin{Thm}
  \label{thm:4.6.3}
给定 \(f : A\rightarrow  \mathbb{R}\) 和 \(A\)的极限点 \(c\),\(\mathop{\lim }\limits_{{x \rightarrow  c}}f\left( x\right)  = L\)  当且仅当

\[
\mathop{\lim }\limits_{{x \rightarrow  {c}^{ + }}}f\left( x\right)  = L\;\text{ and }\;\mathop{\lim }\limits_{{x \rightarrow  {c}^{ + }}}f\left( x\right)  = L.
\]  
\end{Thm}

\begin{proof}
 必要性:设 \(\lim_{x \to c} f(x) = L\),则对任意 \(\varepsilon > 0\),存在 \(\delta > 0\),使得当 \(x \in A\) 且 \(0 < |x - c| < \delta\) 时,\(|f(x) - L| < \varepsilon\)。特别地:当 \(c < x < c + \delta\) 时,右极限 \(|f(x) - L| < \varepsilon\),故 \(\lim_{x \to c^+} f(x) = L\);当 \(c - \delta < x < c\) 时,左极限 \(|f(x) - L| < \varepsilon\),故 \(\lim_{x \to c^-} f(x) = L\)。

充分性:设 \(\lim_{x \to c^+} f(x) = L\) 且 \(\lim_{x \to c^-} f(x) = L\)。对任意 \(\varepsilon > 0\):存在 \(\delta_1 > 0\),使得当 \(x \in A\) 且 \(c < x < c + \delta_1\) 时,\(|f(x) - L| < \varepsilon\);存在 \(\delta_2 > 0\),使得当 \(x \in A\) 且 \(c - \delta_2 < x < c\) 时,\(|f(x) - L| < \varepsilon\)。

取 \(\delta = \min\{\delta_1, \delta_2\}\),则当 \(x \in A\) 且 \(0 < |x - c| < \delta\) 时,无论 \(x > c\) 或 \(x < c\),均有 \(|f(x) - L| < \varepsilon\),故 \(\lim_{x \to c} f(x) = L\)。

因此,原极限存在的充要条件是左右极限均存在且等于 \(L\)。
\end{proof}


一般来说,不连续性可以分为三类:

\begin{enumerate}
\item\label{item:4.6.1}  如果 \(\mathop{\lim }\limits_{{x \rightarrow  c}}f\left( x\right)\) 存在但其值与 \(f\left( c\right)\) 不同,则在 \(c\) 处的不连续性称为可去的。
\item \label{item:4.6.2} 如果 \(\mathop{\lim }\limits_{{x \rightarrow  {c}^{ + }}}f\left( x\right)  \neq  \mathop{\lim }\limits_{{x \rightarrow  {c}^{ - }}}f\left( x\right)\) ,则 \(f\) 在 \(c\) 处有一个跳跃不连续性。
\item \label{item:4.6.3} 如果由于其他原因 \(\mathop{\lim }\limits_{{x \rightarrow  c}}f\left( x\right)\) 不存在,那么在 \(c\) 处的不连续性称为本质不连续性。
\end{enumerate}


我们现在已经准备好描述任意单调函数 \(f\) 的集合 \({D}_{f}\) 的特征。

练习4.6.4。设 \(f : \mathbb{R} \rightarrow  \mathbb{R}\) 为递增函数。证明 \(\mathop{\lim }\limits_{{x \rightarrow  {c}^{ + }}}f\left( x\right)\) 和 \(\mathop{\lim }\limits_{{x \rightarrow  {c}^{ - }}}f\left( x\right)\) 必须在每个点 \(c \in  R\) 处存在。论证单调函数唯一可能具有的不连续性类型是跳跃不连续性。

练习4.6.5。构造一个单调函数 \(f\) 的跳跃间断点集与 \(\mathbb{Q}\) 的一个子集之间的双射。得出结论:单调函数 \(f\) 的 \({D}_{f}\) 必须是有限的或可数的,但不能是不可数的。



\subsection{任意函数的 \({D}_{f}\)}

回想一下,无限个闭集的交集是闭的,但对于并集,我们必须限制为有限个闭集的并集,以确保并集是闭的。对于开集,情况则相反。任意个开集的并集是开的,但只有有限个开集的交集才必然是开的。

\begin{Def}
  \label{def:4.6.4}
  一个可以写成可数个闭集并集的集合属于 \({F}_{\sigma }\) 类。(此定义也出现在第\ref{sec:3.5}节中。)
\end{Def}

到目前为止,我们已经构造了不连续点集为 \(\mathbb{R}\) (Dirichlet函数)、 \(\mathbb{R} \smallsetminus  \{ 0\}\) (修正Dirichlet函数)、 \(\mathbb{Q}\) (Thomae函数)、 \(\mathbb{Z}\) 和 \((0,1\rbrack\) (习题4.6.1)的函数。

习题4.6.6。证明在每种情况下,我们得到一个 \({F}_{\sigma }\) 集作为每个函数的不连续点集。

即将进行的论证依赖于一个称为 \(\alpha\) -连续性的概念。

\begin{Def}
  \label{def:4.6.5}
  设 \(f\) 定义在 \(\mathbb{R}\) 上,且令 \(\alpha  > 0\) 。函数 \(f\) 在 \(x \in  \mathbb{R}\) 处是 \(\alpha\) -连续的,如果存在一个 \(\delta  > 0\) ,使得对于所有 \(y,z \in  \left( {x - \delta ,x + \delta }\right)\) ,都有 \(\left| {f\left( y\right)  - f\left( z\right) }\right|  < \alpha\) 。
\end{Def}

关于这个定义,最重要的是要注意在 \(\alpha  > 0\) 前面没有“对于所有”。正如我们将要探讨的,添加这个量词会使这个定义等同于我们对连续性的定义。从某种意义上说, \(\alpha\) -连续性是对函数在特定点附近变化的度量。如果存在一个以 \(c\) 为中心的区间,在该区间内函数的变化从未超过值 \(\alpha  > 0\) ,则函数在点 \(c\) 处是 \(\alpha\) -连续的。

给定 \(\mathbb{R}\) 上的函数 \(f\) ,定义 \({D}_{\alpha }\) 为函数 \(f\) 不满足 \(\alpha\) -连续的点集。换句话说,

\[
{D}_{\alpha } = \{ x \in  \mathbb{R} : f \text{ 在 }x  \text{处不} \alpha-  \text{连续} \} .
\]

练习4.6.7。证明对于固定的 \(\alpha  > 0\) ,集合 \({D}_{\alpha }\) 是闭集。

舞台已搭好。是时候刻画任意函数 \(f\) 在 \(\mathbb{R}\) 上的不连续点集了。

\begin{Thm}
  \label{thm:4.6.6}
  设 \(f : \mathbb{R} \rightarrow  \mathbb{R}\) 为任意函数。则 \({D}_{f}\) 是一个 \({F}_{\sigma }\) 集。
\end{Thm}

\begin{proof}
  回忆

\[
{D}_{f} = \{ x \in  \mathbb{R} : f\text{ is not continuous at }x\} .
\]

习题4.6.8。如果 \({\alpha }_{1} < {\alpha }_{2}\) ,证明 \({D}_{{\alpha }_{2}} \subseteq  {D}_{{\alpha }_{1}}\) 。

习题4.6.9。设 \(\alpha  > 0\) 给定。证明如果 \(f\) 在 \(x\) 处连续,则它在 \(x\) 处也是 \(\alpha\) -连续的。解释由此如何得出 \({D}_{\alpha } \subseteq  {D}_{f}\) 。

习题4.6.10。证明如果 \(f\) 在 \(x\) 处不连续,则 \(f\) 对于某个 \(\alpha  > 0\) 不是 \(\alpha\) -连续的。现在解释为什么这保证了

\[
{D}_{f} = \mathop{\bigcup }\limits_{{n = 1}}^{\infty }{D}_{\frac{1}{n}}
\]

因为每个 \({D}_{\frac{1}{n}}\) 都是闭集,证明完成。
\end{proof}



\section{结语}
\label{sec:4.7}
定理~\ref{thm:4.6.6}只有在能够证明并非 \(\mathbb{R}\) 的每个子集都属于 \({F}_{\sigma }\) 集时才有意义。这需要一些努力,并且作为练习包含在关于 Baire 定理的第\ref{sec:3.5}节中。Baire 定理指出,如果 \(\mathbb{R}\) 被写成闭集的可数并集,那么这些闭集中至少有一个必须包含一个非空开区间。现在 \(\mathbb{Q}\) 是单点集的可数并集,我们可以将每个点视为一个显然不包含任何区间的闭集。如果无理数集 \(\mathbb{R}\setminus\mathbb{Q}\) 是闭集的可数并集,那么这些闭集中没有一个可以包含任何开区间,否则它们将包含一些有理数。但这与Baire定理相矛盾。因此, \(\mathbb{R}\setminus\mathbb{Q}\) 不是闭集的可数并集,从而它也不是 \({F}_{\sigma }\) 集。因此,我们可以得出结论,不存在在每一个有理点连续且在每一个无理点间断的函数 \(f\) 。这应与之前讨论的Thomae函数进行比较。

相反的问题也很有趣。给定一个任意的 \({F}_{\sigma }\) 集合,W.H. Young 在1903年证明了总可以构造一个函数,使其不连续点恰好位于该集合上。他的构造涉及我们之前见过的相同的Dirichlet型定义,但显然更为复杂。相比之下,描述单调情况下的相应函数并不太困难。设

\[
D = \left\{  {{x}_{1},{x}_{2},{x}_{3},{x}_{4},\ldots }\right\}
\]

为任意可数的实数集合。为了构造一个不连续点恰好位于 \(D\) 上的单调函数,直观上为每个点 \({x}_{n} \in  D\) 附加一个 \(1/{2}^{n}\) 的“权重”。现在,定义

\[
f\left( x\right)  = \mathop{\sum }\limits_{{n : {x}_{n} < x}}\frac{1}{{2}^{n}}
\]

其中对于每个 \(x \in  \mathbb{R}\) ,求和应涵盖所有对应于 \(x\) 左侧点的权重。(如果 \(D\) 中没有点在 \(x\) 的左侧,则设置 \(f\left( x\right)  = 0\) 。)任何关于求和顺序的担忧都可以通过观察到收敛是绝对的来缓解。不难证明,所得函数 \(f\) 是单调的,并且在 \(D\) 中的每个点 \({x}_{n}\) 处具有大小为 \(1/{2}^{n}\) 的跳跃间断,如预期(练习6.4.8)。

